\section{Discrete Dynamical Systems}

\begin{definition}[Discrete dynamical system]
    A \emph{discrete dynamical system} (abbr. DDS) is a tuple $\left( X, f \right)$ where X is a non-empty set and $f: X \rightarrow X$ is a map~\cite{Brin20100706}.
    Set X is called a \emph{state space}.
    The set X is \emph{ invariant} under $f$ if and only if $f(X) \subseteq X$, the set $X$ is \emph{strongly invariant} if and only if $f(X) = X$.
    If map $f$ depends on a parameter $p$, it is denoted as $f_p$.
\end{definition}

\begin{remark}[Logistic map]
    \textcolor{red}{
    The dynamical system that will be used in most of the examples throughout the text is called the Logistic map.
    This map might be considered the most widely known discrete dynamical system [cite here].
    Although it is mathematically very simple, it exhibits a wide range of complex behavior.
    The Logistic map was popularized by May~\cite{May19760610}.
    Originally, the map was used to represent population dynamics in an environment with limited resources.
    }
    \par
    \textcolor{red}{
    Mathematically, Logistic map can be thought of as a DDS of the form $\left( [0, 1], \mathbb{L}_{r} \right)$ where $\mathbb{L}_{r}: [0,1] \rightarrow [0,1]$, $\mathbb{L}_{r}(x) = rx(1-x)$.
    To ensure that $x$ remains in the range $[0,1]$, the parameter $r$ is chosen from the range $r \in [0, 4]$.
    }
    \par
    \textcolor{red}{
    As mentioned earlier, Logistic map was used as a simple model for population dynamics.
    The variable $x$ represents a population of some species.
    $x$ is understood as a normalized population.
    $1$ means that the population is at its maximum and $0$ means that the population is extinct.
    The parameter $r$ represents the growth rate of the population in each generation. [cite here]
    }
\end{remark}

\begin{definition}[\textit{$\mathbf{n}$}-th iteration]
    Let $\left( X, f \right)$ be a DDS. 
    The first iteration of the map $f$ for the initial condition $x_0$ is represented by equation
    \begin{eqnarray}
        x_{1}  & = & f(x_{0})
    \end{eqnarray}
    $n$-th iteration of $x_0$ is
    \begin{equation}
    x_{n} = f^{n}(x_0) =
        \begin{cases}
            f^{0}(x_0) = x_0 & \text{if } n = 0. \\
            \underbrace{f \circ f \circ \cdots \circ f}_\text{$n$ times}(x_0) & \text{if } n > 0. 
        \end{cases}
    \end{equation}
    Therefore, $f^{n}$ is the $n$-fold composition of the map f.
\end{definition}

\begin{definition}[Graph]
\textcolor{red}{
    Let $\left( X, f \right)$ be a DDS. A graph of $f$ is a plot of $f(x)$ for each $x \in X$.
}
\end{definition}

\begin{remark}
    \textcolor{red}{
    An example of a graph of $\mathbb{L}_{r}$ is shown in Figure~\ref{fig:logistic_graph_example}.
    }
    \par
    \textcolor{red}{
    It is sometimes useful to plot the graph of $n$-th iteration of $f$.
     Figure~\ref{fig:logistic_nthcomp_example} shows an example of the graph of $n$-th iteration of $\mathbb{L}_{3.8}$ for different $n$.
    }
\end{remark}

\begin{figure}[!h]
    \centering
    \includegraphics[width=0.5\textwidth]{DDS/Figures/logistic_graph_example.png}
    \caption{
        \textcolor{red}{
        Graphs of the Logistic map $\mathbb{L}_{r}$ for different parameters $r$.
        }
    }
    \label{fig:logistic_graph_example}
\end{figure}

\begin{figure}[!h]
    \centering
    \includegraphics[width=1.0\textwidth]{DDS/Figures/logistic_nthcomp_example.png}
    \caption{
        \textcolor{red}{
        Graphs of the Logistic map $\mathbb{L}_{3.8}^{n}$ for different $n$.
        }
    }
    \label{fig:logistic_nthcomp_example}
\end{figure}

\begin{definition}[Trajectory]
    Let $\left( \mathbb{R}, f \right)$ be a DDS. Successive iterations of $f$ with initial condition $x_0 \in X$ form a trajectory $T(f, x_0)$ where \begin{eqnarray}
        T(f, x_0) = f^0(x_0), f^1(x_0), f^2(x_0), f^3(x_0), \cdots  = x_0, x_1, x_2, x_3, \cdots
    \end{eqnarray}
    More precisely, the trajectory between iterates $n$ and $m$, $n \leq m$, is
    \begin{eqnarray}
        T_{n}^{m}(f, x_0) = f^{n}(x_0), f^{n+1}(x_0), \cdots, f^{m-1}(x_0), f^{m}(x_0) = x_n, \cdots, x_m
    \end{eqnarray}
\end{definition}

\begin{remark}
    \textcolor{red}{
    Generating a trajectory is the most basic tool for understanding a DDS.
    Plotting the trajectory is a simple way to visualize the short-term behavior of the system.
    The graph of a trajectory shows, for example, whether the system is periodic, chaotic, or intermittent. (Intermittency will be explained later.)
    }
    \par
    \textcolor{red}{
    An example of trajectories of the Logistic map is shown in Figure~\ref{fig:trajectory_example}.
    For each trajectory, a projection onto a single line is also provided.
    Understanding how projections relate to the original trajectories is important to understand other tools for analyzing a DDS.
    }
\end{remark}

\begin{figure}[!h]
    \centering
    \begin{subfigure}{0.6\textwidth}
        \centering
        \includegraphics[width=\textwidth]{DDS/Figures/logistic_map_trajectory_example1.png}
        \caption{}
    \end{subfigure}
    \hfill
    \begin{subfigure}{0.6\textwidth}
        \centering
        \includegraphics[width=\textwidth]{DDS/Figures/logistic_map_trajectory_example2.png}
        \caption{}
    \end{subfigure}

    \caption{
        \textcolor{red}{ 
        An example of trajectories and their projections onto a line. 
        Both cases (a) and (b) show a trajectory $T_{50}^{100}(\mathbb{L}_{r}, 0.5)$ (left) and a projection of the trajectory onto a single line (right). 
        Case~(a) exhibits chaotic behavior.
        Case~(b) exhibits $3$-periodic behavior.
        Parameters: 
        (a) $r = 3.828$. 
        (b) $r = 3.829$. 
        }
    }
    \label{fig:trajectory_example}
\end{figure}

\label{def:fixed point}
\begin{definition}[Fixed point]
    Let $\left( \mathbb{R}, f \right)$ be a DDS. Point $x \in X$ is called a \emph{fixed point} of $f$ if it holds $f(x) = x$.
    The set of all fixed point of $f$ is denoted as $\emph{Fix}(f)$~\cite{Devaney20211026}.
\end{definition}

\label{def:sfp}
\begin{definition}[Stable fixed point]
    Let $\left(\mathbb{R}, f\right)$ be a DDS, $f:\mathbb{R} \rightarrow \mathbb{R}$. Let $f$ be differentiable on $X$. Fixed point $x \in X$ is \emph{stable} if $|f'(x)| < 1$.
    This point is called \emph{attracting}.
\end{definition}

\label{def:ufp}
\begin{definition}[Unstable fixed point]
    Let $\left(\mathbb{R}, f\right)$ be a DDS, $f:\mathbb{R} \rightarrow \mathbb{R}$. Let $f$ be differentiable on $X$. Fixed point $x \in X$ is \emph{unstable} if $|f'(x)| > 1$.
    This point is called \emph{repelling}.
\end{definition}

\begin{remark}
    \textcolor{red}{
        It is worth noting what the conditions from Definitions~\ref{def:sfp} and \ref{def:ufp} stand for.
        The point of stability detection is linear approximation, that is replacing the original function by its tangent in the proximity of the fixed point.
        This approach can be generalized to $n$-dimensional system, that is functions $f: \mathbb{R}^{n} \rightarrow \mathbb{R}^{n}$.
        It is done by evaluating absolute values of eigenvalues of the Jacobian matrix of $f$ at the fixed point.
    }
    \par
    \textcolor{red}{
    Direct evaluation of $\mathbb{L}_{r}(x)=x$ yields Fix$(\mathbb{L}_{r}) = \{ 0, (r-1)/r \}$.
    By evaluating the derivative of $\mathbb{L}_{r}$ at these points, stability of the fixed point can be determined.
    Hence, the fixed point $0$ is attracting if $r \in (0, 1)$ and repelling if $r \in (1, 4)$.
    The fixed point $(r-1)/r$ is attracting if $r \in (1, 3)$ and repelling if $r \in (0, 1) \cup (3, 4) $.
    }
    \par
    \textcolor{red}{
    Figure~\ref{fig:fixed_points_stability_example} displays two fixed points of the Logistic map $\mathbb{L}_{r}$.
    One fixed point is attracting and one is repelling. Test for stability is portrayed graphically.
    }
\end{remark}

\begin{figure}[!h]
    \centering
    \begin{subfigure}{0.495\textwidth}
        \centering
        \includegraphics[width=\textwidth]{DDS/Figures/fixed_points_stability_example{1}.png}
        \caption{}
    \end{subfigure}
    \hfill
    \begin{subfigure}{0.495\textwidth}
        \centering
        \includegraphics[width=\textwidth]{DDS/Figures/fixed_points_stability_example{2}.png}
        \caption{}
    \end{subfigure}

    \caption{
        \textcolor{red}{ 
        Two fixed points of the Logistic map $\mathbb{L}_{r}$ and their stability. 
        The red color represents stability. 
        The red fixed point is attracting.
        The blue color represents instability. 
        The blue fixed point is repelling.
        Parameters: 
        (a) $r = 2.3$. 
        (b) $r = 3.3$. 
        }
    }
    \label{fig:fixed_points_stability_example}
\end{figure}

\begin{definition}[Periodic point]
    Let $\left( \mathbb{R}, f\right)$ be a DDS. The point $x \in X$ is called \emph{$n$-periodic} if $f^{n}(x)=x$.
    The set of all $n$-periodic points of $f$ is denoted as $\emph{Per}_{n}(f)$.~\cite{Devaney20211026}
\end{definition}

\begin{definition}[Periodic orbit]
    Let $\left(\mathbb{R}, f\right)$ be a DDS. Let $f$ be differentiable on $X$. For the periodic point $x_0$ the set $\{f^{n}(x_0):n \in \mathbb{N}\}$ is called a periodic orbit.
    If $x_0$ is attracting, periodic orbit is called stable.
    If $x_0$ is repelling, periodic orbit is called unstable (abbr. UPO).
\end{definition}

\begin{theorem}
    Let $\left(\mathbb{R}, \mathbb{L}_r\right)$ be a DDS. Then there exists at most one stable periodic orbit for each $r$.~\cite[p.~74]{Devaney20211026}
\end{theorem}

\begin{proof}
    The proof can be found in~\cite[p.~74]{Devaney20211026}.
\end{proof}

\label{def:cobweb}
\begin{remark}[Cobweb diagram]
    \textcolor{red}{
    Cobweb diagrams are a visual tools for understanding long term behavior of a DDS.
    They are closely related to the trajectory of a DDS and the projection of the trajectory onto a line.
    Furthermore, they are useful, for example, for determining whether a fixed point is attracting or repelling.
    }
    \par
    \textcolor{red}{
    An example of cobweb diagrams is given in Figure~\ref{fig:cobweb_diag_example}.
    Projections onto a line are also provided.
    Note the similarity between Figures~\ref{fig:trajectory_example} and~\ref{fig:cobweb_diag_example}.
    }
    \par
    \textcolor{red}{
    Pseudocode~\ref{cobweb_alg} illustrates how to construct a cobweb diagram algorithmically.
    }
\end{remark}

\begin{figure}[!h]
    \centering
    \begin{subfigure}{0.6\textwidth}
        \centering
        \includegraphics[width=\textwidth]{DDS/Figures/logistic_map_cobweb_diag_example1.png}
        \caption{}
    \end{subfigure}
    \hfill
    \begin{subfigure}{0.6\textwidth}
        \centering
        \includegraphics[width=\textwidth]{DDS/Figures/logistic_map_cobweb_diag_example2.png}
        \caption{}
    \end{subfigure}

    \caption{
        \textcolor{red}{
        Cobweb diagrams of a trajectory $T_{50}^{100}(\mathbb{L}_{r}, 0.5)$ and a projection of the trajectory onto a line. 
        Dashed diagonal line represents the identity function. 
        Parameters: 
        (a) $r = 3.828$. 
        (b) $r = 3.829$. 
        }
        }
    \label{fig:cobweb_diag_example}
\end{figure}


\begin{algorithm}
\caption{Cobweb Diagram Construction}\label{cobweb_alg}
\begin{algorithmic}[1]
\Statex $f \gets$ map
\Statex $x_0 \gets$ initial state
\Statex $p \gets$ parameter of $f$
\Statex $x_{range} \gets$ range of $x$ values
\Statex $n \gets$ number of iterations

\State plot graph of $f_p(x)$ for $x \in x_{range}$
\State plot identity $g(x)=x$ for $x \in x_{range}$
\For{$i$ from $0$ to $n$}
\State plot line from $(f_{p}^{i}(x_0), f_{p}^{i}(x_0))$ to $(f_{p}^{i}(x_0), f_{p}^{i+1}(x_0))$
\State plot line from $(f_{p}^{i}(x_0), f_{p}^{i+1}(x_0))$ to $(f_{p}^{i+1}(x_0), f_{p}^{i+1}(x_0))$
\EndFor

\end{algorithmic}
\end{algorithm}

\label{def:bifurcation}
\begin{definition}[Bifurcation]
    Let $\left( X, f_p \right)$ be a DDS.
    \emph{Bifurcation} occurs when the qualitative behavior of $f_p$ changes as the parameter $p$ slightly changes.
\end{definition}

\label{def:bifurcation_point}
\begin{definition}[Bifurcation point]
    Let $\left( X, f_p \right)$ be a DDS.
    \emph{Bifurcation point} is a value of a parameter $p$ at which bifurcation occurs.
    The qualitative behavior of the system changes when the parameter is moved slightly around the bifurcation point.
\end{definition}

% \label{def:saddle_node_bif}
% \begin{definition} \textbf{(Saddle-node bifurcation)} \\
%     To be defined.
% \end{definition}

\label{def: bif_diag} 
\begin{remark}[Bifurcation diagram]
    \textcolor{red}{
    Bifurcation diagram is a tool for the long-term global analysis of a DDS.
    It shows how the qualitative behavior of an arbitrary system $f_P(x_0)$ changes for varying parameter P.
    It helps to identify stable regions, chaotic regions, and intermittent regions. (Intermittency will be explained later.)
    The bifurcation diagram is especially useful for identifying bifurcations.
    }
    \par
    \textcolor{red}{
    An example of a bifurcation diagram is given in Figure~\ref{fig:bif_diag_example}.
    Notice the similarity to Figures~\ref{fig:trajectory_example} and~\ref{fig:cobweb_diag_example}, especially the projections onto a line.
    The bifurcation diagram is generated as an union of these projections for various parameters.
    Projections fall into one of two categories:
    \begin{itemize}
        \item{A set of isolated points. This case corresponds to the periodic trajectory.}
        \item{A band of points. This is related to more complex phenomena such as chaos or quasi-periodicity.}
    \end{itemize}
    }
    \par
    \textcolor{red}{
    Algorithmic construction of a bifurcation diagram is illustrated in Pseudocode~\ref{bif_diag_alg}.
    }
\end{remark}

\begin{figure}[!h]
    \centering
    \includegraphics[width=0.9\textwidth]{DDS/Figures/logistic_map_bif_diag_example.png}
    \caption{
        \textcolor{red}{
        An example of the bifurcation diagram. 
        Diagram is constructed for Logistic map $\mathbb{L}_{r}$ for parameters $r \in \langle 2.81, 3.87 \rangle$. 
        Parameter interval is sampled to $1000$ points. 
        Trajectories are iterated for $5000$ iterations and last $500$ iterations are plotted.
        }
    }
    \label{fig:bif_diag_example}
\end{figure}

\begin{algorithm}
\caption{Bifurcation Diagram Construction}\label{bif_diag_alg}
\begin{algorithmic}[1]
\Statex $f \gets$ map
\Statex $x_0 \gets$ initial condition
\Statex $p_{range} \gets$ range of parameters for which trajectory will be computed
\Statex $n_{total} \gets$ trajectories will be iterated for $n_{total}$ iterations
\Statex $n_{last} \gets$ $n_{last}$ iterations of a trajectory will be plotted

\For{parameter $p$ in $p_{range}$}
    \State $data \leftarrow T_{n_{total}-n_{last}}^{n_{total}}(f_p, x_0)$
    \For{$y$ in $data$}
        \State plot point $(p, y)$ onto canvas
    \EndFor
\EndFor

\end{algorithmic}
\end{algorithm}

\begin{remark}
    \textcolor{red}{
    In Figure~\ref{fig:bif_diag_example} it is visible that small change of a parameter can cause qualitatively different behavior of the DDS.
    When the qualitative change occurs, it is said that bifurcation has occurred.
    An example of a bifurcation point is $r = 1+\sqrt{8} \approx 3.82842$. For this $r$ the $\mathbb{L}_{r}$ changes its behavior from seemingly chaotic to $3$-periodic. 
    Another example of a bifurcation point is $r \approx 3.446$. For this $r$ the $\mathbb{L}_{r}$ changes its behavior from $2$-periodic to $4$-periodic.
    }
\end{remark}