\section{Discrete Dynamical Systems}

\begin{definition}[Discrete dynamical system]
    A discrete dynamical system (abbr. DDS) is a tuple (X, f) where X is a non-empty set and $f: X \rightarrow X$ is a map \cite{Brin20100706}.
    Set X is called a state space.
    Set X is invariant under $f$ if and only if $f(X) \subseteq X$, moreover set $X$ is strongly inveriant if and only if $f(X) = X$.
    Map $f$ is called an evolution rule. 
    If map $f$ depends on a parameter $p$, we denote it as $f_p$.
\end{definition}

\begin{example}
   Throughout this thesis we will focus on the so called Logistic Map. 
   It is defined as a DDS $([0, 1], \mathbb{L}_{r})$ where $\mathbb{L}_{r}: [0,1] \rightarrow [0,1]$, $\mathbb{L}_{r}(x) = rx(1-x)$ for $r \in [0, 4]$.
\end{example}

\begin{definition}[n-th iteration]
    One iteration of map $f$ for initial condition $x_0$ is represented by equation
    \begin{eqnarray}
        x_{1}  & = & f(x_{0})
    \end{eqnarray}
    $n$-th iteration of $x_0$ is
    \begin{equation}
    x_{n} = f^{n}(x_0) =
        \begin{cases}
            x_0 & \text{if } n = 0\\
            \underbrace{f \circ f \circ \cdots \circ f}_\text{$n$ times}(x_0) & \text{if } n > 0 
        \end{cases}
    \end{equation}
    Hence $f^{n}$ is $n$-fold composition of map f.
\end{definition}


\begin{definition}[Trajectory]
    Let $(X, f)$ be a DDS. Successive iterates of $f$ with initial condition $x_0 \in X$ form a trajectory $T(f, x_0)$ where
    \begin{eqnarray}
        T(f, x_0) = f^0(x_0), f^1(x_0), f^2(x_0), f^3(x_0), \cdots  = x_0, x_1, x_2, x_3, \cdots
    \end{eqnarray}
    For $n \leq m$ we define
    \begin{eqnarray}
        T_{n}^{m}(f, x_0) = x_n, \cdots, x_m
    \end{eqnarray}
\end{definition}

\begin{remark}
    An Example of a trajectory is shown in Figure \ref{fig:trajectory_example}.
\end{remark}

% \begin{figure}[!h]
%     \centering
%     \includegraphics[width=1.0\textwidth]{DDS/Figures/logistic_map_trajectory_example.png}
%     \caption{Trajectory $T_{0}^{80}(\mathbb{L}_p, 0.5)$ for logistic map $\mathbb{L}$ with parameter $p=3.841$}
%     \label{fig:trajectory_example}
% \end{figure}

\begin{figure}[!h]
    \centering
    \begin{subfigure}{0.495\textwidth}
        \centering
        \includegraphics[width=\textwidth]{DDS/Figures/logistic_map_trajectory_example1.png}
        \caption{$T_{0}^{80}(\mathbb{L}_{3.85} ,0.5)$}
    \end{subfigure}
    \hfill
    \begin{subfigure}{0.495\textwidth}
        \centering
        \includegraphics[width=\textwidth]{DDS/Figures/logistic_map_trajectory_example2.png}
        \caption{$T_{0}^{80}(\mathbb{L}_{3.86} ,0.5)$}
    \end{subfigure}

    \caption{Example trajectories of logistic map $\mathbb{L}$.}
    \label{fig:trajectory_example}
\end{figure}

\label{def:fixed point}
\begin{definition}[Fixed point]
    Let $(X, f)$ be a DDS. Point $x \in X$ is called a fixed point of $f$ if it holds $f(x) = x$.
    We denote set of all fixed points of $f$ as $Fix(f)$ \cite{Devaney20211026}.
\end{definition}

\begin{definition}[Stable fixed point]
    Let $(X \subseteq \mathbb{R}, f)$ be a DDS. Let $f$ be differentiable on $X$. Fixed point $x \in X$ is stable if $|\frac{d{f(x)}}{dx}| < 1$.
    We call this point attracting.
\end{definition}

\begin{definition}[Unstable fixed point]
    Let $(X \subseteq \mathbb{R}, f)$ be a DDS. Let $f$ be differentiable on $X$. Fixed point $x \in X$ is unstable if $|\frac{d{f(x)}}{dx}| > 1$.
    We call this point repelling.
\end{definition}

\begin{definition}[Periodic point]
    Let $(X, f)$ be a DDS. Point $x \in X$ is called $n$-periodic if $f^{n}(x)=x$.
    We denote set of all $n$-periodic points of $f$ as $Per_{n}(f)$. \cite{Devaney20211026}
\end{definition}

\begin{definition}[Periodic orbit]
    Let $(X \subseteq \mathbb{R}, f)$ be a DDS. Let $f$ be differentiable on $X$. For periodic point $x_0$ the set $\{f^{n}(x_0):n \in \mathbb{N}\}$ is called periodic orbit.
    If $x_0$ is attracting, we call periodic orbit stable (abbr. PO).
    If $x_0$ is repelling, we call periodic orbit unstable (abbr. UPO).
\end{definition}

\begin{theorem}
    Let ($\mathbb{R}$, $\mathbb{L}_r$) be a DDS. Then there exists at most one stable periodic orbit for each $r$. \cite[p.~74]{Devaney20211026}
\end{theorem}

\begin{proof}
    Proof can be found in \cite[p.~74]{Devaney20211026}.
\end{proof}

\label{def:cobweb}
\begin{remark}[Cobweb diagram]
    Cobweb diagrams are a visual tools for understanding long term behavior of a DDS.
    Pseudocode \ref{cobweb_alg} illustrates construction of cobweb diagram.
    Example of cobweb diagram can be seen in Figure \ref{fig:cobweb_diag_example}.
\end{remark}

\begin{algorithm}
\caption{Cobweb Diagram Construction}\label{cobweb_alg}
\begin{algorithmic}[1]
\Statex \textbf{parameter description:}
\Statex $f$: evolution rule
\Statex $x_0$: initial state
\Statex $p$: parameter of $f$
\Statex $x_{range}$: range of $x$ values
\Statex $n$: number of iterations
\Statex

\Function{CobwebDiag}{$f$, $x_0$, $p$, $x_{range}$, $n$}
\State plot graph of $f_p(x)$ for $x \in x_{range}$
\State plot identity $g(x)=x$ for $x \in x_{range}$
\For{$i$ from $0$ to $n$}
\State plot line from $(f_{p}^{i}(x_0), f_{p}^{i}(x_0))$ to $(f_{p}^{i}(x_0), f_{p}^{i+1}(x_0))$
\State plot line from $(f_{p}^{i}(x_0), f_{p}^{i+1}(x_0))$ to $(f_{p}^{i+1}(x_0), f_{p}^{i+1}(x_0))$
\EndFor
\EndFunction

\end{algorithmic}
\end{algorithm}

% \begin{figure}[!h]
%     \centering
%     \includegraphics[width=0.8\textwidth]{DDS/Figures/logistic_map_cobweb_diag_example.png}
%     \caption{CobwebDiag($f = \mathbb{L}^2$, $x_0 = 0.6$, $p = 3.81$, $x_{range} = [0, 1]$, $n = 5$)}
%     \label{fig:cobweb_diag_example}
% \end{figure}

\begin{figure}[!h]
    \centering
    \begin{subfigure}{0.495\textwidth}
        \centering
        \includegraphics[width=\textwidth]{DDS/Figures/logistic_map_cobweb_diag_example1.png}
        \caption{$f = \mathbb{L}^2$, $x_0 = 0.6$, $p = 3.6$, $x_{range} = [0, 1]$, $n = 50$}
    \end{subfigure}
    \hfill
    \begin{subfigure}{0.495\textwidth}
        \centering
        \includegraphics[width=\textwidth]{DDS/Figures/logistic_map_cobweb_diag_example2.png}
        \caption{$f = \mathbb{L}^2$, $x_0 = 0.6$, $p = 3.81$, $x_{range} = [0, 1]$, $n = 50$}
    \end{subfigure}

    \caption{Examples of cobweb diagram for second iterate of logistic map $\mathbb{L}^2$}
    \label{fig:cobweb_diag_example}
\end{figure}

\label{def:bifurcation}
\begin{definition}[Bifurcation]
    Bifurcation occurs when qualiative behavior of DDS $f$ changes as parameter varies.
\end{definition}

\label{def:bifurcation_point}
\begin{definition}[Bifurcation point]
    Parameter at which \hyperref[def:bifurcation]{bifurcation} occurs.
\end{definition}

% \label{def:saddle_node_bif}
% \begin{definition} \textbf{(Saddle-node bifurcation)} \\
%     To be defined.
% \end{definition}

\label{def: bif_diag} 
\begin{remark}[Bifurcation diagram]
    Bifurcation diagrams are visual representations of a DDS.
    They show us how qualitative behavior of $f_P(x_0)$ change for varying parameters P.
    Bifurcation diagrams illustrate \hyperref[def:bifurcation]{bifurcations} and help us identify \hyperref[def:bifurcation_point]{bifurcation points}.
    Construction of a bifurcation diagram is illustrated in Pseudocode \ref{bif_diag_alg}.
    Example of a bifurcation diagram can be seen in Figure \ref{fig:bif_diag_example}.
\end{remark}

\begin{algorithm}
\caption{Bifurcation Diagram Construction}\label{bif_diag_alg}
\begin{algorithmic}[1]
\Statex \textbf{parameter description:}
\Statex $f$: evolution rule
\Statex $x_0$: initial condition
\Statex $p_{range}$: range of parameters for which trajectory will be computed
\Statex $n_{total}$: trajectories will be iterated for $n_{total}$ iterations
\Statex $n_{last}$: $n_{last}$ iterations of a trajectory will be plotted
\Statex

\Function{BifDiag}{$f$, $x_0$, $p_{range}$, $n_{total}$, $n_{last}$}
\For{parameter $p$ in $p_{range}$}
    \State $data \leftarrow T_{n_{total}-n_{last}}^{n_{total}}(f_p, x_0)$
    \For{$y$ in $data$}
        \State plot point $(p, y)$ onto canvas
    \EndFor
\EndFor
\EndFunction

\end{algorithmic}
\end{algorithm}

\begin{figure}[!h]
    \centering
    \includegraphics[width=0.9\textwidth]{DDS/Figures/logistic_map_bif_diag_example.png}
    \caption{BifDiag($f = \mathbb{L}$, $x_0 = 0.5$, $p_{range} = [2.81,3.87]$ sampled to $1000$ points, $n_{total} = 5000$, $n_{last} = 500$)}
    \label{fig:bif_diag_example}
\end{figure}

% More precise theoretical background can be found at \cite{Devaney20211026,Elaskar2017,Lynch2014}.