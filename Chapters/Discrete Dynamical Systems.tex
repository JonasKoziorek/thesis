\chapter{Discrete Dynamical Systems}

This chapter introduces the reader to the theory of discrete dynamical systems.
Several standard definitions are provided and explained graphically or computationally.
Two common visualization tools are described and their algorithms are provided.

\section{Introduction to Theory}

\begin{definition}[Discrete dynamical system]
    A \emph{discrete dynamical system} (abbr. DDS) is a tuple $\left( X, f \right)$ where X is a nonempty set and $f: X \rightarrow X$ is a map~\emph{\cite{Brin20100706}}.
    Set X is called a \emph{state space}.
    The set X is \emph{invariant} under $f$ if and only if $f(X) \subseteq X$, the set $X$ is \emph{strongly invariant} if and only if $f(X) = X$.
    If map $f$ depends on a parameter $p$, it is denoted as $f_p$.
\end{definition}

\begin{remark}[Logistic map]
    The dynamical system that will be used in most of the examples throughout the text is the so-called Logistic map.
    This map is very popular because, although it is mathematically very simple, it exhibits a wide range of complex behavior.
    The Logistic map was popularized by May~\cite{May19760610}.
    He proposed that the student should be introduced to the Logistic map early on in their studies to understand that simple rules can yield complex behavior.
    \par
    Mathematically, Logistic map can be thought of as a DDS of the form $\left( [0, 1], \mathcal{L}_{r} \right)$ where $\mathcal{L}_{r}: [0,1] \rightarrow [0,1]$, $\mathcal{L}_{r}(x) = rx(1-x)$.
    To ensure that $x$ remains in the range $[0,1]$, the parameter $r$ is chosen from the range $r \in [0, 4]$.
    \par
    Originally, Logistic map was used as a simple model of population dynamics in an environment with limited resources.
    The variable $x$ represents a population of some species and is understood as a normalized population.
    The value $1$ means that the population is at its maximum and the value $0$ means that the population is extinct.
    The parameter $r$ represents the growth rate of the population in each generation. \cite{Strogatz201854}
\end{remark}

\begin{definition}[\textit{$\mathbf{n}$}-th iteration]
    Let $\left( X, f \right)$ be a DDS. 
    The \emph{first iteration} of the map $f$ for the initial condition $x_0$ is represented by equation
    \begin{eqnarray}
        x_{1}  & = & f(x_{0})
    \end{eqnarray}
    \emph{$n$-th iteration} of $x_0$ is
    \begin{equation}
    x_{n} = f^{n}(x_0) =
        \begin{cases}
            f^{0}(x_0) = x_0 & \text{if } n = 0. \\
            \underbrace{f \circ f \circ \cdots \circ f}_\text{$n$ times}(x_0) & \text{if } n > 0. 
        \end{cases}
    \end{equation}
    Therefore, $f^{n}$ is the $n$-fold composition of the map f.
\end{definition}

\begin{definition}[Graph]
    Let $\left( X, f \right)$ be a DDS. A graph of $f$ is defined as a set $\{\, (x, f(x)) \mid x \in X \,\}$.
\end{definition}

\begin{remark}
    A graph of $\mathcal{L}_{r}$ for different $r$ is shown in Figure~\ref{fig:logistic_graph_example}.
    \par
    It is sometimes useful to generate the graph of $f^{n}$.
    Figure~\ref{fig:logistic_nthcomp_example} shows an example of the graph of $\mathcal{L}_{3.8}^{n}$ for different $n$.
\end{remark}

\begin{figure}[!h]
    \centering
    \includegraphics[width=0.5\textwidth]{Figures/logistic_graph_example.png}
    \caption{
        Graphs of $\mathcal{L}_{r}$ for different parameters $r$.
    }
    \label{fig:logistic_graph_example}
\end{figure}

\begin{figure}[!h]
    \centering
    \includegraphics[width=0.95\textwidth]{Figures/logistic_nthcomp_example.png}
    \caption{
        Graphs of $\mathcal{L}_{3.8}^{n}$ for different $n$.
    }
    \label{fig:logistic_nthcomp_example}
\end{figure}

\begin{definition}[Trajectory]
    Let $\left( X, f \right)$ be a DDS. Successive iterations of $f$ with initial condition $x_0 \in X$ form a \emph{trajectory} $\mathcal{T}(f, x_0)$ where \begin{eqnarray}
        \mathcal{T}(f, x_0) = f^0(x_0), f^1(x_0), f^2(x_0), f^3(x_0), \cdots  = x_0, x_1, x_2, x_3, \cdots
    \end{eqnarray}
    Moreover, the \emph{trajectory} between iterates $n$ and $m$, $n \leq m$, is
    \begin{eqnarray}
        \mathcal{T}_{n}^{m}(f, x_0) = f^{n}(x_0), f^{n+1}(x_0), \cdots, f^{m-1}(x_0), f^{m}(x_0) = x_n, \cdots, x_m
    \end{eqnarray}
\end{definition}

\begin{remark}
    Generating a trajectory is the most basic tool for understanding a DDS.
    Plotting the trajectory is a simple way to visualize the short-term behavior of the system.
    The graph of a trajectory shows, for example, whether the system is periodic, chaotic, or intermittent. (Intermittency is introduced in Chapter~\ref{chap:type-I intermittency}.)
    \par
    An example of trajectories of the Logistic map is shown in Figure~\ref{fig:trajectory_example}.
    For each trajectory, a projection onto a single line is also provided.
    Understanding how projections relate to the original trajectories is important to understand other tools for analyzing a DDS.
\end{remark}

\begin{figure}[!h]
    \centering
    \begin{subfigure}{0.6\textwidth}
        \centering
        \includegraphics[width=\textwidth]{Figures/logistic_map_trajectory_example1.png}
        \caption{}
    \end{subfigure}
    \hfill
    \begin{subfigure}{0.6\textwidth}
        \centering
        \includegraphics[width=\textwidth]{Figures/logistic_map_trajectory_example2.png}
        \caption{}
    \end{subfigure}

    \caption{
        Trajectories and their projections onto a line.
        Both cases (a) and (b) show a trajectory $\mathcal{T}_{50}^{100}(\mathcal{L}_{r}, 0.5)$ (left) and a projection of the trajectory onto a single line (right). 
        (a) chaotic behavior, $r = 3.828$. 
        (b) $3$-periodic behavior, $r = 3.829$. 
    }
    \label{fig:trajectory_example}
\end{figure}

\begin{definition}[Fixed point]
\label{def:fixed point}
    Let $\left( X, f \right)$ be a DDS. The point $x \in X$ is called a \emph{fixed point} of $f$ if and only if $f(x) = x$.
\end{definition}

\begin{definition}[Stable fixed point]
\label{def:sfp}
    Let $\left(X, f\right)$ be a DDS.
    Let $f$ be differentiable on $X$. 
    A fixed point $x \in X$ for which  $|f'(x)| < 1$ is called \emph{attracting} or \emph{stable}.
\end{definition}

\begin{definition}[Unstable fixed point]
\label{def:ufp}
    Let $\left(X, f\right)$ be a DDS.
    Let $f$ be differentiable on $X$. 
    A fixed point $x \in X$ for which $|f'(x)| > 1$ is called \emph{repelling} or \emph{unstable}.
\end{definition}

\begin{remark}
        It is worth noting what the conditions from Definitions~\ref{def:sfp} and \ref{def:ufp} stand for.
        The point of stability detection is the linear approximation, that is, replacing the original function by its tangent in the proximity of the fixed point.
        This approach can be generalized to an $n$-dimensional system $\left( \mathbb{R}^{n}, f \right)$.
        In this case, the stability of a fixed point is determined by the absolute values of the eigenvalues of the Jacobian matrix of $f$ at the fixed point.
    \par
    Direct evaluation of $\mathcal{L}_{r}(x)=x$ yields Fix$(\mathcal{L}_{r}) = \{ 0, (r-1)/r \}$.
    When the derivative of $\mathcal{L}_{r}$ is evaluated at these points, the stability of the fixed point can be determined.
    \par
    The fixed point $0$ is attracting if $r \in (0, 1)$ and repelling if $r \in (1, 4)$.
    For the parameter $r = 1$, the derivative at $x=0$ is $\mathcal{L}_{r}'(0)=1$.
    The map $\mathcal{L}_{r}$ undergoes a so-called saddle-node bifurcation (see Definition~\ref{def:bifurcation}).
    Consequently, the point $x=0$ becomes repelling and an attractive fixed point $x=(r-1)/r$ is created.
    \par
    Note that the fixed point $(r-1)/r$ doesn't exist for $r \in (0, 1)$ since $(r-1)/r \notin [0, 1]$. 
    The fixed point $(r-1)/r$ is attracting if $r \in (1, 3)$ and repelling if $r \in (3, 4) $.
    For parameter $r = 3$, the derivative at $x=(r-1)/r$ is $\mathcal{L}_{r}'((r-1)/r)=-1$.
    The map $\mathcal{L}_{r}$ undergoes a so-called period-doubling bifurcation (see Definition~\ref{def:bifurcation}).
    Consequently, point $x=(r-1)/r$ becomes repelling and $x=0$ remains repelling.
    \par
    Figure~\ref{fig:fixed_points_stability_example} displays two fixed points of the Logistic map $\mathcal{L}_{r}$.
    One fixed point is attracting, and one is repelling.
    Test for stability is portrayed graphically.
    The red area represents the region of stability and thus if the tangent of the graph of $\mathcal{L}_{r}$ at the fixed point is in the red area, the fixed point is attracting.
    Analogously, if the tangent of the graph is in the blue area, the fixed point is repelling.
\end{remark}

\begin{figure}[!h]
    \centering
    \begin{subfigure}{0.495\textwidth}
        \centering
        \includegraphics[width=\textwidth]{Figures/fixed_points_stability_example{1}.png}
        \caption{}
    \end{subfigure}
    \hfill
    \begin{subfigure}{0.495\textwidth}
        \centering
        \includegraphics[width=\textwidth]{Figures/fixed_points_stability_example{2}.png}
        \caption{}
    \end{subfigure}

    \caption{
        Two fixed points of $\mathcal{L}_{r}$ and their stability. 
        Parameters: (a) $r = 2.3$ and (b) $r = 3.3$. 
        Red area: region of stability.
        Red point in (a): attracting fixed point.
        Blue area: region of instability.
        Blue point in (b): repelling fixed point.
    }
    \label{fig:fixed_points_stability_example}
\end{figure}

\begin{definition}[Periodic point]
    Let $\left( X, f\right)$ be a DDS. The point $x \in X$ is called \emph{$n$-periodic} if $f^{n}(x)=x$ for $n \in \mathbb{N}$.
\end{definition}

\begin{definition}[Eventually periodic point]
    Let $\left( X, f\right)$ be a DDS. The point $x \in X$ is called \emph{eventually $n$-periodic} if $x$ is not $n$-periodic but there exists $m>0$ such that $f^{n+i}(x) = f^{i}(x)$ for all $i \geq m$.
    Hence, $f^{i}(x)$ is $n$-periodic for all $i \geq m$.~\emph{\cite{Devaney20211026}}
\end{definition}

\begin{definition}[Periodic orbit]
    Let $\left( X, f \right)$ be a DDS.
    Let $x_0$ be an $n$-periodic point.
    Then the set $O:=\{f^{n+m}(x_0):m \in \mathbb{N}\}$ is called an \emph{$n$-periodic orbit}.
    If $x_{0}$ is attracting, then the set $O$ is called \emph{stable periodic orbit}.
    If $x_{0}$ is repelling, then the set $O$ is called \emph{unstable periodic orbit}.
\end{definition}

\begin{theorem}
\label{theorem:single_orbit}
    Let $\left( [0, 1], \mathcal{L}_r\right)$ be a DDS. Then there exists at most one stable periodic orbit for each $r$.~\cite[p.~74]{Devaney20211026}
\end{theorem}

\begin{proof}
    The proof can be found in~\cite[p.~74]{Devaney20211026}.
\end{proof}

\begin{remark}
Theorem~\ref{theorem:single_orbit} will be useful in Chapter~\ref{chapter:intdetection} when constructing an algorithm for the detection of type-I intermittency.
\end{remark}

\begin{remark}[Cobweb diagram]
\label{def:cobweb}
    Cobweb diagrams are visual tools for understanding long term behavior of a DDS.
    They are closely related to the trajectory of a DDS and the projection of the trajectory onto a line.
    Furthermore, they are useful, for example, for determining whether a fixed point is attracting or repelling.
    \par
    An example of cobweb diagrams is given in Figure~\ref{fig:cobweb_diag_example}.
    Projections onto a line are also provided.
    Note the correspondence between Figures~\ref{fig:trajectory_example} and~\ref{fig:cobweb_diag_example}.
    \par
    Pseudocode~\ref{cobweb_alg} illustrates how to construct a cobweb diagram algorithmically.
\end{remark}

\begin{figure}[!h]
    \centering
    \begin{subfigure}{0.6\textwidth}
        \centering
        \includegraphics[width=\textwidth]{Figures/logistic_map_cobweb_diag_example1.png}
        \caption{}
    \end{subfigure}
    \hfill
    \begin{subfigure}{0.6\textwidth}
        \centering
        \includegraphics[width=\textwidth]{Figures/logistic_map_cobweb_diag_example2.png}
        \caption{}
    \end{subfigure}

    \caption{
        Cobweb diagrams of a trajectory $\mathcal{T}_{50}^{100}(\mathcal{L}_{r}, 0.5)$ and a projection of the trajectory onto a line. 
        Dashed diagonal line: identity function. 
        (a) $r = 3.828$. 
        (b) $r = 3.829$. 
        }
    \label{fig:cobweb_diag_example}
\end{figure}

\begin{algorithm}
\caption{Cobweb Diagram Construction}\label{cobweb_alg}
\begin{algorithmic}[1]
\Statex $f \gets$ map
\Statex $x_0 \gets$ initial state
\Statex $p \gets$ parameter of $f$
\Statex $x_{range} \gets$ range of $x$ values
\Statex $n \gets$ number of iterations

\State plot graph of $f_p(x)$ for $x \in x_{range}$
\State plot identity $g(x)=x$ for $x \in x_{range}$
\For{$i$ from $0$ to $n$}
\State plot line from $(f_{p}^{i}(x_0), f_{p}^{i}(x_0))$ to $(f_{p}^{i}(x_0), f_{p}^{i+1}(x_0))$
\State plot line from $(f_{p}^{i}(x_0), f_{p}^{i+1}(x_0))$ to $(f_{p}^{i+1}(x_0), f_{p}^{i+1}(x_0))$
\EndFor

\end{algorithmic}
\end{algorithm}

\begin{definition}[Bifurcation]
\label{def:bifurcation}
    Let $\left( X, f_p \right)$ be a DDS.
    A \emph{bifurcation} occurs when the qualitative behavior of $f_p$ changes as the parameter $p$ changes slightly.
\end{definition}

\begin{definition}[Bifurcation point]
\label{def:bifurcation_point}
    Let $\left( X, f_p \right)$ be a DDS.
    \emph{Bifurcation point} is a value of a parameter $p$ at which bifurcation occurs.
    The qualitative behavior of $\left( X, f_{p} \right)$ changes when the parameter $p$ is moved slightly around the bifurcation point.
\end{definition}

\begin{remark}[Bifurcation diagram]
\label{def: bif_diag} 
    Bifurcation diagram is a tool for the long-term global analysis of a DDS.
    It shows how the qualitative behavior of an arbitrary system $f_{p}(x_0)$ changes with varying parameters $p$.
    It helps identify stable regions, chaotic regions, and intermittent regions. (Intermittency is explained in Chapter~\ref{chap:type-I intermittency}.)
    The bifurcation diagram is especially useful for identifying bifurcations.
    \par
    An example of a bifurcation diagram of $\left( [0,1], \mathcal{L}_{r} \right)$ is shown in Figure~\ref{fig:bif_diag_example}.
    Notice the connection to Figures~\ref{fig:trajectory_example} and~\ref{fig:cobweb_diag_example}, especially the projections onto a line.
    The bifurcation diagram is generated as a union of these projections for various parameters.
    Projections fall into one of two categories:
    \begin{itemize}
        \item{A set of isolated points. This case corresponds to the periodic trajectory.}
        \item{A band of points. This is related to more complex phenomena such as chaos or quasi-periodicity.}
    \end{itemize}
    \par
    Algorithmic construction of a bifurcation diagram is given in Pseudocode~\ref{alg:bif_diag}.
\end{remark}

\begin{figure}[!h]
    \centering
    \includegraphics[width=0.95\textwidth]{Figures/logistic_map_bif_diag_example.png}
    \caption{
        Bifurcation diagram of $\mathcal{L}_{r}$ for $x_0 = 0.5$ and $r \in I := [ 2.81, 3.87 ]$. 
        Interval $I$ is sampled to $1000$ points. 
        Trajectories $\mathcal{T}_{4500}^{5000}(\mathcal{L}_{r}, x_0)$ are plotted.
    }
    \label{fig:bif_diag_example}
\end{figure}

\begin{algorithm}
\caption{Bifurcation Diagram Construction}\label{alg:bif_diag}
\begin{algorithmic}[1]
\Statex $f \gets$ map
\Statex $x_0 \gets$ initial condition
\Statex $p_{range} \gets$ range of parameters for which trajectory will be computed
\Statex $n_{total} \gets$ trajectories will be iterated for $n_{total}$ iterations
\Statex $n_{last} \gets$ $n_{last}$ iterations of a trajectory will be plotted

\For{parameter $p$ in $p_{range}$}
    \State $A \gets n_{total}$
    \State $B \gets n_{total} - n_{last}$
    \State $data \gets \mathcal{T}_{B}^{A}(f_p, x_0)$
    \For{$y$ in $data$}
        \State plot point $(p, y)$ onto canvas
    \EndFor
\EndFor

\end{algorithmic}
\end{algorithm}

\begin{remark}
    In Figure~\ref{fig:bif_diag_example} it is visible that a small change of a parameter can cause qualitatively different behavior of the DDS's trajectories.
    When the qualitative change occurs, it is said that a bifurcation has occurred.
    An example of a bifurcation point is $r = 1+\sqrt{8} \approx 3.82842$. For this $r$, $\mathcal{L}_{r}$ changes its behavior from seemingly chaotic to $3$-periodic. 
    This type of bifurcation is called a saddle-node bifurcation.
    Another example of a bifurcation point is $r \approx 3.446$. For this $r$, $\mathcal{L}_{r}$ changes its behavior from $2$-periodic to $4$-periodic.
    This type of bifurcation is called a period-doubling bifurcation.
\end{remark}

\endinput