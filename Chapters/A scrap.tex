\textcolor{red}{
This thesis is aimed at developing an algorithm for detection of type-I intermittency in discrete dynamical systems.
However, the initial aim was more broad and changed focus over the course of writing the thesis.
}
\par
\textcolor{red}{
In the beginning, the aim of the thesis was simply to study intermittency \cite{Elaskar2023}.
Intermittency is a phenomenon where the dynamical system changes between seemingly periodic and chaotic behavior.
The mathematical study of intermittency originated in 1980s through observation of intermittent transition to turbulence in convective fluids \cite{Pomeau1980}.
Nowadays, the study of intermittency has grown considerably \cite{Elaskar2017}.
This phenomenon is exhibited by both continuous and discrete dynamical systems \cite{Devaney20211026, Hirsch2013, Strogatz201854}.
In fact, the first three types of intermittency were discovered in continuous systems \cite{Manneville1980-2}.
}
\par
\textcolor{red}{
Later, the focus shifted on intermittency in discrete dynamical systems.
The reason was that these systems are easier to simulate and the mathematics needed for their study is more accessible for an undergraduate student.
}
\par
\textcolor{red}{
Through studying trajectories and bifurcation diagrams of discrete dynamical systems, the realization that intermittency is related to breakpoints was made.
A breakpoint is a place in bifurcation diagram where chaotic behavior almost instantly changes to periodic behavior.
Thus there must be some kind of progression from chaos to periodicity and vice versa happening near the breakpoint.
This route is called an intermittency route to chaos \cite{Strogatz201854}.
}
\par
\textcolor{red}{
By looking at the breakpoints in the bifurcation diagrams of various systems, it was discovered that intermittency is indeed related to them.
This has led to creating the four big Figures~\ref{fig:complex_pomeau_manneville},~\ref{fig:complex_logistic},~\ref{fig:complex_henon},~\ref{fig:complex_duffing}.
These Figures show the breakpoint and the trajectories of the system for parameters near the breakpoint.
It was shown that there is an intermittency on one side of the breakpoint and periodicity on the other side.
}
\par
\textcolor{red}{
Furthermore, a realization that the intermittency near the breakpoint causes ambiguity of the bifurcation diagram was made.
An example of such ambiguity was produced in Figure~\ref{fig:ambiguous_bif_diag_example}.
Since the bifurcation diagram is a tool commonly used for analysis of the system, it was deemed useful to warn about the ambiguity.
Additionally, discrete dynamical systems are often used in engineering and science to model various real-world situations [cite here].
Having knowledge that intermittency might occur for certain parameters of the system could be an useful tool for prediction of the system's behavior.
Thus, a decision was made to develop an algorithm that would detect these breakpoints and thus intermittency.
}
\par
\textcolor{red}{
Later, it was found out that the intermittency near the breakpoint is actually a type-I intermittency \cite{Elaskar2022}.
This type of intermittency is associated with saddle-node bifurcations and it is one of the most studied types of intermittency [cite here].
}
\par
\textcolor{red}{
The first attempt to detect type-I intermittency was done in a very naive way.
By looking at the bifurcation diagram, the breakpoint are places where there is a change between periodic and non-periodic behavior.
The initial idea was that if there is a quick algorithm to determine periodicity of the system for a given parameter, then the breakpoints could be found.

The Julia programming language~\cite{Bezanson2017,Bezanson20181024} is used for numerical simulations in this thesis.
There exists a library in Julia, DynamicalSystems.jl~\cite{Datseris2018}, used for simulations of dynamical systems.
This library mentioned and implements an algorithm for search of unstable periodic orbits \cite{Schmelcher1997}.
This algorithm could be used for computation of stable periodic orbits which are related to the periodic behavior seen in a bifurcation diagram.
However some systems can have several periodic orbits for a single parameter [cite here].
The focus of the study shifted to the Logistic map because it is proven that it has only one stable periodic orbit for each parameter \cite{Devaney20211026}.
}
\par
\textcolor{red}{
The algorithm was implemented and tested on the Logistic map.
Later, an improvement of the algorithm was found and implemented~\cite{Davidchack1999, Davidchack2001, Klebanoff2001, Crofts2007}.
However its implementation was found to be a bit difficult.
In the end, another algorithm was adopted and implemented~\cite{Bu2004}.
}
\par
\textcolor{red}{
Through using the algorithms mentioned in the last paragraph, the detection of type-I intermittency was achieved.
It was decided that the output of the algorithm should be graphical and should warn about the ambiguity of the bifurcation diagram in presence of intermittency.
In the end a colorization scheme of the bifurcation diagram was implemented.
It is based on certain probabilistic properties of type-I intermittency \cite{DelRio2014}.
More specifically it is based on the characteristic relation of type-I intermittency.
}
\par
\textcolor{red}{
Additionally, it was understood that search for breakpoints in the bifurcation diagram is actually the same as search for saddle-node bifurcations.
There already exists an algorithm for detection of saddle-node bifurcations \cite{Matsushita20170721}.
Furthermore, by understanding that saddle-node bifurcations can be found using optimization techniques, a new method was proposed.
}
\par
\textcolor{red}{
Overall, the thesis aimed at developing an algorithm for detection of type-I intermittency in discrete dynamical systems.
This aim was achieved for the Logistic map.
There is big potential for further development of the algorithm.
Firstly understanding how reliable and robust the algorithm is.
Secondly generalizing the algorithm to other systems and systems of high dimensions.
Thirdly, creating algorithms for detection of other types of intermittency.
Lastly, implementing the algorithm for continuous time systems.
}
