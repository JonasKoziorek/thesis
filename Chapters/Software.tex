\chapter{Software}
\label{sec:software}

This chapter discusses the software used for numerical computations in this thesis.
Some explanation of the source code used for generating everything in this thesis is also included.

\section{Julia Programming Language}
Numerical simulations play a crucial role in the field of dynamical systems.
Analytical approach is often impossible and numerical simulation is the only option.
In the past it was common to use Fortran or C for fast numerical simulations.
However these languages are low-level and require a lot of effort and skill to write even simple programs.
Later Matlab and Python became popular in the scientific world.
The ease of sharing code and writing human-readable programs made them a good choice for many scientists.
However, Matlab is a proprietary software and Python is an interpreted language which makes it slow.
\par
Julia is a high-level programming language with a syntax similar to Matlab and Python.
It aims to bridge a gap between dynamically typed languages like Python and statically typed languages like C.
The language is designed in a way that it allows you to write high level code that is as fast as low level code.
Code written in Julia resembles mathematical abstraction because of it's use of generic types and multiple dispatch. 
Sharing code with other scientists is easy because Julia is open-source, free and has big community. 
Ability to access low-level details enables user to achieve maximum performance.
Other features include built-in package manager, meta-programming, optional typing, just-in-time compilation and more.
It is also suitable for parallel and distributed computing. \cite{Bezanson2017,Bezanson20181024}
\par
Thanks to it's vibrant community of scientists and developers, Julia has a lot of packages for scientific computing.
One of the notable ones is DynamicalSystems.jl open source library.
It provides a framework and efficient algorithms for computation with dynamical systems.
Generic algorithms work both for discrete time and continuous time dynamical systems.
Library contains algorithms for many subfields of dynamical systems theory. \cite{Datseris2018}
List of some of the features is presented in the following list:
\begin{itemize}
    \item Delay coordinates embedding.
    \item Poincaré surface of sections.
    \item Spectrum of Lyapunov exponents.
    \item Detecting and distinguishing chaos using the GALI method.
    \item Automated production of orbit diagrams for continuous systems.
    \item Generalized entropies and permutation entropy.
    \item Recurrence quantification analysis
    \item 0-1 test for chaos.
    \item and much more...
\end{itemize} [cite the website https://juliadynamics.github.io/DynamicalSystems.jl/dev/contents/]

DynamicalSystems.jl package wasn't used in this thesis because better control and understanding of the algorithms was needed.
\par
Julia is well suited for numerical simulations in the field of dynamical systems.
Such simulations often require many different mathematical algorithms and methods.
Support for differential equation solvers, linear algebra solvers, automatic evaluation of jacobians, rootfinding algorithms etc. are all needed.
Need for high quality plotting libraries that are easy to use is also important.
Interactive developent via REPL and Jupyter notebooks is also a big advantage.
One of the big advantages is great support for profilling, benchmarking and alrogithm optimization.
Every scientist without skills of a computer scientist can write track down bottlenecks and limit number of heap allocations.
Good support for multi-threading and parallel computing is also a big advantage.
Many of the algorithms used in this thesis are easily parallelizable.
Julia can also run on a supercomputer \cite{Regier2016-vq}

[Add table comparing computation time between julia, python and matlab]

\section{Implementation commentary}
This section includes commentary about the source code used for generating everything in this thesis.
\par
All data, plots and results included in this thesis were generated using Julia programming language.
The whole source code is available on github [add link] and as an attachment to this thesis.
For the purposes of exploration of this thesis I have created my own library called DDS.jl (Discrete Dynamical Systems).
The main features include:
\begin{itemize}
    \item Support of multi dimensional discrete dynamical systems.
    \item Trajectory generation and its plots.
    \item Cobweb diagram generation.
    \item Bifurcation diagram generation.
    \item Automatic differentiation and jacobians of arbitrary system.
    \item Davidchack-Lai algorithm.
    \item Bu-Wang-Jiang algorithm.
    \item Global search, Local search, Colorization algorithms
\end{itemize}
Many of the algorithms are parallelizable.
Bifurcation diagrams can be generated in parallel by splitting up the parameter space.
Local search can run in parallel by computing left boundary estimation and right boundary estimation separately.
Local search can also run in parallel for each interval found in global search.
Global search can run in parallel by splitting up the parameter space.
\par
All plots were creating using Julia plotting packages [cite makie.jl, plots.jl, pyplot.jl].
2D Vector graphics was creates using the Luxor.jl package [cite Luxor.jl].

\endinput