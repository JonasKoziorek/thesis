\chapter{Mathematical Background}
In this chapter we introduce neccessary notions from theory of discrete dynamial systems needed in this thesis.
For purposes of this text we provide only simplified definitions from standard theory of dynamical systems.

\section{Discrete dynamical systems}

\begin{definition} \textbf{(Discrete dynamical system)} \\
    Discrete dynamical system (abbr. DDS) is a map $f: \mathbb{R}^{m} \rightarrow \mathbb{R}^{m}$.
    Map f can depend on one or more parameters $P \in \mathbb{R}^n$.
    We denote map $f$ with parameters $P$ as $f_P$.
\end{definition}

\begin{remark}
    In this text we work with one dimensional or two dimentional discrete dymical systems.
    We will call simply call them 1D or 2D maps.
    If dimension does not matter we will call such systems maps.
\end{remark}

\begin{definition} \textbf{(n-th iteration)} \\
    One iteration of map $f$ for initial condition $x_0$ is represented by equation
    \begin{eqnarray}
        x_{1}  & = & f(x_{0})
    \end{eqnarray}
    $n$-th iteration of $x_0$ is
    \begin{equation}
        x_{n} = f^{n}(x_0) = \underbrace{f \circ f \circ \cdots \circ f}_\text{$n$ times}(x_0)
    \end{equation}
    Where $f^{n}(x_0)$ is $n$-th composition of f into itself. \\
    We define $f^{0}(x_0) = x_0$.
\end{definition}

\begin{definition} \textbf{(Trajectory)} \\
    Let $(X, f)$ be a DDS. 
    Successive iterates of $f$ with initial condition $x_0$ form a trajectory $T(f, x_0)$ where
    \begin{eqnarray}
        T(f, x_0) = f^0(x_0), f^1(x_0), f^2(x_0), f^3(x_0), \cdots  = x_0, x_1, x_2, x_3, \cdots
    \end{eqnarray}
    For $n \leq m$ we define
    \begin{eqnarray}
        T_{n}^{m}(f, x_0) = x_n, \cdots, x_m
    \end{eqnarray}
\end{definition}

\begin{remark}
    Example of a trajectory is shown in Figure \ref{fig:trajectory_example}.
\end{remark}

\begin{figure}[!h]
    \centering
    \includegraphics[width=1.0\textwidth]{DDS/Figures/logistic_map_trajectory_example.png}
    \caption{Trajectory $T_{0}^{80}(\mathbb{L}_p, 0.5)$ for logistic map $\mathbb{L}$ with parameter $p=3.841$}
    \label{fig:trajectory_example}
\end{figure}

\begin{definition} \textbf{(Fixed point)} \\
    \label{def:fixed point}
    Point $x \in \mathbb{R}^n$ is called a fixed point of $f$ if it holds $f(x) = x$.
    We denote set of all fixed points of $f$ as $Fix(f)$. \cite{Devaney20211026}
\end{definition}

\begin{definition} (\textbf{Stable fixed point (1D)})\\
    Fixed point $x \in \mathbb{R}$ of map $f$ is stable if $|\frac{d{f(x)}}{dx}| < 1$.
    We call this point attracting.
\end{definition}

\begin{definition} (\textbf{Unstable fixed point (1D)})\\
    Fixed point $x \in \mathbb{R}$ of map $f$ is unstable if $|\frac{d{f(x)}}{dx}| > 1$.
    We call this point repelling.
\end{definition}

\begin{definition} (\textbf{Periodic point})\\
    Point $x \in \mathbb{R}^m$ is called $n$-periodic if $f^{n}(x)=x$.
    We denote set of all $n$-periodic points of $f$ as $Per_{n}(f)$. \cite{Devaney20211026}
\end{definition}

\begin{definition} (\textbf{Periodic orbit})\\
    For $n$-periodic point $x_0$ the set $\{f^{in}(x_0):i \in \mathbb{N}\}$ is called periodic orbit of period $n$.
    If $x_0$ is attracting, we call periodic orbit stable (abbr. PO).
    If $x_0$ is repelling, we call periodic orbit unstable (abbr. UPO).
\end{definition}

\begin{theorem} (\textbf{my theorem, needs to be proved})\\
$Fix(f^n) = Per_n(f)$
\end{theorem}

\begin{theorem} (\textbf{my theorem, needs to be proved})\\
If number of stable fixed points of $f^n$ is $m$ and $0<m<n$.
Then $m|n$ and $f$ is $m$ periodic.
\end{theorem}

\begin{remark} (\textbf{Cobweb diagram}) \\
    \label{def:cobweb}
    Cobweb diagrams are a visual tools for understanding long term behavior of a DDS.
    Pseudocode \ref{cobweb_alg} illustrates construction of cobweb diagram.
    Example of cobweb diagram can be seen in Figure \ref{fig:cobweb_diag_example}.
\end{remark}

\begin{algorithm}
\caption{Cobweb Diagram Construction}\label{cobweb_alg}
\begin{algorithmic}[1]
\Statex \textbf{parameter describtion:}
\Statex $f$: evolution rule
\Statex $x_0$: initial state
\Statex $p$: parameter of $f$
\Statex $x_{range}$: range of $x$ values
\Statex $n$: number of iterations
\Statex

\Function{CobwebDiag}{$f$, $x_0$, $p$, $x_{range}$, $n$}
\State plot graph of $f_p(x)$ for $x \in x_{range}$
\State plot identity $g(x)=x$ for $x \in x_{range}$
\For{$i$ from $0$ to $n$}
\State plot line from $(f_{p}^{i}(x_0), f_{p}^{i}(x_0))$ to $(f_{p}^{i}(x_0), f_{p}^{i+1}(x_0))$
\State plot line from $(f_{p}^{i}(x_0), f_{p}^{i+1}(x_0))$ to $(f_{p}^{i+1}(x_0), f_{p}^{i+1}(x_0))$
\EndFor
\EndFunction

\end{algorithmic}
\end{algorithm}

\begin{figure}[!h]
    \centering
    \includegraphics[width=0.8\textwidth]{DDS/Figures/logistic_map_cobweb_diag_example.png}
    \caption{CobwebDiag($f = \mathbb{L}^2$, $x_0 = 0.6$, $p = 3.81$, $x_{range} = [0, 1]$, $n = 5$)}
    \label{fig:cobweb_diag_example}
\end{figure}

\begin{definition} \textbf{(Bifurcation)} \\
    \label{def:bifurcation}
    Bifurcation occurs when qualiative behavior of DDS $f$ changes as parameter varies.
\end{definition}

\begin{definition} \textbf{(Bifurcation point)} \\
    \label{def:bifurcation_point}
    Parameter at which \hyperref[def:bifurcation]{bifurcation} occurs.
\end{definition}

\begin{definition} \textbf{(Saddle-node bifurcation)} \\
    \label{def:saddle_node_bif}
    To be defined.
\end{definition}


\begin{remark} (\textbf{Bifurcation diagram}) \label{def: bif_diag} \\
    Bifurcation diagrams are visual representations of a DDS.
    They show us how qualitative behavior of $f_P(x_0)$ change for varying parameters P.
    Bifurcation diagrams illustrate \hyperref[def:bifurcation]{bifurcations} and help us identify \hyperref[def:bifurcation_point]{bifurcation points}.
    Construction of a bifurcation diagram is illustrated in Pseudocode \ref{bif_diag_alg}.
    Example of a bifurcation diagram can be seen in Figure \ref{fig:bif_diag_example}.
\end{remark}

\begin{algorithm}
\caption{Bifurcation Diagram Construction}\label{bif_diag_alg}
\begin{algorithmic}[1]
\Statex \textbf{parameter describtion:}
\Statex $f$: evolution rule
\Statex $x_0$: initial condition
\Statex $p_{range}$: range of parameters for which trajectory will be computed
\Statex $n_{total}$: trajectories will be iterated for $n_{total}$ iterations
\Statex $n_{last}$: $n_{last}$ iterations of a trajectory will be plotted
\Statex

\Function{BifDiag}{$f$, $x_0$, $p_{range}$, $n_{total}$, $n_{last}$}
\For{parameter $p$ in $p_{range}$}
    \State $data \leftarrow T_{n_{total}-n_{last}}^{n_{total}}(f_p, x_0)$
    \For{$y$ in $data$}
        \State plot point $(p, y)$ onto canvas
    \EndFor
\EndFor
\EndFunction

\end{algorithmic}
\end{algorithm}

\begin{figure}[!h]
    \centering
    \includegraphics[width=0.9\textwidth]{DDS/Figures/logistic_map_bif_diag_example.png}
    \caption{BifDiag($f = \mathbb{L}$, $x_0 = 0.5$, $p_{range} = [2.81,3.87]$ sampled to $1000$ points, $n_{total} = 5000$, $n_{last} = 500$)}
    \label{fig:bif_diag_example}
\end{figure}

More precise theoretical background can be found at \cite{Devaney20211026,Elaskar2017,Lynch2014}.

\endinput