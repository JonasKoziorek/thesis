\chapter{Mathematical Background}
In this chapter we introduce neccessary notions from theory of discrete dynamial systems needed in this thesis.
For purposes of this text we provide only simplified definitions from standard theory of dynamical systems.

\section{Discrete dynamical systems}

\begin{definition} \textbf{(Discrete dynamical system)} \\
    \label{def:discrete dynamical system}
    $m$-dimensional discrete dynamical system is a map $f: (\mathbb{R}^{m}, P) \rightarrow (\mathbb{R}^{m}, P)$ where selected $P \in \mathbb{R}^j$ are parameters.
\end{definition}

\begin{remark}
    In this text we work with one dimensional or two dimentional discrete dymical systems.
    We will call simply call them 1D or 2D maps.
    If dimension doesn't matter we will call such systems maps.
\end{remark}

\begin{remark}
    We may shorten map notation by specifying parameters as subscript $f_P: \mathbb{R}^m \rightarrow \mathbb{R}^m$.
    Map $f_P$ is equivalent to map $f$ in Definition \ref{def:discrete dynamical system}.
\end{remark}

\begin{definition} \textbf{(n-th iteration)} \\
    $n$-th iteration of map $f_P$ for initial condition $x_0$ and parameters $P$ is represented by difference equation
    \begin{eqnarray}
        x_{n+1}  & = & f_P(x_{n}) \\
    \end{eqnarray}
    We can shortly write
    \begin{equation}
        x_{n} = f_P(f_P(f_P(...f_P(x_0)))) = f^{n}_{P}(x_0) \nonumber
    \end{equation}
    Where $f^{n}_{P}(x_0)$ is $n$-th composition of f into itself. \\
    We define $f^{0}_{P}(x_0) = x_0$.
\end{definition}

\begin{definition} \textbf{(Fixed point)} \\
    \label{def:fixed point}
    Point $x$ is called a fixed point of map $f_P$ if it holds $f_P(x) = x$.
\end{definition}

\begin{definition} (\textbf{Periodic point})\\
    Point $x$ is called $n$-periodic if $f^{n}_P(x)=x$. Point $x$ is called eventually $n$-periodic if $(\exists m \in \mathbb{N}, m>0) (\forall i \geq m): f^{n+i}_P(x)=f^{i}_P(x)$.
    Thus point $f^{i}_P(x)$ in $n$-periodic for some $i \geq m$.
\end{definition}

\begin{remark}
    Notice that every fixed point x is also 1-periodic point.  
\end{remark}

\begin{definition} (\textbf{Stable fixed point})\\
    Fixed point $x$ of map $f_P$ is stable if $|\frac{d{f_P(x)}}{dx}| < 1$.
\end{definition}

\begin{definition} (\textbf{Periodic orbit})\\
    Set of stable fixed points of map $f^{n}$ is called $n$-periodic orbit or stable $n$-periodic orbit.
\end{definition}

\begin{definition} (\textbf{Unstable periodic orbit})\\
    Set of all fixed points of map $f^{n}$ is called unstable $n$-periodic orbit.
\end{definition}

\begin{definition} \textbf{(Bifurcation)} \\
    \label{def:bifurcation}
    Bifurcation occurs when qualiative behavior of the system changes as parameter varies.
\end{definition}

\begin{definition} \textbf{(Bifurcation point)} \\
    \label{def:bifurcation_point}
    Parameter at which \hyperref[def:bifurcation]{bifurcation} occurs.
\end{definition}

\begin{definition} \textbf{(Saddle-node bifurcation)} \\
    \label{def:saddle_node_bif}
    To be defined.
\end{definition}

\begin{definition} (\textbf{Forward orbit}) \\
    Line graph with $n=1,2,3...$ on $x$-axis and corresponding iterates $f^{n}_P(x_0)$ on $y$-axis is called forward orbit of $f_{P}(x_0)$ for some parameters $P$ and initial condition $x_0$.
\end{definition}

\begin{remark}
    Forward orbits are useful for studying short term and long term behavior of studied map.
    They cleanly show how initial condition is evolved.
\end{remark}

\begin{example}
    In figure \ref{fig:forward_orbit_example} we present an example of forward orbit for logistic map. More information about logistic map will be presented in section \ref{subsec:logistic map}.
    \begin{figure}[!h]
        \centering
        \includegraphics[width=1.0\textwidth]{DDS/Figures/logistic_map_forward_orbit_example.png}
        \caption{Example of forward orbit}
        \label{fig:forward_orbit_example}
    \end{figure}
\end{example}

\begin{definition} (\textbf{Cobweb diagram}) \\
    \label{def:cobweb}
    For 1D maps we can construct so called cobweb diagram with following algorithm:
    \begin{enumerate}
        \item select map $f_P$ for some parameters $P$ and initial condition $x_0$ and starting point $f^{j}_P(x_0) = x_j$
        \item select order of the algorith $m \in \mathbb{N}$
        \item plot graph of $f^{m}_P(x)$ for all $x$ in selected range
        \item plot identity function $g(x) = x$
        \item set $i:=j$

        \item plot line from point $(f^{i}_P(x_0), f^{i}_P(x_0))$ to point $(f^{i}_P(x_0), f^{i+m}_P(x_0))$ \label{item1}
        \item plot line from point point $(f^{i}_P(x_0), f^{i+m}_P(x_0))$ to point $(f^{i+m}_P(x_0), f^{i+m}_P(x_0))$ \label{item2}
        \item set $i:=i+m$ \label{item3}

        \item repeat steps \ref{item1}-\ref{item3} as long as desired
    \end{enumerate}

\end{definition}

\begin{remark}
    Cobweb diagrams are useful for understanding long term behavior of studied map.
    Fixed points and $n$-periodic points are seen in the diagram right away since they are represented by intersection of the graph of the map and identity function.
\end{remark}

\begin{example}
    In figure \ref{fig:cobweb_diag_example} we present an example of cobweb diagram for logistic map. More information about logistic map will be presented in section \ref{subsec:logistic map}.
    \begin{figure}[!h]
        \centering
        \includegraphics[width=0.8\textwidth]{DDS/Figures/logistic_map_cobweb_diag_example.png}
        \caption{Example of cobweb diagram}
        \label{fig:cobweb_diag_example}
    \end{figure}
\end{example}

\begin{definition} (\textbf{Bifurcation diagram}) \\
    \label{def: bif_diag}
    We can construct so called bifurcation diagram for map $f$ with following algorithm:
    \begin{enumerate}
        \item select map $f$ and some initial condition $x_0$
        \item if $f$ takes single parameter $\epsilon \in \mathbb{R}$ select desired range for this parameter
        \item if $f$ takes several parameters $P \in \mathbb{R}^n$ select single parameter $\epsilon$ from $P$ and select it's desired range, set remaining parameters to some fixed value
        \item for selected range of $\epsilon$, discretize range to finite number of samples \label{item_sample}
        \item choose number of iterations $N_{total}$
        \item choose number of last iterations that will be ploted $N_{last}$
        \item for each parameter $\epsilon$ from step \ref{item_sample} construct forward orbit of $f_{\epsilon}(x_0)$ for $N_{total}$ iterations \label{item_forward_orbit}
        \item bifurcation diagram is a scatter plot with parameter range from item \ref{item_sample} on $x$-axis and $N_{last}$ iterations of the coresponding evolution of $f(x_0)$ from step \ref{item_forward_orbit} on $y$-axis
    \end{enumerate}
\end{definition}

\begin{remark}
    Bifurcation diagrams show us how qualitative behavior of forward orbits of $f_P(x_0)$ change for varying parameters P.
    In other words bifurcation diagrams illustrate \hyperref[def:bifurcation]{bifurcations} and help us identify \hyperref[def:bifurcation_point]{bifurcation points}
\end{remark}

\begin{example}
    In figure \ref{fig:bif_diag_example} we present an example of bifurcation diagram for logistic map. More information about logistic map will be presented in section \ref{subsec:logistic map}.
    \begin{figure}[!h]
        \centering
        \includegraphics[width=0.9\textwidth]{DDS/Figures/logistic_map_bif_diag_example.png}
        \caption{Example of bifurcation diagram}
        \label{fig:bif_diag_example}
    \end{figure}
\end{example}

More precise theoretical background can be found at \cite{Devaney20211026,Elaskar2017,Lynch2014}.

\endinput