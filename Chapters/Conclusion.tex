\chapter{Conclusion}

\textcolor{red}{
This thesis presented an approach for the detection of type I interference in discrete dynamical systems.
The main body of the text was divided into three chapters.
}
\par
\textcolor{red}{
The first section of Chapter 2 establishes basic concepts from the theory of discrete dynamical systems.
Tools such as bifurcation diagrams, cobweb diagrams, and trajectory plots are introduced and used throughout the text to support the concepts.
Fixed points, periodic points, and their stability are also introduced.
The visualization tools are applied to the Logistic map.
Fixed points are computed for the first iterate of the Logistic map and their stability is analyzed.
}
\par
\textcolor{red}{
The second section of Chapter 2 establishes the phenomenon of type-I intermittency.
Motivation behind this type of intermittency is provided and illustrated through Figures.
An ambiguity of the bifurcation diagram is brought up, and the creation of such an ambiguous diagram is shown.
Furthermore, a theory behind the laminar phase length is introduced.
The characteristic relation is explained.
A verification of the characteristic relation is carried out for the Logistic map.
}
\par
\textcolor{red}{
Chapter 3 deals with the development of the algorithm for the detection of type I intermittency.
The Chapter has three sections - Global search, Local search and Colorization.
These sections correspond to the three sub-parts of the algorithm.
}
\textcolor{red}{
The Global search section comes up with the idea of a Naive global search.
Additionally, it presents the Bu-Wang-Jiang algorithm for computation of periodic orbits.
In general, the Global search subpart of the algorithm intends to find approximate locations of the breakpoints in the bifurcation diagram.
Global search is applied to the Logistic map and the results are presented.
}
\par
\textcolor{red}{
The Local search section presents three algorithms - Naive local search, Nested-Layer Particle Swarm Optimization, and Local search through global optimization.
These three algorithms can be used to find the exact location of the breakpoints in the bifurcation diagram by using information from the Global search.
Local search is applied to the Logistic map and the results are presented.
}
\par
\textcolor{red}{
The Colorization sections deal with the colorization of the bifurcation diagram.
It builds on top of the Global search and Local search.
It utilizes the characteristic relation developed in the previous chapter.
An algorithm for optimal colorization bounds is presented.
Colorization is applied to the Logistic map and the results are presented.
}
\par
\textcolor{red}{
Chapter 4 discusses the software.
It talks about the Julia programming language and its ecosystem.
Additionally, it mentions some information about the code used for the numerical experiments and plot creation.
}

\textcolor{red}{
Overall, the thesis aimed to develop an algorithm for the detection of type-I intermittency in discrete dynamical systems.
This aim was achieved for the Logistic map.
There is a great potential for further development of the algorithm.
Firstly understanding how reliable and robust the algorithm is.
Secondly generalizing the algorithm to other systems and systems of high dimensions.
Thirdly, creating algorithms for detection of other types of intermittency.
Lastly, implementing the algorithm for continuous dynamical systems.
}

\endinput