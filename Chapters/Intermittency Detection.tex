\chapter{Intermittency Detection}
Section \ref{sec:ambiguous_bif_diag} of the previous chapter introduced the problem of ambiguity of the bifurcation diagram.
This chapter introduces an algorithm that aims to detect the ambiguity.
The algorithm consists of 3 core parts.

% \begin{description}
% 	\item [1. Global Search] -- searching for areas where breakpoints occur
% 	\item [2. Local Search] -- searching for precise locations of breakpoints
% 	\item [3. Coloring] -- coloring the bifurcation diagram in proximity of breakpoints
% \end{description}

\begin{enumerate}
	\item \textbf{Global Search} -- searching for areas where breakpoints occur.
	\item \textbf{Local Search} -- searching for precise locations of breakpoints.
	\item \textbf{Coloring} -- coloring the bifurcation diagram in the proximity of breakpoints.
\end{enumerate}

Each of these parts are described in the following sections.
The last section of this chapter combines all the parts together.

\section{Global Search}
The initial phase of the algorithm aims to find approximate parameter intervals during which type-I intermittency occurs.
More specifically, parameter intervals that contain a breakpoint are identified.
\par
A breakpoint is a parameter value at which the system transitions from chaotic/intermittent behavior to periodic behavior.
The breakpoints of interest have chaotic behavior on the left side and periodic behavior on the right side.
Each breakpoint can be identified by a unique number $n$ such that on the right side the behavior is $n$-periodic.
If the parameter space could be searched through and the period of the system for each parameter value could be determined, the breakpoints could be easily found.
To do that, an algorithm is needed to determine the period of the system for any given parameter value.
\par
If system has stable $n$-periodic orbit or the system starts to evolve at an unstable $n$-periodic point, then the system is $n$-periodic or eventually $n$-periodic.
It is known that the Logistic map $\mathcal{L}_r$ can only have at most one unique stable $n$-periodic orbit for the parameter $r$.
This means that if a stable $n$-periodic orbit is found for parameter $r$, then it is known that $\mathcal{L}_r$ is $n$-periodic.
At least in the case where the initial condition is not an unstable periodic point.
To find out whether $\mathcal{L}_r$ has a stable $n$ periodic orbit, it is necessary to find fixed points of $\mathcal{L}_r, \mathcal{L}_{r}^{2}, \mathcal{L}_{r}^{3}, \dots$ and determine their stability.
Therefore, the smallest $n$ for which $\mathcal{L}_{r}^{n}(x^{*})=x^{*}$ and $x^{*}$ are stable is the period of $\mathcal{L}_r$.
\par
There exist algorithms for finding all fixed points of an arbitrary dynamical system.
These algorithms are quite efficient and can be used for high-dimensional systems.
However, the aim is to search through the entire parameter space in search of breakpoints.
This search requires finding fixed points for hundreds or thousands of parameters in the parameter space.
Although existing algorithms search for fixed points efficiently, they are not efficient enough.
They will come in handy later in the algorithm, but for now it is needed to find a way to approximate period of a system quickly.
\par
When periodic behavior is present in a bifurcation diagram, it means that the system is periodic or eventually periodic for a specific initial condition $x_0$.
In other words, there exists $m > 0$ such that $f^{i}(x)=x$ is periodic for $i \geq m$.
A naive approach to determine whether a system is periodic is simply iterating the system for some large enough $m$ and checking whether $f^{m}(x)=f^{m+i}(x)$.
That implies that point $x$ is eventually periodic (or periodic) of period $i$.
\par
Simple check presented in the last paragraph is not precise in areas with intermittency.
The first $m$ iterations could end up in the laminar phase, which might be deemed periodic.
It also fails when the system is eventually periodic for some $n > m$, e.g. higher than the parameter $m$.
Regardless of its flaws, this simple check seems to be good enough for the naive search through the parameter space.
It is much quicker than finding all fixed points for each parameter, and it is precise enough.
The pseudocode for the naive global search is presented in Algorithm~\ref{alg:naive_global_search}.

\begin{algorithm}[!h]
    \caption{Naive global search}
    \label{alg:naive_global_search}
    \begin{algorithmic}[1]
        \Statex $f \gets$ map
        \Statex $p_{A} \gets$ left boundary of the parameter space
        \Statex $p_{B} \gets$ right boundary of the parameter space
        \Statex $n \gets$ number of samples in the parameter space
        \Statex $m \gets$ number of iterations
        \Statex $o \gets$ maximum period to check
        \Statex $x_0 \gets$ initial condition
        \State $p \gets p_{A}$
        \State $p_{step} \gets \frac{p_{B} - p_{A}}{n}$
        \While{$p \leq p_{B}$}
            \State $x_1 \gets f_{p}^{m}(x_0)$
            \For{$i \gets 1$ to $o$}
                \State $x_2 \gets f_{p}^{i}(x_1)$
                \If{$x_2 = x_1$}
                    \State $f_p$ is $i$-periodic
                \EndIf
            \EndFor
            \State $p \gets p + p_{step}$
        \EndWhile
    \end{algorithmic}
\end{algorithm}

This algorithm is able to approximate the period of a system for each parameter in the parameter space.
If the algorithm does not find any periodicity for some parameter $p$, then it is assumed that the system is chaotic for this parameter.
The goal is to look for breakpoints, e.g. regions where there is a shift from chaotic (non-periodic) behavior to periodic one.
To do that it is needed to use the results of the previous algorithm and identify places where two neighboring parameters change from no period to some period $n$.
As noted previously, the Naive global search is not very precise. For that reason, the verification of the parameters near the identified breakpoint is carried out with an algorithm for the detection of periodic orbits.
\par
There are several efficient algorithms for the detection of unstable periodic orbits (UPOs).
These algorithms are able to identify all points $x^{*}$ for which $f(x^{*})^{n}=x^{*}$ for any $n$.
Modern algorithms include the Schmelcher-Diakonos algorithm~\cite{Schmelcher1997}, Davidchack-Lai algorithm~\cite{Davidchack1999, Davidchack2001, Klebanoff2001, Crofts2007} and the Bu-Wang-Jiang algorithm~\cite{Bu2004}.
An example of the Davidchack-Lai algorithm is shown in Figure~\ref{fig:upo_search_example}.

\begin{figure}[!h]
    \centering
    \includegraphics[width=1.0\textwidth]{Figures/upo_search_example.png}
    \caption{
        \textcolor{darkred}{
        Detection of the fixed point of $\mathcal{L}_{r}^{6}$ for $r = 3.7$ using the DL algorithm.
        }
    }
    \label{fig:upo_search_example}
\end{figure}

\par
It is worth mentioning that finding fixed points can also be achieved using standard methods.
For example, the Newton-Raphson algorithm can be used to find the roots of $g(x) = f(x)^{n} - x$.
However, for high $n$ it becomes difficult to find these roots, since there are too many of them and the chance of converging to a stable fixed point decreases.
Comparison of basins of convergence of the Newton-Raphson method, the Schmelcher-Diakonos algorithm, and the Davidchack-Lai algorithm is given in~\cite{Davidchack1999}.
Another example is to use methods from the theory of mathematical optimization.
Minimizing the following function $J(x)= \norm{f^{p}(x)-x}^{2}$ will yield a $p$-periodic point of $f$~\cite{Fuh2009}.
\par
We have decided to include the Bu-Wang-Jiang algorithm (BWJ) in our implementation.
It is easy to implement and works well enough.
The pseudocode for the BWJ algorithm is presented in Algorithm~\ref{alg:bwj}.

\begin{algorithm}[!h]
    \caption{Bu-Wang-Jiang (BWJ)}
    \label{alg:bwj}
    \begin{algorithmic}[1]
        \Statex $f \gets$ map
        \Statex $p \gets$ period
        \Statex $seeds \gets$ seeds
        \Statex $maxiter \gets$ maximum number of iterations
        \Statex $tol \gets$ tolerance for determining a fixed point
        \For{each s in seeds}
            \State $\textbf{x}_{0} \gets s$
            \While{current iteration $< maxiter$}
                \State $J(\textbf{x}_{0}) = \partial f^{p}(\textbf{x}_{0}) / \partial \textbf{x}$ is the Jacobian at $\textbf{x}_{0}$
                \State $Q \gets (cI-J(\textbf{x}_{0}))(J(\textbf{x}_{0})-I)^{-1}$ where $I$ is identity matrix, $c \in (-1, 1)$ is a constant 
                \State $\textbf{x}_1 \gets f^{p}(\textbf{x}_{0}) + Q(f(\textbf{x}_{0})^{p}-\textbf{x}_{0})$
                \If{$\norm{f^{p}(\textbf{x}_1)-\textbf{x}_1} < tol$}
                    \State $\textbf{x}_{1}$ is a fixed point of $f^{p}$
                \EndIf
                \State $\textbf{x}_{0} \gets \textbf{x}_{1}$
            \EndWhile
        \EndFor
    \end{algorithmic}
\end{algorithm}

The $seeds$ parameter is a set of initial conditions from which the algorithm starts searching for fixed points.
They can be chosen as a uniformly sampled interval (if $x \in \mathbb{R}$) or grid (if $x \in \mathbb{R}^{n}$) in the state space.
Our version of the algorithm includes several optimizations suitable for our use case.
BWJ is able to find all fixed points of a system but only the stable fixed points are of interest are needed.
Logistic map is known to have at most one stable periodic orbit for each parameter $r$.
That means that after finding $n$ stable fixed points for the system $\mathcal{L}_{r}^{n}$ the algorithm can be terminated.
After finding a fixed point $x^{*}$ of $\mathcal{L}_{r}^{n}$, iterations of $x^{*}$ $n$ times are performed and check whether resulting points are also fixed points is conducted.
Before the algorithm starts, a similar idea as in the Naive global search is used.
The system $f^{n}(s)$ is iterated $m$ times for arbitrary initial seed $s$ and check whether the resulting point is $n$-periodic is performed.
These optimizations make the algorithm more efficient.
The advantage of using the BWJ algorithm to determine the periodicity of a system $f_{p}$ is that it is reliable even if $f_{p}$ exhibits intermittency for the parameter $p$.
\par
This section has introduced an algorithm for performing a global search for breakpoints in the parameter space.
First, discretization of the parameter space is performed, and the Naive global search is used to approximate periodicity of each parameter.
Afterwards, neighboring pairs with non-periodicity on the left side and periodicity on the right side are identified.
It is verified that the Naive global search determined the periodicity of these pairs correctly using the BWJ algorithm, which is not prone to the effects of intermittency.
Finally, ranges of parameters where a breakpoint is expected are identified.
Example of the algorithm results are shown in Figure~\ref{fig:bif_diag_search_example}.

\begin{figure}[!h]
    \centering
    \includegraphics[width=1.0\textwidth]{Figures/bif_diag_search_example.png}
    \caption{
        \textcolor{darkred}{
        Global search of $\mathcal{L}_{r}$.
        The bifurcation diagram consists of projections of $\mathcal{T}_{900}^{1000}(\mathcal{L}_{r}, 0.5)$.
        Red lines: detected breakpoints up to period $16$.
        }
    }
    \label{fig:bif_diag_search_example}
\end{figure}

\section{Local Search}
The previous section introduced a so-called global search, an approach to find approximate intervals in the parameter space where a breakpoint occurs.
This section introduces an algorithm to find the exact location of each breakpoint.
These locations will be later used to color the bifurcation diagram in the proximity of the breakpoint.
\par
\textcolor{red}{
This section discusses three algorithms for the local search - Naive Local Search, Nested-Layer Particle Swarm Optimization and Local Search using Global Optimization.
The global search has already identified two boundaries $(p_A, p_B)$, period of the system with parameter $p_B$ and the fact that there is a breakpoint between $(p_A, p_B)$.
Subsequently, a better approximation of the parameter at which the breakpoint occurs is needed.
The three algorithms are able to approximate the parameter using the information from the global search.
All three algorithms solve the problem, but they use different approaches.
}

\subsection{Naive Local Search}
The fundamental idea behind Naive Local Search (NLS) is similar to the one used in global search.
Parameter space will be searched through, and the period of the system for each parameter value will be determined.
However, instead of performing the search for the entire parameter space, the search will be performed only in the intervals found in the global search.
These intervals are special because several facts about them are known.
Their left boundary is non-periodic, and their right boundary is periodic of some period $n$.
The $n$ is known from the global search.
The goal is to push the left boundary close to the left side of the breakpoint and the right boundary close to the right side of the breakpoint.
An example of the NLS is shown in Figure~\ref{fig:break_point_search_example}.

\begin{figure}[!h]
    \centering
    \includegraphics[width=1.0\textwidth]{Figures/break_point_search_example.png}
    \caption{
        \textcolor{darkred}{
        Naive local search of $\mathcal{L}_{r}$ for $r \in [ 3.825, 3.83 ]$.
        The bifurcation diagram consists of projections of $\mathcal{T}_{900}^{1000}(\mathcal{L}_{r}, 0.5)$.
        Red lines: Left and right estimates of the period $3$ breakpoint.
        Left estimate is found using Algorithm~\ref{alg:local_search} with parameters $p_A = 3.825$, $p_B = 3.83$, $n = 3$ and $maxiter = 20$.
        Right estimate is found analogically.
        }
    }
    \label{fig:break_point_search_example}
\end{figure}



\par
Firstly the left boundary is searched.
It is desirable to get it as close to the breakpoint as possible.
To do this combination of BWJ algorithm and a binary search is used.
For each parameter $p$ in the interval $(p_{A}, p_{B})$ identified during the global search a check whether $f_{p}$ is $n$-periodic or not is performed.
That determines whether the breakpoint is on the left or right side of the parameter.
In the beginning the interval is halved in a way that $[p_{A}, \frac{p_{A}+p_{B}}{2}, p_{B}]$ is obtained.
If $\frac{p_{A}+p_{B}}{2}$ is $n$-periodic then it is known that the breakpoint is in the interval $[\frac{p_{A}+p_{B}}{2}, p_{B}]$.
If $\frac{p_{A}+p_{B}}{2}$ is non-periodic then it is known that the breakpoint is in the interval $[p_{A}, \frac{p_{A}+p_{B}}{2}]$.
Using the new interval the same process is repeated recursively.
The halving process is repeated for some desired number of iterations $maxiter$.
Pseudocode for the NLS of the left boundary is given in Algorithm~\ref{alg:local_search}.

\begin{algorithm}[!h]
    \caption{NLS - left boundary}
    \label{alg:local_search}
    \begin{algorithmic}[1]
        \Statex $f \gets$ map
        \Statex $p_{A} \gets$ left boundary of the parameter space
        \Statex $p_{B} \gets$ right boundary of the parameter space
        \Statex $n \gets$ period of $f{p_{B}}$
        \Statex $maxiter \gets$ number of iterations
        \State $p_{C} \gets \frac{p_{A}+p_{B}}{2}$
        \State $i \gets$ current iteration
        \If{$i = maxiter$}
            \State left boundary of the breakpoint is $p_{A}$
        \EndIf
        \If{$f_{p_{C}}$ is $n$-periodic}
            \State repeat the algorithm with the same arguments except $p_{B} \gets p_{C}$ and $i \gets i+1$
        \Else
            \State repeat the algorithm with the same arguments except $p_{A} \gets p_{C}$ and $i \gets i+1$
        \EndIf
    \end{algorithmic}
\end{algorithm}

\par
The Algorithm~\ref{alg:local_search} searches only for left boundary. Search for right boundary can be done analogically by reversing the direction of the search.
The complete NLS is both the search of the left boundary and the search of the right boundary.
These two boundaries approximate precise location of a breakpoint.
By calculating the average $p_{avg}$ of the left and right boundary, a good approximation of the parameter at which the breakpoint occurs is obtained.
\par
This subsection introduced an algorithm which approximates the parameter at which the breakpoint occurs.
This parameter is also the bifurcation point at which a saddle-node bifurcation occurs.
The algorithm is based on the idea of a binary search and is able to approximate the breakpoint parameter.

\subsection{Nested-Layer Particle Swarm Optimization}
This subsection describes an alternative to the NLS algorithm.
It achieves the same goal as the NLS, but uses a different approach.
This is done by reformulating the original problem as an optimization problem.
\par
The NLS introduced in the previous subsection approximates the parameter at which the breakpoint occurs.
It is based on the idea of a binary search.
However, the search for the breakpoint can be rephrased as a search for a parameter where saddle-node bifurcation occurs.
Matsushita, Kurokawa, and Kousaka~\cite{Matsushita2019} introduced an approach to search for saddle-node bifurcation of a DDS.
Their method uses the Nested-Layer Particle Swarm Optimization (NLPSO) method.
They have previously used the same approach to detect period-doubling bifurcations of a DDS~\cite{Matsushita20170721}.
\par
Understanding how NLPSO detection of saddle-node bifurcation works requires understanding of the Particle Swarm Optimization (PSO).
PSO is a popular population-based evolutionary algorithm.
It tracks several particles which represent potential solution.
Each particle has its own position and velocity. It also tracks its previous best position and score.
Each article moves through the search space based on its velocity and position.
Its movement is also influenced by its previous best position and the best positions of other particles.~\cite{Matsushita2019}
\par
Each particle has a position $pos \in \mathbb{R}^{n}$, a velocity $vel \in \mathbb{R}^{n}$, a best position $bpos \in \mathbb{R}^{n}$ and a best score $bscore \in \mathbb{R}$.
The algorithm also tracks global best position $g_{pos}$ and global best score $g_{score}$.
The pseudocode for the PSO algorithm is given in Algorithm~\ref{alg:pso}.
The Algorithm~\ref{alg:pso} includes several parameters.
These parameters are $w$, $c_{1}$ and $c_{2}$.
They shall be set as $w=0.729$ and $c_{1}=c_{2}=1.494$~\cite{Matsushita2019}.

\begin{algorithm}[!h]
    \caption{Particle Swarm Optimization (PSO)}
    \label{alg:pso}
    \begin{algorithmic}[1]
        \Statex $f \gets$ function to minimize. $f: \mathbb{R}^{m} \rightarrow \mathbb{R}$  
        \Statex $(a, b) \gets$ search-space range
        \Statex $n \gets$ number of particles
        \Statex $maxiter \gets$ maximum number of iterations
        \Statex $tol \gets$ tolerance for determining solution
        \Statex $w, c_{1}, c_{2} \gets$ parameters described in the text
        \State Create $n$ particles.
        \For{each particle}
            \State $pos$ $m$-dimensional vector of uniform random numbers between $a$ and $b$.
            \State $vel$ $m$-dimensional zero vector.
            \State $bpos \gets$ $m$-dimensional vector of uniform random numbers between $a$ and $b$.
            \State $bscore \gets \infty$
        \EndFor
        \State $g_{pos}$ $m$-dimensional vector of uniform random numbers between $a$ and $b$.
        \State $g_{score} \gets \infty$

        \For{iteration less than $maxiter$}
            \For{each particle}
                \State $score \gets f(pos)$ 
                \If{$score < bscore$}
                    \State $bscore \gets score$
                    \State $bpos \gets pos$
                \EndIf
                \If{$score < g_{score}$}
                    \State $g_{score} \gets score$
                    \State $g_{pos} \gets pos$
                \EndIf
            \EndFor
            \If{$g_{score} < tol$}
                \State break the loop
            \EndIf
            \For{each particle}
                \State $r_{1}, r_{2} \gets$ random numbers between $0$ and $1$.
                \State $vel \gets w(vel) + c_{1}r_{1}(bpos-pos) + c_{2}r_{2}(g_{pos}-pos)$
                \State $pos \gets pos + vel$
            \EndFor
        \EndFor
    \end{algorithmic}
\end{algorithm}

\par
The next step is to combine two PSOs.
One of them will be looking for a parameter $p_b$ at which saddle-node bifurcation occurs.
The other one will be looking for periodic point $x_b$ corresponding to $p_b$.
Each PSO requites a function that is to be minimized.
Its worth noticing that the algorithm works for dynamical systems of arbitrary dimensions.
For that reason the minimization functions are presented in a general form.
\par
Let $(\mathbb{R}^{m}, f)$ be a DDS.
Search for a periodic point $x_0$ for an arbitrary parameter $p$ is performed.
After selecting parameter $p$, a search for a periodic point $x_0$ is performed by minimizing the function $F$ from Equation~\eqref{eq:minimize_x}.
If $x_0$ is a fixed point of $f^{n}_{p}$ then $x_0$ is an $n$-periodic point of $f$.
It is a fixed point in case $f^{n}_{p}(x_0)-x_0 = 0$.
Hence, minimizing the $F$ brings us closer to the $n$-periodic point.
PSO can be used to search for optimal parameter $x_0$ by minimizing $F$.

\begin{equation}
\label{eq:minimize_x}
    F(x_0) = \norm{f^{n}_{p}(x_0)-x_0}
\end{equation}

\par
Let $(\mathbb{R}^{m}, f)$ be a DDS.
A search for parameter $p_b$ at which saddle-node bifurcation occurs is performed.
Saddle-node bifurcation occurs at parameter $p$ when the characteristic equation~\ref{eq:characteristic_eq} is equal to $0$, $x_0$ is $n$-periodic point and parameter $\mu = 1$.
% Aditionally period-doubling bifurcation occurs when the when the characteristic equation~\ref{eq:characteristic_eq} is equal to $0$ and parameter $\mu = -1$.

\begin{equation}
\label{eq:characteristic_eq}
        \text{det}(Df^{n}_{p}(x_0)-\mu I)
\end{equation}
Hence minimize function $G_{mock}(p_0) = |\text{det}(Df^{n}_{p_0}(x_0)-\mu I)|$ for different parameters $p$ would search for saddle-node bifurcation $p_b$.
However, in order to evaluate $G_{mock}$ it is needed to know $x_0$ which is currently unknown.
Thus, a PSO can be used for minimizing~\ref{eq:minimize_x} to find $x_0$ and use $p_0$ as a system parameter that was described in the previous paragraph.
It is important to be sure that found point $x_0$ is $n$-periodic point and thus $G_{mock}$ is only evaluated if PSO converged correctly to $x_0$.
Updated function $G_{mock}$ is presented in Equation~\eqref{eq:minimize_p} and is now called $G$.
Thus minimizing the function $G$ from Equation~\eqref{eq:minimize_p} by PSO finds optimal parameter $p_b$.

\begin{equation}
\label{eq:minimize_p}
    G(p) =
    \begin{cases}
        |\text{det}(Df^{n}_{p}(x_0)-\mu I)| & \text{if } F(x_0) < C_{pp}, \\
        \infty & \rm{otherwise}
    \end{cases}
\end{equation}

\par
The previous paragraphs described how to use the two types of PSOs.
The main PSO looks for parameter $p_b$ at which saddle-node bifurcation occurs.
It does so by minimizing function $G$ from~\ref{eq:minimize_p}.
During every evaluation of $G$ at parameter $p_0$ it is needed to find $x_0$ which is $n$-periodic point of $f^{n}_{p_0}$.
That can be achieved by using the second nested PSO which minimizes function $F$ from~\ref{eq:minimize_x} by using $p_0$ as parameter in $F$.
This nesting of PSOs is the reason why this algorithm is called Nested-Layer Particle Swarm Optimization.
Period $n$ is the same in both functions $F$ and $G$.
It corresponds to the period of the system at the saddle-node bifurcation.
Fortunately the parameter $n$ is known from the global search.
Parameter range $(p_A, p_B)$ is also known from the global search and is used in the top-level PSO as search-space range.
The search-space range used for the nested PSO is also known.
It is because the system is invariant under some set.
For example for the Logistic equation, the search-space range can be set to $[0, 1]$.
\par
This subsection described the NLPSO algorithm as an alternative to NLS algorithm.
Both of these algorithms can be used interchangeably when dealing with 1D systems.
However, NLPSO is more general and can be used for higher dimensional systems as well.
In general NLPSO is much better algorithm than NLS.
Nevertheless, it is also harder to implement and understand.

\subsection{Local Search using Global Optimization}
\textcolor{red}{
The previous subsection introduced the NLPSO algorithm for finding saddle-node bifurcations.
It achieves its goal by formulating the problem as an optimization problem and by nesting two optimization algorithms into each other.
However, there exists another approach which also uses global optimization approach.
This approach shall be called Local Search using Global Optimization (LSGO).
}
\par
\textcolor{red}{
The LSGO approach formulates the saddle-node bifurcation detection as an optimization problem.
However, it uses different formulation that the NLPSO.
As mentioned earlier, detecting saddle-node bifurcations helps to find the regions exhibiting the type-I intermittency.
}
\par
\textcolor{red}{
The LSGO uses the following formulation:
}
\begin{align*} 
x_0 &= f_{p}^{n}(x_0)\\
x_1 &= f_{p}^{n}(x_1)\\
\vdots \\
x_n &= f_{p}^{n}(x_n)\\
\end{align*}

\textcolor{red}{
In the presented formulation $x_0$ is an $n$-periodic point of $f_{p}$.
Hence, there are $n$ equations of the form $x_i = f_{p}^{n}(x_i)$ which hold.
These equations can further be formulated to depend only on the parameter $p$ and point $x_0$.
}

\begin{align*} 
f_{p}^{0}(x_0) &= f_{p}^{n}(f_{p}^{0}(x_0)) \\
f_{p}^{1}(x_0) &= f_{p}^{n}(f_{p}^{1}(x_0)) \\
\vdots \\
f_{p}^{n}(x_0) &= f_{p}^{n}(f_{p}^{n}(x_0)) \\
\end{align*}

\textcolor{red}{
Furthermore, if $p$ is a saddle-node bifurcation parameter then for all $\{ x_{i}: i \in [1, n] \}$ it holds $(f_{p}^{n})'(x_i) = 1$. (In one dimensional case.)
Thus, these $n$ equations hold:
}

\begin{align*} 
(f_{p}^{n})'(f_{p}^{0}(x_0)) &= 1 \\
(f_{p}^{n})'(f_{p}^{1}(x_0)) &= 1 \\
\vdots \\
(f_{p}^{n})'(f_{p}^{n}(x_0)) &= 1 \\
\end{align*}

\textcolor{red}{
Combining all these equations yields a system of $2n$ equations.
Putting all terms on one side results in:
}

\begin{align*} 
f_{p}^{n}(f_{p}^{0}(x_0)) - f_{p}^{0}(x_0) &= 0 \\
f_{p}^{n}(f_{p}^{1}(x_0)) - f_{p}^{1}(x_0) &= 0 \\
\vdots \\
f_{p}^{n}(f_{p}^{n}(x_0)) - f_{p}^{n}(x_0) &= 0 \\
(f_{p}^{n})'(f_{p}^{0}(x_0)) - 1 &= 0 \\
(f_{p}^{n})'(f_{p}^{1}(x_0)) - 1 &= 0 \\
\vdots \\
(f_{p}^{n})'(f_{p}^{n}(x_0)) - 1 &= 0 \\
\end{align*}

\textcolor{red}{
Finding analytical solution to the system of equations would yield $p$ at which the saddle-node bifurcation occurs.
Additionally, $x_0$ would be found as well and could be used to retrieve the whole periodic orbit.
However, finding analytical solution might not be possible and feasible.
Using tools from theory of optimization might be a better approach.
}
\par
\textcolor{red}{
Let's define a function $H: \mathbb{R}^{2} \rightarrow \mathbb{R}$ as follows:
}

\begin{multline*} 
H(p, x_0) = (f_{p}^{n}(f_{p}^{0}(x_0)) - f_{p}^{0}(x_0))^2 + \dots + (f_{p}^{n}(f_{p}^{n}(x_0)) - f_{p}^{n}(x_0))^2 + \\
    + ((f_{p}^{n})'(f_{p}^{0}(x_0)) -1)^2 + \dots + ((f_{p}^{n})'(f_{p}^{n}(x_0)) -1)^2
\end{multline*}

\textcolor{red}{
Function $H$ has global minimum of $0$ at point $(p, x_0) \in \mathbb{R}^{2}$.
By minimizing this function, such point can be found.
By setting tolerance criterion for the minimization algorithm low enough, it can be decided whether $p$ is indeed a parameter of saddle-node bifurcation or not.
}
\par
\textcolor{red}{
Note that the search space might not be the whole $\mathbb{R}^{2}$.
For instance in the case of the Logistic map the whole search space is $[0, 4] \times [0, 1]$ since $x \in [0,1]$ and $p \in [0,4]$.
However, the search space can be reduced even further.
The global search has already identified the intervals $[p_A, p_B]$ in the parameter space where the saddle-node bifurcation occurs.
Hence, the search space can be reduced to $[p_A, p_B] \times [0, 1]$.
If the optimization converges with enough precision, then it can be supposed that the parameter at which the breakpoint occurs has been found.
}
\par
\textcolor{red}{
Furthermore, note that different types of optimization algorithms can be used.
For example, the Particle Swarm Optimization (PSO) algorithm described earlier can be used.
In our implementation we used the Evolutionary Centers Algorithm~\cite{MejiadeDios20190913} (ECA)  which has proven to be the most effective.
}
\par
\textcolor{red}{
This subsection introduced another approach for the Local search.
This approach formulated the problem as an optimization problem.
Subsequently, it used global optimization algorithms to find the parameter at which the saddle-node bifurcation occurs.
}

% \section{UPO detection}
% Our goal is to ultimately find regions in the bifurcation diagram where intermittency type-I occurs.
% As noted in chapter~\ref{chap:intermittency_review} we suppose that intermittency type-I is present at "breakpoints" in the bifurcation diagram.
% Such breakpoint is given for example in Figure~\ref{fig:complex_logistic}.
% \par
% The breakpoints are characterized by shift from chaotic to periodic behavior.
% Periodic behavior on one side of the breakpoint in the neighborhood of it always has some fixed period $n$.
% Hence the system has $n$ stable fixed points on one side of the breakpoint.
% On the other side the behavior is dynamic and there are no stable fixed points.
% If we could quickly determine whether system for chosen parameter $p$ is periodic or not we could use this knowledge to find the breakpoints.
% The reason for that is that on one side of the breakpoint the check of periodicity would always yield infinity and on the other side it would yield period $n$.
% We could use this knowledge to get arbitralily close to the breakpoint.
% \par
% To check whether a system $f_P$ is $n$ periodic we just need to determine the number of stable fixed points of system $f^{n}_P$.
% If number of the stable fixed points is $n$ then system $f_P$ is $n$ periodic.
% If we know that $x_0$ is a stable fixed point of $f^{n}_P$ then we know that it is $n$ periodic point.
% The remaining $n-1$ points of the stable periodic orbit can be found by iterating $f_P(x_0)$ $(n-1)$-times.
% However how do we reliably and algorithmically get the first periodic orbit?
% \par
% Finding a fixed point of $f^{n}_{P}$ is a question of finding a root of $g(x) = f^{n}_{P}(x) - x$.
% Standard root finding algorithm such as Newton-Raphson algorithm can be used to finds roots of $g(x)$.
% However for high $n$ it becomes difficult to find these roots since there is too many of them and a chance of converging to a stable fixed point decreases.
% Since our algorithm requires certainty that we found complete stable periodic orbit we decided to take another approach.
% \par
% Other option is to find all periodic orbits and determine their stability.
% There exist quick algorithm to determine all periodic orbits.
% By using it we will always have certainty that we found all stable periodic orbits.

% \subsection{Schmelcher-Diakonos algorithm}
% Schmelcher and Diakonos came up with a general algorithm for detecting complete set of unstable periodic orbits.
% The algorithm transforms the original system so that its unstable fixed points become stable.
% Additionally their basins of attraction are long and convergence to each unstable orbit is probable.
% However number of system transformations grow very quickly for high dimentional systems and computation becomes too long.
% At the same time parameter $\lambda$ has to be specified correctly to ensure convergence but there is no general way of how to choose this parameter.~\cite{Schmelcher1997}

% \subsection{Davidchack-Lai algorithm}
% Davidchack and Lai extended SD algorithm so that it works as a hybrid between Newton-Raphson algorithm and SD algorithm.
% They propose that to find UPOs of period $n$ one can use UPOs of period $n-1$ and $n+1$.
% This technique is supposed to reduce number of iterations significantly compared to SD algorithm.~\cite{Davidchack1999}

% \subsection{Our implementation}
% DL algorithm is supposed to be superior to SD algorithm.
% Even though there the DL algoritm is described in~\cite{Davidchack1999,Davidchack2001,Klebanoff2001} we were not able to implement it in a way that would fulfill its advertised performance.
% Since the algorithm is over 20 years old and there are no implementations of it online we had to resort to our own version.
% \par
% Our version simply the DL iteration scheme but does not use previously UPOs as seeds.
% We choose seeds uniformly from an interval as it was suggested for SD algorithm~\cite{Schmelcher1997}.
% We use efficient binary search tree structure which makes the algorithm efficient enough for our use case.

\section{Coloring}
\textcolor{red}{
The goal of this phase is to color the neighborhood of the breakpoint to warn about its ambiguity.
Parameters at which the breakpoint occurs were found in the previous steps.
Theory of type-I intermittency is used to color the neighborhood of the breakpoint.
}

\subsection{Description of the algorithm}
\textcolor{red}{
The local search finds parameters of the parameter space at which the breakpoint occurs.
Let $p_b$ be one of them.
It is known that parameters to the left side $p_b$ in its proximity exhibit intermittency.
Nevertheless, there is a need to quantify how far from $p_b$ is intermittency still occurring.
Additionally, it is needed to measure how prominent the intermittency is for some $p$ near $p_b$.
}
\par
\textcolor{red}{
A useful indicator about the intermittent behavior is average laminar length $l_{avg}$ described in Section~\ref{sec:type-I intermittency}.
The characteristic relation $l_{avg} \varpropto 1 / \sqrt{\varepsilon}$ can be used to estimate the average laminar length.
Characteristic relation is dependent on the local map around fixed points associated with saddle-node bifurcation.
Fortunately, these fixed point have been found in the local search.
Hence, by calculating $\varepsilon$ for each stable fixed point of $f_{p_b}$, $l_{avg}$ can be estimated for laminar region around each fixed point.
Taking their average yields an estimation of the average laminar length for $p_b$.
}
\par
\textcolor{red}{
Depending on the parameter $p$ the laminar phases have various lengths.
Far from to the left of $p_b$ the laminar phases are about $50$ iterations long.
Very close to the left of $p_b$ the length of the laminar phase can get arbitrary long.
The closer the parameter $p$ gets to $p_b$ the longer the laminar phase gets.
}
\par
\textcolor{red}{
To colorize the diagram it is needed to determine the average laminar length for the parameters $p$ to the left of $p_b$ and to compare them.
However, the average laminar lengths can get arbitrary large.
It is convenient to color only parameters $p$ that have average laminar lengths between $L$ and $U$. 
$L$ denotes the lower bound and $U$ denotes the upper bound.
For example $L$ can be set to $100$ and $U$ to $1000$.
The next step is to find parameters $p_L$ and $p_U$ such that parameters $p$ between $p_L$ and $p_U$ have $l_{avg}$ from range $[ L, U ]$.
}
\par
\textcolor{red}{
The idea behind an algorithm to find $p_L$ and $p_U$ is to step to the left from $p_b$ with incrementally smaller steps.
The algorithm is illustrated in Algorithm~\ref{alg:optimal_bound}
}

\begin{algorithm}[!h]
    \caption{Optimal bound search}
    \label{alg:optimal_bound}
    \begin{algorithmic}[1]
        \Statex $f \gets$ map
        \Statex $p \gets$ initial bound estimate
        \Statex $s \gets$ step
        \Statex $B \gets$ desired average laminar length
        \Statex $t \gets$ tolerance
        \Statex $i_{m} \gets$ maximum of iterations
        \Statex $i_{c} \gets$ current iteration

        \For{$i$ from $i_{c}$ to $i_{m}$}
            \State $p \gets p - s$
            \State $l_{avg} \gets$ average laminar length for $p$
            \If{$B-t \leq l_{avg} \leq B+t$}
                \State Terminate program with $p$ as the optimal bound.
            \EndIf
            \If{$l_{avg} < B-t$}
                \State Repeat the program for $p = p+s$, $s = s/2$ and $i_{c} = i+1$. The rest of the parameters remain unchanged.
            \EndIf
        \EndFor
    \end{algorithmic}
\end{algorithm}

\textcolor{red}{
The Algorithm~\ref{alg:optimal_bound} can be used for finding both $p_L$ and $p_U$.
The algorithm can be started with $p = p_b$, $s = 0.1$, $i_{m} = 200$, $i_{c} = 0$, $t = 10$.
For finding $p_L$, initial bound estimate $B = 100$.
For finding $p_U$, initial bound estimate $B = 1000$.
This way both $p_L$ and $p_U$ is obtained.
}
\par
\textcolor{red}{
To finish the coloring procedure, bifurcation diagram is constructed.
A color range is defined for numbers in range $[ l_{avg}$ for $p_L, l_{avg}$ for $p_U ]$.
The trajectory projections for parameters $p \in [ p_L, p_U ]$ are colored using this color range.
The same coloring procedure is repeated for all breakpoint parameters found in the local search.
}
\par
\textcolor{red}{
The result of the coloring algorithm for the Logistic map is shown in Figure~\ref{fig:coloring_example}.
Note that when the parameter space is large, the coloring of individual breakpoint is not very useful.
However, the user can always zoom in to the breakpoint if needed and then the coloring becomes useful.
}

\begin{figure}[!h]
    \centering
    \includegraphics[width=0.8\textwidth]{Figures/logistic_map_coloring_example.png}
    \caption{
        \textcolor{darkred}{
        Colorization of a single breakpoint of $\mathcal{L}_{r}$ for $r \in I := [ 3.6263, 3.6267 ]$.
        Algorithm~\ref{alg:optimal_bound} was used to specify the colorization bounds.
        The bifurcation diagram consists of projections $\mathcal{T}_{900}^{1000}(\mathcal{L}_{r}, 0.5)$ for $r \in I$.
        }
    }
    \label{fig:coloring_example}
\end{figure}

\section{Complete algorithm}
\textcolor{red}{
In the previous sections each part of the algorithm was described.
It was also pointed out how each part of the algorithm relates to the other parts.
For clarity, this section explains how to combine all the parts together.
}
\par
\textcolor{red}{
First, the Global search is used to identify intervals where breakpoint could occur.
Secondly, the Local search is employed for each of the intervals to find precise parameter, periodicity and the fixed points of the breakpoint.
Three alternatives for the Local search were introduced.
Any one of them can be used.
Lastly, areas next to the found breakpoint are colored using the Coloring method.
The result is a bifurcation diagram with identified breakpoint whose left neighborhoods are colored.
}
\par
\textcolor{red}{
An example of the complete algorithm is shown in Figure~\ref{fig:complete_colorization}
}

\begin{figure}[!h]
    \centering
    \includegraphics[width=0.9\textwidth]{Figures/complete_colorization.png}
    \caption{
        \textcolor{darkred}{
        Full Colorization of $\mathcal{L}_{r}$ for $r \in [ 3.62, 3.65 ]$.
        The bifurcation diagram consists of projections of $\mathcal{T}_{900}^{1000}(\mathcal{L}_{r}, 0.5)$.
        Colorful lines: areas near the detected breakpoints (up to period $20$).
        }
    }
    \label{fig:complete_colorization}
\end{figure}

\endinput