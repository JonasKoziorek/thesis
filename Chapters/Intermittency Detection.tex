\chapter{Intermittency Detection}
Last chapter introduced the problem of ambiguity of bifurcation diagram.
This chapter introduces an algorithm that aims to detect the ambiguity.
The algorithm consists of 3 core parts

\begin{description}
	\item [Global Search] -- searching for areas where breakpoint occurs
	\item [Local Search] -- searching for precise location of the breakpoint
	\item [Coloring] -- coloring the bifurcation diagram in proximity of the breakpoint
\end{description}

Each of these parts are described in the following sections.
The last section of this chapter combines all the parts together.

\section{Global Search}
The initial phase of the algorithm aims to find approximate parameter intervals during which type-I intermittency occurs.
More specifically, parameter intervals containing a breakpoint are identified.
\par
A breakpoint is a parameter value at which the system transitions from chaotic/intermittent behavior to periodic one.
The breakpoints we are interested in have chaotic behaviour on the left side and periodic behavior on the right side.
Each breakpoint can be identified by a unique number $n$ such that on the right side the behavior is $n$-periodid.
If we could search through the parameter space and determine the period of the system for each parameter value we could easily find the breakpoints.
To do that we would need an algorithm to determine the period of the system for any given parameter value.
\par
If system has stable $n$-periodic orbit or the system starts evolving at unstable $n$-periodic point then the system is $n$-periodic or eventually $n$-periodic.
We know that logistic map $\mathbb{L}_p$ can only have at most one unique stable $n$-periodic orbit for parameter $p$.
That means that if we find a stable $n$-periodic orbit for parameter $p$ then we know that $\mathbb{L}_p$ is $n$-periodic.
At least in the case where initial condition isn't an unstable periodic point.
To find out whether $\mathbb{L}_p$ has a stable $n$-periodic orbit we need to find fixed points of $\mathbb{L}_p, \mathbb{L}_{p}^{2}, \mathbb{L}_{p}^{3}, \dots$ and determine their stability.
Hence a smallest $n$ for which $\mathbb{L}_{p}^{n}(x^{*})=x^{*}$ and $x^{*}$ is stable is the period of $\mathbb{L}_p$.
\par
There exist algorithms for finding all fixed points of a system.
These algorithms are quite efficient and can be used to for high-dimensional systems.
However, our aim is to search through the whole parameter space in search of breakpoints.
This search requires to find fixed points for hundreds or thousands of parameters in the parameter space.
Even though existing algorithms search for fixed points efficiently, they are not efficient enough for our use case.
They will come in handy later in the algorithm but for now we need to find a way to approximate period of a system simply.
\par
When we see a periodic behavior in a bifurcation diagram, it means that the system is periodic or eventually periodic for a specific initial condition $x_0$.
In other words there exists $m > 0$ such that $f^{i}(x)=x$ is periodic for $i \geq m$.
A naive approach to determine whether a system is periodic is to simply iterate the system for some large enough $m$ and check whether $f^{m}(x)=f^{m+i}(x)$.
That implies that point $x$ is eventually periodic (or periodic) of period $i$.
\par
Simple check presented in the last paragraph fails to be precise in areas with intermittency.
The first $m$ iterations could end up in the laminar phase which might be deemed as periodic.
It also fails when the system is eventually periodic for some $n > m$, eg. higher than our parameter $m$.
Regardless of it's flaws this simple check seems to be good enough for our naive search through the parameter space.
It's much quicker than finding all fixed points for each parameter and it's precise enough for our use case.
Pseudocode for the naive global search is presented in Algorithm \ref{alg:naive_global_search}.

\begin{algorithm}[!h]
    \caption{Naive global search}
    \label{alg:naive_global_search}
    \begin{algorithmic}[1]
        \Statex $f \gets$ system
        \Statex $p_{A} \gets$ left boundary of the parameter space
        \Statex $p_{B} \gets$ right boundary of the parameter space
        \Statex $n \gets$ number of samples in the parameter space
        \Statex $m \gets$ number of iterations
        \Statex $o \gets$ maximum period to check
        \Statex $x_0 \gets$ initial condition
        \State $p \gets p_{A}$
        \State $p_{step} \gets \frac{p_{B} - p_{A}}{n}$
        \While{$p \leq p_{B}$}
            \State $x_1 \gets f_{p}^{m}(x_0)$
            \For{$i \gets 1$ to $o$}
                \State $x_2 \gets f_{p}^{i}(x_1)$
                \If{$x_2 = x_1$}
                    \State $f_p$ is $i$-periodic
                \EndIf
            \EndFor
            \State $p \gets p + p_{step}$
        \EndWhile
    \end{algorithmic}
\end{algorithm}

With this algorithm we approximate the period of a system for each parameter in sampled parameter space.
If the algorithm doesn't find any periodicity for some parameter $p$ then we can assume that the system is chaotic for this parameter.
Our goal is to look for breakpoints, eg. regions where there is a shift from chaotic (non-periodic) behavior to periodic one.
To do that we look at the results from the previous algorithm and identify places where two neighboring parameters change from no period to some period $n$.
As noted previously, the Naive global search is not very precise. For that reason we will verify the parameters near the identified breakpoint with an algorithm for detection of periodic orbits.
\par
There are several efficient algorithms for unstable periodic orbits (UPOs) detection.
These algorithms are able to identify all points $x^{*}$ for which $f(x^{*})^{n}=x^{*}$ for any $n$.
The modern algorithms include Schmelcher-Diakonos algorithm \cite{Schmelcher1997}, Davidchack-Lai algorithm \cite{Davidchack1999,Davidchack2001,Klebanoff2001} and Bu-Wang-Jiang algorithm \cite{Bu2004}.
An example of the results of the algorithm is shown in the Figure \ref{fig:upo_search_example}.

\begin{figure}[!h]
    \centering
    \includegraphics[width=1.0\textwidth]{DDS/Figures/upo_search_example.png}
    \caption{Fixed points of $\mathbb{L}_{3.7}^{6}$}
    \label{fig:upo_search_example}
\end{figure}

\par
It is worth mentioning that finding fixed points can be achiaved using standard methods too.
For example Newton-Raphson algorithm can be used to find roots of $g(x) = f(x)^{n} - x$.
However for high $n$ it becomes difficult to find these roots since there is too many of them and a chance of converging to a stable fixed point decreases.
Comparison of basins of convergence of the Newton-Raphson method, Schmelchel-Diakonos algorithm and Davidchack-Lai algorithm can be seen in \cite{Davidchack1999}.
Another example is to use methods from theory of mathematical optimization.
Minimizing the following function $J(x)= \norm{f^{p}(x)-x}^{2}$ will yield a $p$-periodic point of $f$ \cite{Fuh2009}.
\par
We have decided to include the Bu-Wang-Jiang algorithm (BWJ) in our implementation.
It's easy to implement and works well for our use case.
The pseudocode for BWJ algorithm is presented in Algorithm \ref{alg:bwj}.

\begin{algorithm}[!h]
    \caption{Bu-Wang-Jiang (BWJ)}
    \label{alg:bwj}
    \begin{algorithmic}[1]
        \Statex $f \gets$ system
        \Statex $p \gets$ period
        \Statex $seeds \gets$ seeds
        \Statex $maxiter \gets$ maximum number of iterations
        \Statex $tol \gets$ tolerance for determining a fixed point
        \For{each s in seeds}
            \State $\textbf{x}_{0} \gets s$
            \While{current iteration $< maxiter$}
                \State $J(\textbf{x}_{0}) = \partial f^{p}(\textbf{x}_{0}) / \partial \textbf{x}$ is jacobian at $\textbf{x}_{0}$
                \State $Q \gets (cI-J(\textbf{x}_{0}))(J(\textbf{x}_{0})-I)^{-1}$ where $I$ is identity matrix, $c \in (-1, 1)$ is a constant 
                \State $\textbf{x}_1 \gets f^{p}(\textbf{x}_{0}) + Q(f(\textbf{x}_{0})^{p}-\textbf{x}_{0})$
                \If{$\norm{f^{p}(\textbf{x}_1)-\textbf{x}_1} < tol$}
                    \State $\textbf{x}_{1}$ is a fixed point of $f^{p}$
                \EndIf
                \State $\textbf{x}_{0} \gets \textbf{x}_{1}$
            \EndWhile
        \EndFor
    \end{algorithmic}
\end{algorithm}

The $seeds$ parameter is a set of initial conditions from which the algorithm starts searching for fixed points.
They can be chosen as a uniformly sampled interval (if $x \in \mathbb{R}$) or grid (if $x \in \mathbb{R}^{n}$) in the state space.
Our version of the algorithm includes several optimizations suitable for our use case.
JBW will find all fixed points of a system but we are only looking for fixed points that are stable.
We are working with the logistic map for which we know that there can be at most one stable periodic orbit.
That means that if we find $n$ stable fixed points for the system $\mathbb{L}_{p}^{n}$ we terminate the algorithm.
If we find a fixed point $x^{*}$ of $\mathbb{L}_{p}^{n}$, we iterate $x^{*}$ $n$ times and check whether resulting points are also fixed points.
Before the algorithm starts we use similar idea as in the Naive global search. We iterate the system $f^{n}(s)$ $m$ times for arbitrary ititial seed $s$ and check whether the resulting point is $n$-periodic.
These optimizations make the algorithm more efficient for our use case.
The advantage of using BWJ algorithm to determine periodicity of a system $f_{p}$ is that it is reliable even if $f_{p}$ exhibits intermittency for parameter $p$.
\par
In this section we have introduced an algorithm to perform global search for breakpoints in the parameter space.
First we discretize the parameter space and we use the Naive global search to approximate periodicity of each parameter.
Then we look at neighboring pairs of parameters where there is nonperiodicity on the left side and periodicity on the right side.
We verify that Naive global search determined periodicity for these pairs correctly using BWJ algorithm which isn't prone to effects of intermittency.
In the end we have ranges of parameters where we expect a breakpoint to occur.
Example of the results of the algorithm are shown in the Figure \ref{fig:bif_diag_search_example}.

\begin{figure}[!h]
    \centering
    \includegraphics[width=1.0\textwidth]{DDS/Figures/bif_diag_search_example.png}
    \caption{Example of a Global search}
    \label{fig:bif_diag_search_example}
\end{figure}

\section{Local Search}
The previous section introduced a so called global search, an approach to find approximate intervals in the parameter space where a breakpoint occurs.
In this section we will develop an algorithm to find the precise location of each breakpoint.
These locations will be later used to color the bifurcation diagram in proximity of the breakpoint.
\par
The fundamental idea behind the local search is similar to the one used in the global search.
We will search through the parameter space and determine the period of each parameter.
However, instead of searching through the whole parameter space we will only search through the intervals found in the global search.
These intervals are special because we know several facts about them.
Their left boundary in nonperiodic and their right boundary is periodic of period some $n$. We know this $n$ from the global search.
We would like to push the left boundary close to the left side of the breakpoint.
The same applies for the right boundary.
On example result of a local search can be seen in Figure \ref{fig:break_point_search_example}.

\begin{figure}[!h]
    \centering
    \includegraphics[width=1.0\textwidth]{DDS/Figures/break_point_search_example.png}
    \caption{Example result of the local search}
    \label{fig:break_point_search_example}
\end{figure}



\par
We will start with the left boundary. We want to get it as close to the breakpoint as possible.
To do this we will use combination of BWJ algorithm and a binary search.
For each parameter $p$ in the interval $(p_{A}, p_{B})$ identified during the global search we are able to determine whether $f_{p}$ is $n$-periodic or not.
That will tell us whether we are on the left or right side of the breakpoint.
We start by halving the interval so that we have $[p_{A}, \frac{p_{A}+p_{B}}{2}, p_{B}]$.
If $\frac{p_{A}+p_{B}}{2}$ is $n$-periodic then we know that the breakpoint is in the interval $[\frac{p_{A}+p_{B}}{2}, p_{B}]$.
If $\frac{p_{A}+p_{B}}{2}$ is nonperiodic then we know that the breakpoint is in the interval $[p_{A}, \frac{p_{A}+p_{B}}{2}]$.
With the new interval we perform the same process again recursively.
We repeat the halving process for some desired number of iterations $maxiter$.
Pseudocode for the local search of the left boundary can be seen in Algorithm \ref{alg:local_search}.

\begin{algorithm}[!h]
    \caption{Local search - left boundary}
    \label{alg:local_search}
    \begin{algorithmic}[1]
        \Statex $f \gets$ system
        \Statex $p_{A} \gets$ left boundary of the parameter space
        \Statex $p_{B} \gets$ right boundary of the parameter space
        \Statex $n \gets$ period of $f{p_{B}}$
        \Statex $maxiter \gets$ number of iterations
        \State $p_{C} \gets \frac{p_{A}+p_{B}}{2}$
        \State $i \gets$ current iteration
        \If{$i = maxiter$}
            \State left boundary of the breakpoint is $p_{A}$
        \EndIf
        \If{$f_{p_{C}}$ is $n$-periodic}
            \State repeat the algorithm with the same arguments except $p_{B} \gets p_{C}$ and $i \gets i+1$
        \Else
            \State repeat the algorithm with the same arguments except $p_{A} \gets p_{C}$ and $i \gets i+1$
        \EndIf
    \end{algorithmic}
\end{algorithm}

\par
The Algorithm \ref{alg:local_search} searches only for left boundary. Search for right boundary can be done analogically.
The complete local search is the local search of the left boundary and local search of the right boundary.
These two boundaries approximate precise location of a breakpoint.
By taking their average $p_{avg}$ we get a good approximation of the parameter at which the breakpoint occurs.
\par
In this section we have introduced an algorithm to which gives good approximation of the parameter at which the breakpoint occurs.
This parameter is also the bifurcation point at which a saddle-node bifurcation occurs.
This saddle-node bifurcation is the reason of the type-I intermittency on the right side of the break point.
The next section will describe how to visually warn about the locations of type-I intermittency.

% \section{UPO detection}
% Our goal is to ultimately find regions in the bifurcation diagram where intermittency type-I occurs.
% As noted in chapter \ref{chap:intermittency_review} we suppose that intermittency type-I is present at "breakpoints" in the bifurcation diagram.
% Such breakpoint can be seen for example in Figure \ref{fig:complex_logistic}.
% \par
% The breakpoints are characterized by shift from chaotic to periodic behavior.
% Periodic behavior on one side of the breakpoint in the neighborhood of it always has some fixed period $n$.
% Hence the system has $n$ stable fixed points on one side of the breakpoint.
% On the other side the behavior is dynamic and there are no stable fixed points.
% If we could quickly determine whether system for chosen parameter $p$ is periodic or not we could use this knowledge to find the breakpoints.
% The reason for that is that on one side of the breakpoint the check of periodicity would always yield infinity and on the other side it would yield period $n$.
% We could use this knowledge to get arbitralily close to the breakpoint.
% \par
% To check whether a system $f_P$ is $n$ periodic we just need to determine the number of stable fixed points of system $f^{n}_P$.
% If number of the stable fixed points is $n$ then system $f_P$ is $n$ periodic.
% If we know that $x_0$ is a stable fixed point of $f^{n}_P$ then we know that it is $n$ periodic point.
% The remaining $n-1$ points of the stable periodic orbit can be found by iterating $f_P(x_0)$ $(n-1)$-times.
% However how do we reliably and algorithmically get the first periodic orbit?
% \par
% Finding a fixed point of $f^{n}_{P}$ is a question of finding a root of $g(x) = f^{n}_{P}(x) - x$.
% Standard root finding algorithm such as Newton-Raphson algorithm can be used to finds roots of $g(x)$.
% However for high $n$ it becomes difficult to find these roots since there is too many of them and a chance of converging to a stable fixed point decreases.
% Since our algorithm requires certainty that we found complete stable periodic orbit we decided to take another approach.
% \par
% Other option is to find all periodic orbits and determine their stability.
% There exist quick algorithm to determine all periodic orbits.
% By using it we will always have certainty that we found all stable periodic orbits.

% \subsection{Schmelcher-Diakonos algorithm}
% Schmelcher and Diakonos came up with a general algorithm for detecting complete set of unstable periodic orbits.
% The algorithm transforms the original system so that its unstable fixed points become stable.
% Additionally their basins of attraction are long and convergence to each unstable orbit is probable.
% However number of system transformations grow very quickly for high dimentional systems and computation becomes too long.
% At the same time parameter $\lambda$ has to be specified correctly to ensure convergence but there is no general way of how to choose this parameter. \cite{Schmelcher1997}

% \subsection{Davidchack-Lai algorithm}
% Davidchack and Lai extended SD algorithm so that it works as a hybrid between Newton-Raphson algorithm and SD algorithm.
% They propose that to find UPOs of period $n$ one can use UPOs of period $n-1$ and $n+1$.
% This technique is supposed to reduce number of iterations significantly compared to SD algorithm. \cite{Davidchack1999}

% \subsection{Our implementation}
% DL algorithm is supposed to be superior to SD algorithm.
% Even though there the DL algoritm is described in \cite{Davidchack1999,Davidchack2001,Klebanoff2001} we were not able to implement it in a way that would fulfill its advertised performance.
% Since the algorithm is over 20 years old and there are no implementations of it online we had to resort to our own version.
% \par
% Our version simply the DL evolution rule but does not use previously UPOs as seeds.
% We choose seeds uniformly from an interval as it was suggested for SD algorithm \cite{Schmelcher1997}.
% We use efficient binary search tree structure which makes the algorithm efficient enough for our use case.
% Example of the results of the algorithm are shown in the Figure \ref{fig:upo_search_example}.

% \begin{figure}[!h]
%     \centering
%     \includegraphics[width=1.0\textwidth]{DDS/Figures/upo_search_example.png}
%     \caption{All unstable periodic orbits}
%     \label{fig:upo_search_example}
% \end{figure}


% \section{Global Search}
% Thanks to algorithm from the previous section we are able to find potential intervals in the bifurcation diagram containing type-I intermittency.
% We start by selecting range of parameters $(p_{A}, p_{B})$ in which we would like to search for intermittency.
% We discretize $(p_{A}, p_{B})$ to $n$ uniformly spaced samples.
% We go through each sample $p$ and check if it is $f_p$ is $m$-periodic.
% We only check periods $m$ until a certain limit $m_{limit}$.
% If we cannot detect any period we simply denote this boundary point as $m_{limit}+1$ which can be understood as infinity.
% Since we know period of each sample we simply compare periods of neighboring samples.
% If we spot a difference between period of sample $a$ and sample $b$, then a bifurcation must have occured in the interval $(a, b)$.
% Additionally if $a=m_{limit}+1$ and $b \leq m_{limit}+1$ then we have found an interval which potentionally contains a breakpoint.
% \par
% Figure \ref{fig:bif_diag_search_example} illustrates global search through bifurcation diagram of logistic map for periods up to 18.
% Two lines of the same color represent an interval in which bifurcation occurs.
% We are looking for breakpoints associated with saddle-node bifurcations.
% These breakpoints are hence also identified by this search.

% \begin{figure}[!h]
%     \centering
%     \includegraphics[width=1.0\textwidth]{DDS/Figures/bif_diag_search_example.png}
%     \caption{Global search example}
%     \label{fig:bif_diag_search_example}
% \end{figure}

% \section{Local Search}
% Once we identify suspicious intervals through global search we can search locally for precise location of a breakpoint.
% Again we utilize the ability to determine number of period of $f_p$ for some $p$.
% Since we know that to the left of the breakpoint there is no periodicity and to the right there is always the same periodicity, we can effectively search for the midpoint.
% We can use similar approach as in the traditional binary search.
% We start with an interval found by global search.
% We know that left boundary has no periodicity and right boundary has periodicity.
% We split the interval and check if the boundaries retain its properties.
% If property at the boundary changes we simply update the initial interval and start the algorithm recursively again.
% This way we are able to precisely locate the breakpoint parameter.
% \par
% An example of the found breakpoint is illustrated in Figure \ref{fig:break_point_search_example}.

% \begin{figure}[!h]
%     \centering
%     \includegraphics[width=1.0\textwidth]{DDS/Figures/break_point_search_example.png}
%     \caption{Local search example}
%     \label{fig:break_point_search_example}
% \end{figure}


\section{Coloring}
The goal of this phase is to color the neighborhood of the breakpoint to warn about its ambiguity.
We have identified the paramters at which the breakpoint occurs in the previous steps.
Now we will use statistical methods to determine the average laminar phase length $l_{avg}$ in proximity of the breakpoint.
These statistical methods are described in \cite{Elaskar2022}.

\subsection{Description of the algorithm}
We will use the fact that the average laminar length for 

\begin{figure}[!h]
    \centering
    \includegraphics[width=0.8\textwidth]{DDS/Figures/logistic_map_coloring_example.png}
    \caption{Bifurcation diagram coloring example}
    \label{fig:coloring_example}
\end{figure}

\section{Complete algorithm}

\endinput