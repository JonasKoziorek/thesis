\chapter{Intermittency Detection}
\label{chapter:intdetection}
Section \ref{sec:ambiguous_bif_diag} of the previous chapter introduced the problem of ambiguity of the bifurcation diagram.
This chapter introduces an algorithm that aims to detect the ambiguity.
The algorithm consists of 3 core parts.

\begin{enumerate}
	\item \textbf{Global Search} -- searching for areas where breakpoints occur.
	\item \textbf{Local Search} -- searching for precise locations of breakpoints.
	\item \textbf{Coloring} -- coloring the bifurcation diagram in the proximity of breakpoints.
\end{enumerate}

Each of these parts are described in the following sections.
The last section of this chapter combines all the parts together.

\section{Global Search}
\label{sec:globsearch}
\textcolor{blue}{
The initial phase of the algorithm, which shall be called the Global Search, aims to find approximate parameter intervals in which type-I intermittency occurs.
The algorithm searches through the whole parameter space globally to find these intervals.
More specifically, the search for type-I intermittency regions is equivalent to a search for the breakpoints in the bifurcation diagram.
Global Search consists of two parts - Naive Global Search and detection of periodic points.
These two parts are described in the following sections and then combined together.
}
\subsection{Naive Global Search}

\textcolor{blue}{
A breakpoint is a parameter value at which the bifurcation diagram transitions from nonperiodic behavior to periodic behavior.
Such breakpoint can be seen in Figure~\ref{fig:break_point_search_example}.
The Figure shows that there is nonperiodic behavior to the left of the breakpoint and periodic behavior to the right of it.
With this knowledge, each breakpoint can be identified by a unique number $n$ which corresponds to $n$ periodic behavior to the right of it.
If the parameter space could be searched through and the period of the system for each parameter value could be determined, the breakpoints could be easily found.
To do that, an algorithm is needed to determine the period of the system for any given parameter value.
}
\par
\textcolor{blue}{
Let $(X, f_{p})$ be a DDS. Suppose a bifurcation diagram of $f_{p}$ is constructed for some $x_0$, $n_{total}$, $n_{last}$ and $p_{range}$ (see Algorithm~\ref{alg:bif_diag}).
The fact that bifurcation diagram shows $n$-periodic behavior ($n$ dots) for a parameter $p$ implies that map $f_{p}$ is $n$-periodic or eventually $n$-periodic.
Two situations can happen:
\begin{enumerate}
    \item Map $f_{p}$ has a stable periodic orbit and the $x_0$ has converged to it in $n_{total}-n_{last}$ iterations.
    \item Initial condition $x_0$ belongs to a UPO or SPO of $f_p$.
\end{enumerate}
When talking about periodic behavior in a bifurcation diagram, the focus will be on case $1$.
Case $2$ will be disregarded since it is not very probable that an initial condition will belong to a periodic orbit.
}
\par
\textcolor{blue}{
Theorem~\ref{theorem:single_orbit} states that the Logistic map $\mathcal{L}_r$ can only have at most one unique stable $n$-periodic orbit for each parameter $r$.
Hence, if bifurcation diagram exhibits $n$-periodic behavior at $r$, it can be concluded that $\mathcal{L}_{r}$ has an $n$-periodic SPO.
Consequently, finding the smallest SPO of $\mathcal{L}_{r}$ would indicate how many dots there are in the bifurcation diagram at $r$.
}
\par
\textcolor{blue}{
There exist algorithms for finding the UPO (and thus SPO) of an arbitrary DDS.
They will be discussed in detail in Subsection~\ref{subsec: detection_of_periodic_points}.
Although they are efficient, they are not efficient enough for the current goal.
The aim is to search through the whole parameter space $p_{range}$ and try to detect an SPO for every parameter $p \in p_{range}$.
}
\par
\textcolor{blue}{
A naive approach is to use \emph{brute-force approach}~\cite{Parker1989} or equivalently find \emph{pseudo periodic orbits}~\cite{Galias2023}.
Initial condition $x_0$ is iterated $m$ times $x_{m} = f^{m}_{p}(x_0)$, where $m = n_{total}-n_{last}$.
Then, it is checked if $|f^{n}_{p}(x_m) - x_m| \leq \delta$, where $\delta$ is some small tolerance.
If $n$ is the smallest number for which the inequality holds, then $x_m$ can be considered an $n$-periodic point.
Consequently, the bifurcation diagram has $n$ dots at $p$.
If the inequality is checked for $n = 1,2,\dots,o$ in an ascending order, then it is clear which $n$ is the smallest.
Number $o$ denotes the maximal period which is checked.
}
\par
\textcolor{blue}{
The procedure that was just described shall be called Naive Global Search (NGS).
NGS is a quick way to estimate periodicity in the bifurcation diagram for any parameter $p$.
The pseudocode for this procedure is presented in Algorithm~\ref{alg:naive_global_search}.
The result of NGS is portrayed graphically in Figure~\ref{fig:naive_global_search}.
There, the points that were identified by NGS as periodic are colored blue.
}
\par
\textcolor{blue}{
After applying NGS, the breakpoint can found easily.
It can be concluded that the breakpoint occurs between parameters $[p_{A}, p_{B}]$ such that NGS found no periodicity for $p_{A}$ and $n$-periodicity for $p_{B}$.
However, the shortcoming of NGS is that it is deceiving in the proximity of a breakpoint.
Imagine that $f^{m}_{p}(x_0)$ for $m = n_{total}-n_{last}$ lies in the laminar phase of an intermittent trajectory of $f_{p}$.
Then $p$ will be determined as periodic even though it is not.
In addition, the bifurcation diagram at $p$ might or might not show $n$ dots.
Since $\mathcal{T}_{m}^{n_{total}}(f_{p}, x_{0})$ is intermittent, the bifurcation diagram could show band of point of just $n$ dots.
That depends on whether the trajectory subset is in the laminar or turbulent phase.
}
\par
\textcolor{blue}{
To overcome this issue, it is useful to verify that period of $p_{A}$ and $p_{B}$ was estimated correctly.
The next subsection introduces efficient algorithms for computing UPO of arbitrary DDS.
By checking how many periodic points are stable for $f_{p_{A}}$ and $f_{p_{B}}$, period of $p_{A}$ and $p_{B}$ can be verified.
}

\begin{algorithm}[!h]
    \caption{Naive Global Search}
    \label{alg:naive_global_search}
    \begin{algorithmic}[1]
        \Statex $f \gets$ map
        \Statex $p_{A} \gets$ left boundary of the parameter space
        \Statex $p_{B} \gets$ right boundary of the parameter space
        \Statex $n \gets$ number of samples in the parameter space
        \Statex $m \gets$ number of iterations
        \Statex $o \gets$ maximum period to check
        \Statex $\delta \gets$ tolerance
        \Statex $x_0 \gets$ initial condition
        \State $p \gets p_{A}$
        \State $p_{step} \gets \frac{p_{B} - p_{A}}{n}$
        \While{$p \leq p_{B}$}
            \State $x_m \gets f_{p}^{m}(x_0)$
            \For{$i \gets 1$ to $o$}
                \State $x \gets f_{p}^{i}(x_m)$
                \If{$|x_m - x| \leq \delta$}
                    \State $f_p$ is $i$-periodic
                \EndIf
            \EndFor
            \State $p \gets p + p_{step}$
        \EndWhile
    \end{algorithmic}
\end{algorithm}

\begin{figure}[!h]
    \centering
    \includegraphics[width=0.95\textwidth]{Figures/naive_global_search.png}
    \caption{
        \textcolor{blue}{
        NGS of $\mathcal{L}_{r}$.
        Grey points: bifurcation diagram constructed of trajectories $\mathcal{T}_{3500}^{4000}(\mathcal{L}_{r}, 0.5)$ for $r \in [3.55, 3.75]$.
        Blue points: points identified as periodic. 
        Parameters of Algorithm~\ref{alg:naive_global_search}: $f = \mathcal{L}_{r}$, $p_{A} = 3.55$, $p_{B} = 3.75$, $n=5000$, $m=1800$, $o=40$ and $x_0 = 0.5$.
        }
    }
    \label{fig:naive_global_search}
\end{figure}

\subsection{Detection of Periodic Points}
\label{subsec: detection_of_periodic_points}

\textcolor{blue}{
Detecting periodic points numerically can be approached in a straightforward manner.
An $n$-periodic point is a fixed point of the $n$-th iterate of a map.
Hence, finding roots of $f^{n}_{p}(x)-x = 0$ yields all periodic points $x$ of $n$-th iteration of map $f_{p}$.
Standard root-finding algorithms such as Newton-Raphson~\cite{Haeseler1988} can be applied on function $g(x) = f^{n}(x)-x$ to find its roots.
Newton-Raphson algorithm converges to a single root. To ensure that all roots were found, the search space can be sampled to $N$ seeds.
Seeds can be created for example as a uniform grid of the search space.
Detecting roots when $n$ is relatively low is successful with proper seeding, however as $n$ increases, the number of roots increases exponentially \cite{Davidchack1999}.
As the number of roots increases, the need for finer and finer seeding arises.
With the rising number of seeds, the algorithm becomes computationally very expensive.
Another issue with Newton-Raphson is that the basins of convergence to each respective root are not very big \cite{Davidchack1999}.
}
\par
\textcolor{blue}{
To overcome the issues with standard root-finding algorithms, researchers came up with new strategies to detect periodic points.
The central idea of the new approach is to use stabilizing transformations.
A set of transformations is proposed such that these transformations stabilize some of the fixed points so that they become attractive.
Afterwards, an iterative scheme is introduced which converges to these fixed points.
Schmelcher and Diakonos~\cite{Schmelcher1997,Pingel2000, Pingel2001} were the first to come up with the idea of stabilizing transformations.
Their algorithm is globally convergent and detects all periodic points of low periods.
The main problem with their approach is that the number of stabilizing transformation grows rapidly with increasing dimension of the dynamical system.
Another issue is that the algorithm relies on fine seeding the same way Newton-Raphson does, although it usually needs much less seeds.
}
\par
\textcolor{blue}{
The algorithm of Schmelcher and Diakonos was subsequently improved by Davidchack and Lai (DL)~\cite{Davidchack1999, Davidchack2001, Klebanoff2001}.
They introduced two improvements: smarter seeding procedure and improved iterative scheme with enhanced convergence speed.
Through their modifications, they were able to detect periodic points of higher periods than it was previously possible.
However, their method is still not very applicable to high dimensional dynamical systems.
}
\par
\textcolor{blue}{
Davidchack's and Lai's algorithm was futher improved by Davidchack's PhD student Crofts~\cite{Crofts2005,Crofts2007,Crofts2008,Crofts20090901}.
His version proposes smaller set of stabilizing transformations such that the method can be used for high dimensional dynamical systems.
}
\par
\textcolor{blue}{
Another approach was proposed by Bu-Wang-Jiang (BWJ)~\cite{Bu2004}.
This approach is not based on stabilizing transformation, but on an iterative scheme together with fine seeding.
}
\par
\textcolor{blue}{
For the purposes of the Global Search, Bu-Wang-Jiang is chosen for detection of periodic points.
It is easy to implement and it works well enough.
Pseudocode for this algorithm is presented in Algorithm~\ref{alg:bwj}.
}

\begin{figure}[!h]
    \centering
    \includegraphics[width=0.95\textwidth]{Figures/upo_search_example.png}
    \caption{
        \textcolor{blue}{
        Detection of the fixed point of $\mathcal{L}_{r}^{6}$ for $r = 3.7$ using the DL algorithm.
        }
    }
    \label{fig:upo_search_example}
\end{figure}

\begin{algorithm}[!h]
    \caption{Bu-Wang-Jiang (BWJ)}
    \label{alg:bwj}
    \begin{algorithmic}[1]
        \Statex $f \gets$ map
        \Statex $p \gets$ period
        \Statex $seeds \gets$ seeds
        \Statex $maxiter \gets$ maximum number of iterations
        \Statex $tol \gets$ tolerance for determining a fixed point
        \For{each s in seeds}
            \State $\textbf{x}_{0} \gets s$
            \While{current iteration $< maxiter$}
                \State $J(\textbf{x}_{0}) = \partial f^{p}(\textbf{x}_{0}) / \partial \textbf{x}$ is the Jacobian at $\textbf{x}_{0}$
                \State $Q \gets (cI-J(\textbf{x}_{0}))(J(\textbf{x}_{0})-I)^{-1}$ where $I$ is identity matrix, $c \in (-1, 1)$ is a constant 
                \State $\textbf{x}_1 \gets f^{p}(\textbf{x}_{0}) + Q(f(\textbf{x}_{0})^{p}-\textbf{x}_{0})$
                \If{$\norm{f^{p}(\textbf{x}_1)-\textbf{x}_1} < tol$}
                    \State $\textbf{x}_{1}$ is a fixed point of $f^{p}$
                \EndIf
                \State $\textbf{x}_{0} \gets \textbf{x}_{1}$
            \EndWhile
        \EndFor
    \end{algorithmic}
\end{algorithm}

\textcolor{blue}{
Note that the $seeds$ parameter in Algorithm~\ref{alg:bwj} is a set of initial conditions from which the algorithm starts.
This parameter can be generated by uniformly sampling the search space.
In case of $\mathcal{L}_{r}$, the interval $[0, 1]$ can be uniformly sampled to $N$ points.
}

\bigskip
\textcolor{blue}{
This section has introduced an algorithm for performing a search for breakpoints in the parameter space.
First, discretization of the parameter space is performed, and the NGS is used to approximate periodicity of each parameter.
Afterwards, neighboring pairs of parameters $p_{A}$ and $p_{B}$ such that $p_{A}$ is nonperiodic and $p_{B}$ is periodic are chosen.
BWJ algorithm is used to verify that periodicity of $p_{A}$ and $p_{B}$ is correct.
If so, it is concluded that there is a breakpoint in a bifurcation diagram somewhere in the interval $[p_{A}, p_{B}]$.
At the end of the algorithm, approximate intervals $[p_{A}, p_{B}]$ and periods of $p_{B}$ are known for each breakpoint.
This algorithm is called the Global Search.
The approximate breakpoint locations evaluated by this algorithm are shown in Figure~\ref{fig:bif_diag_search_example}.
}

\begin{figure}[!h]
    \centering
    \includegraphics[width=0.95\textwidth]{Figures/bif_diag_search_example.png}
    \caption{
        \textcolor{blue}{
        Global Search of $\mathcal{L}_{r}$.
        The bifurcation diagram consists of projections of $\mathcal{T}_{900}^{1000}(\mathcal{L}_{r}, 0.5)$.
        Red lines: detected breakpoints up to period $16$.
        }
    }
    \label{fig:bif_diag_search_example}
\end{figure}

\section{Local Search}

The previous section introduced a so-called Global Search, an approach to find approximate intervals in the parameter space where a breakpoint occurs.
This section introduces algorithms to find the exact location of each breakpoint.
These locations will be later used to color the bifurcation diagram in the proximity of the breakpoint.
\par
This section discusses two algorithms for the Local Search - Naive Local Search and Nested-Layer Particle Swarm Optimization.
The Global Search has already identified two boundaries $[p_A, p_B]$, period of the system at parameter $p_B$ and the fact that there is a breakpoint between $[p_A, p_B]$.
Subsequently, a better approximation of the parameter at which the breakpoint occurs is needed.
The two algorithms are able to approximate the parameter using the information from the Global Search.
Both algorithms solve the problem, but they use different approaches.

\subsection{Naive Local Search}
\label{subsec:naive_local_search}

\textcolor{blue}{
Naive Local Search (NLS) uses the approximate intervals containing a breakpoint that were found during the Global Search and finds precise location of the breakpoint through combination of the binary search and BWJ algorithm.
}
\par
\textcolor{blue}{
Suppose that $[p_{A}, p_{B}]$ is the interval identified during the Global Search.
Furthermore, it is known that $f_{p_{A}}$ is non-periodic and $f_{p_{B}}$ is $n$-periodic.
Next, split the interval $[p_{A}, p_{B}]$ to obtain two intervals, $[p_{A}, \frac{p_{A}+p_{B}}{2}]$ and $[\frac{p_{A}+p_{B}}{2}, p_{B}]$.
If $\frac{p_{A}+p_{B}}{2}$ is $n$-periodic then it is known that the breakpoint is in the interval $[p_{A}, \frac{p_{A}+p_{B}}{2}]$.
If $\frac{p_{A}+p_{B}}{2}$ is non-periodic then it is known that the breakpoint is in the interval $[\frac{p_{A}+p_{B}}{2}, p_{B}]$.
Whether a parameter is $n$-periodic or not is determined using the BWJ algorithm.
The new interval is halved again and the process is repeated recursively.
The halving process is repeated for some desired number of iterations $maxiter$.
Pseudocode for the NLS of the left boundary is given in Algorithm~\ref{alg:local_search}.
}

\begin{algorithm}[!h]
    \caption{NLS - left boundary}
    \label{alg:local_search}
    \begin{algorithmic}[1]
        \Statex $f \gets$ map
        \Statex $p_{A} \gets$ left boundary of the parameter space
        \Statex $p_{B} \gets$ right boundary of the parameter space
        \Statex $n \gets$ period of $f{p_{B}}$
        \Statex $maxiter \gets$ number of iterations
        \State $p_{C} \gets \frac{p_{A}+p_{B}}{2}$
        \State $i \gets$ current iteration
        \If{$i = maxiter$}
            \State left boundary of the breakpoint is $p_{A}$
        \EndIf
        \If{$f_{p_{C}}$ is $n$-periodic}
            \State repeat the algorithm with the same arguments except $p_{B} \gets p_{C}$ and $i \gets i+1$
        \Else
            \State repeat the algorithm with the same arguments except $p_{A} \gets p_{C}$ and $i \gets i+1$
        \EndIf
    \end{algorithmic}
\end{algorithm}

\par
The Algorithm~\ref{alg:local_search} searches only for the left boundary. Search for the right boundary can be done analogically by reversing the direction of the search.
The complete NLS is both the search of the left boundary and the search of the right boundary.
These two boundaries approximate precise location of a breakpoint.
By calculating the average $p_{avg}$ of the left and right boundary, a good approximation of the parameter at which the breakpoint occurs is obtained.
An example of NLS for $\mathcal{L}_{r}$ is shown in Figure~\ref{fig:break_point_search_example}.

\begin{figure}[!h]
    \centering
    \includegraphics[width=0.95\textwidth]{Figures/break_point_search_example.png}
    \caption{
        \textcolor{blue}{
        NLS of $\mathcal{L}_{r}$ for $r \in [ 3.825, 3.83 ]$.
        The bifurcation diagram consists of projections of $\mathcal{T}_{900}^{1000}(\mathcal{L}_{r}, 0.5)$.
        Red lines: Left and right estimates of the period $3$ breakpoint.
        Left estimate is found using Algorithm~\ref{alg:local_search} with parameters $p_A = 3.825$, $p_B = 3.83$, $n = 3$ and $maxiter = 20$.
        Right estimate is found analogically.
        }
    }
    \label{fig:break_point_search_example}
\end{figure}


\subsection{Nested-Layer Particle Swarm Optimization}
The NLS introduced in the previous subsection approximates the parameter at which the breakpoint occurs.
It is based on the idea of a binary search.
However, the search for the breakpoint can be rephrased as a search for a parameter where saddle-node bifurcation occurs.
Matsushita, Kurokawa, and Kousaka~\cite{Matsushita2019} introduced an approach to search for saddle-node bifurcation of a DDS.
Their method uses the Nested-Layer Particle Swarm Optimization (NLPSO) method.
In addition, their approach can be used for detection of other types of bifurcations~\cite{Matsushita20170721}.
\par
Understanding how NLPSO detection of saddle-node bifurcation works requires understanding of the Particle Swarm Optimization (PSO).
PSO is a popular population-based evolutionary algorithm.
It tracks several particles which represent potential solution.
Each particle has its own position and velocity. It also tracks its previous best position and score which is evaluated by a cost function.
Each particle moves through the search space based on its velocity and position.
Its movement is also influenced by its previous best position and the best positions of other particles.~\cite{Matsushita2019}
\par
Each particle has a position $pos \in \mathbb{R}^{n}$, a velocity $vel \in \mathbb{R}^{n}$, a best position $b_{pos} \in \mathbb{R}^{n}$ and a best score $b_{score} \in \mathbb{R}$.
The algorithm also tracks global best position $g_{pos}$ and global best score $g_{score}$.
The pseudocode for the PSO algorithm is given in Algorithm~\ref{alg:pso}.
Parameters $w$, $c_{1}$ and $c_{2}$ in Algorithm~\ref{alg:pso} shall be set as $w=0.729$ and $c_{1}=c_{2}=1.494$.~\cite{Matsushita2019}

\begin{algorithm}[!h]
    \caption{Particle Swarm Optimization (PSO)}
    \label{alg:pso}
    \begin{algorithmic}[1]
        \Statex $f \gets$ function to minimize, $f: \mathbb{R}^{m} \rightarrow \mathbb{R}$.
        \Statex $(a, b) \gets$ search-space range
        \Statex $n \gets$ number of particles
        \Statex $maxiter \gets$ maximum number of iterations
        \Statex $tol \gets$ tolerance for determining solution
        \Statex $w, c_{1}, c_{2} \gets$ parameters described in the text
        \State Create $n$ particles.
        \For{each particle}
            \State $pos$ $m$-dimensional vector of uniform random numbers between $a$ and $b$.
            \State $vel$ $m$-dimensional zero vector.
            \State $b_{pos} \gets$ $m$-dimensional vector of uniform random numbers between $a$ and $b$.
            \State $b_{score} \gets \infty$
        \EndFor
        \State $g_{pos}$ $m$-dimensional vector of uniform random numbers between $a$ and $b$.
        \State $g_{score} \gets \infty$

        \For{iteration less than $maxiter$}
            \For{each particle}
                \State $score \gets f(pos)$ 
                \If{$score < b_{score}$}
                    \State $b_{score} \gets score$
                    \State $b_{pos} \gets pos$
                \EndIf
                \If{$score < g_{score}$}
                    \State $g_{score} \gets score$
                    \State $g_{pos} \gets pos$
                \EndIf
            \EndFor
            \If{$g_{score} < tol$}
                \State break the loop
            \EndIf
            \For{each particle}
                \State $r_{1}, r_{2} \gets$ random numbers between $0$ and $1$.
                \State $vel \gets w(vel) + c_{1}r_{1}(b_{pos}-pos) + c_{2}r_{2}(g_{pos}-pos)$
                \State $pos \gets pos + vel$
            \EndFor
        \EndFor
    \end{algorithmic}
\end{algorithm}

\par
The next step is to combine two PSOs.
One of them will be looking for a parameter $p_b$ at which saddle-node bifurcation occurs.
The other one will be looking for periodic point $x_b$ corresponding to $p_b$.
Its worth noticing that the algorithm works for dynamical systems of arbitrary dimensions.
For that reason the minimization functions are presented in a general form.
\par
\textcolor{blue}{
Let $(\mathbb{R}^{m}, f_{p})$ be a DDS.
The first PSO, denoted as PSO1, is searching for a periodic point $x_0$ given some parameter $p_{1}$.
The PSO1 does this by minimizing the function $F_{loss}(x_0) = \norm{f^{n}_{p_{1}}(x_0)-x_0}$.
Furthermore, another PSO, denoted as PSO2, is searching for a parameter $p_{b}$ which corresponds to a saddle-node bifurcation.
It achieves that by minimizing the function $G_{loss}$~\eqref{eq:minimize_p}.
Note that $G_{loss}$ has a minimum of $0$ in case that $p = p_{b}$ and $x_{0}$ is a periodic point corresponding to $p_{b}$.
Tolerance $tol$ in~\eqref{eq:minimize_p} specifies whether $F_{loss}$ is close enough to the solution.
}

\begin{equation}
\label{eq:minimize_p}
    G_{loss}(p) =
    \begin{cases}
        |\text{det}(Df^{n}_{p}(x_0)-I)| & \text{if } F_{loss}(x_0) < tol, \\
        \infty & \rm{otherwise}
    \end{cases}
\end{equation}

\textcolor{blue}{
PSO2 has to evaluate its loss function $G_{loss}(p)$ during its runtime.
However, in order to evaluate $G_{loss}(p)$, it needs to know a periodic point $x_0$ of $f_{p}$.
To do that, it uses PSO1 and sets its parameter $p_{1} = p$.
Consequently, PSO1 converges to a periodic point of $f_{p}$, if there is any.
This nesting of two PSOs enables the PSO2 to converge to a saddle-node bifurcation parameter $p_{b}$.
}
\par
\textcolor{blue}{
To initiate PSO2, information from the Global Search can be used.
Suppose that $[p_{A}, p_{B}]$ is the interval identified during the Global Search of a DDS $(X, f_{p})$.
Furthermore, it is known that $f_{p_{A}}$ is non-periodic and $f_{p_{B}}$ is $m$-periodic.
Then, the parameters of PSO2 (Algorithm~\ref{alg:pso}) can be chosen as follows: $f = f_{p}$, $(a, b) = (p_{A}, p_{B})$, $n = m$, $maxiter = 500$ and $tol = 10^{-6}$.
The parameters of PSO1 can be chosen equivalently, except $(a, b)$ has to be chosen with respect to $f_{p}$.
Since $X$ is invariant under $f_{p}$, $(a, b)$ can be chosen appropriately.
For example for the Logistic map, $(a, b)$ of PSO1 can be set to $[0, 1]$.
}

\section{Coloring}
The goal of this phase is to color the neighborhood of the breakpoint to warn about its ambiguity.
Parameters at which the breakpoint occurs were found in the previous steps.
Characteristic relation of type-I intermittency is used to color the neighborhood of the breakpoint.

\subsection{Description of the algorithm}
The Local Search found parameters of the parameter space at which the breakpoint occurs.
Let $p_b$ be one of them.
It is known that parameters to the left side $p_b$ in its proximity exhibit intermittency.
Nevertheless, there is a need to quantify how far from $p_b$ is intermittency still occurring.
Additionally, it is needed to measure how prominent the intermittency is for some $p$ near $p_b$.
\par
A useful indicator about the intermittent behavior is average laminar length $l_{avg}$ described in Chapter~\ref{chap:type-I intermittency}.
The characteristic relation $l_{avg} \varpropto 1 / \sqrt{\varepsilon}$ can be used to estimate the average laminar length.
Characteristic relation is dependent on the local map around fixed points associated with saddle-node bifurcation.
Fortunately, these fixed points have been found in the Local Search.
Hence, by calculating $\varepsilon$ for each stable fixed point of $f_{p_b}$, $l_{avg}$ can be estimated for laminar region around each fixed point.
Taking their average yields an estimation of the average laminar length for $p_b$.
\par
Depending on the parameter $p$ the laminar phases have various lengths.
Far to the left of $p_b$ the laminar phases are about $50$ iterations long.
Very close to the left of $p_b$ the length of the laminar phase can get arbitrary long.
The closer the parameter $p$ gets to $p_b$ the longer the laminar phase gets.
\par
To colorize the diagram it is needed to determine the average laminar length for the parameters $p$ to the left of $p_b$ and to compare them.
However, the average laminar lengths can get arbitrary large.
It is convenient to color only parameters $p$ that have average laminar lengths between $L$ and $U$. 
$L$ denotes the lower bound and $U$ denotes the upper bound.
For example $L$ can be set to $100$ and $U$ to $1000$.
The next step is to find parameters $p_L$ and $p_U$ such that parameters $p$ between $p_L$ and $p_U$ have $l_{avg}$ from range $[ L, U ]$.
\par
The idea behind an algorithm to find $p_L$ and $p_U$ is to step to the left from $p_b$ with incrementally smaller steps.
The algorithm is illustrated in Algorithm~\ref{alg:optimal_bound}.

\begin{algorithm}[!h]
    \caption{Optimal bound search}
    \label{alg:optimal_bound}
    \begin{algorithmic}[1]
        \Statex $f \gets$ map
        \Statex $p \gets$ initial bound estimate
        \Statex $s \gets$ step
        \Statex $B \gets$ desired average laminar length
        \Statex $t \gets$ $B$ deviation tolerance
        \Statex $i_{m} \gets$ maximum of iterations
        \Statex $i_{c} \gets$ current iteration

        \For{$i$ from $i_{c}$ to $i_{m}$}
            \State $p \gets p - s$
            \State $l_{avg} \gets$ average laminar length for $p$
            \If{$B-t \leq l_{avg} \leq B+t$}
                \State Terminate program with $p$ as the optimal bound.
            \EndIf
            \If{$l_{avg} < B-t$}
                \State Repeat the program for $p = p+s$, $s = s/2$ and $i_{c} = i+1$. The rest of the parameters remain unchanged.
            \EndIf
        \EndFor
    \end{algorithmic}
\end{algorithm}

The Algorithm~\ref{alg:optimal_bound} can be used for finding both $p_L$ and $p_U$.
The algorithm can be started with $p = p_b$, $s = 0.1$, $i_{m} = 200$, $i_{c} = 0$, $t = 10$.
For finding $p_L$, initial bound estimate $B = 100$.
For finding $p_U$, initial bound estimate $B = 1000$.
This way both $p_L$ and $p_U$ is obtained.
\par
To finish the coloring procedure, bifurcation diagram is constructed.
A color range is defined for numbers in range $[ l_{avg}$ for $p_L, l_{avg}$ for $p_U ]$.
The trajectory projections for parameters $p \in [ p_L, p_U ]$ are colored using this color range.
A trajectory projection at parameter $p$ is colored based on $l_{avg}$ corresponding to $p$.
The same coloring procedure is repeated for all breakpoint parameters found in the Local Search.
\par
The result of the coloring algorithm for a single breakpoint of the Logistic map is shown in Figure~\ref{fig:coloring_example}.
Note that in order to see the full range of colors, the bifurcation diagram has to be zoomed-in to the breakpoint.

\begin{figure}[!h]
    \centering
    \includegraphics[width=0.95\textwidth]{Figures/logistic_map_coloring_example.png}
    \caption{
        \textcolor{blue}{
        Colorization of a single breakpoint of $\mathcal{L}_{r}$ for $r \in I := [ 3.6263, 3.6267 ]$.
        Algorithm~\ref{alg:optimal_bound} was used to specify the colorization bounds.
        Lower bound $L$: $200$.
        Upper bound $U$: $1000$.
        The bifurcation diagram consists of projections $\mathcal{T}_{900}^{1000}(\mathcal{L}_{r}, 0.5)$ for $r \in I$.
        }
    }
    \label{fig:coloring_example}
\end{figure}

\section{Complete Algorithm}
In the previous sections each part of the algorithm was described.
It was also pointed out how each part of the algorithm relates to the other parts.
For clarity, this section explains how to combine all the parts together.
\par
First, the Global Search is used to identify intervals where breakpoint could occur.
Secondly, the Local Search is employed for each of the intervals to find precise parameter, periodicity and the fixed points of the breakpoint.
Two alternatives for the Local Search were introduced.
Each of them can be used.
Lastly, areas next to the found breakpoint are colored using the Coloring algorithm.
The result is a bifurcation diagram with identified breakpoints whose left neighborhoods are colored.
\par
An example result of the complete algorithm is shown in Figure~\ref{fig:complete_colorization}.
Note that periodic windows (or equivalently breakpoints) that weren't visible at first are detected through this algorithm.

\begin{figure}[!h]
    \centering
    \includegraphics[width=0.95\textwidth]{Figures/complete_colorization.png}
    \caption{
        \textcolor{blue}{
        Full Colorization of $\mathcal{L}_{r}$ for $r \in [ 3.62, 3.65 ]$.
        The bifurcation diagram consists of projections of $\mathcal{T}_{900}^{1000}(\mathcal{L}_{r}, 0.5)$.
        Colorful lines: areas near the detected breakpoints (up to period $20$).
        }
    }
    \label{fig:complete_colorization}
\end{figure}

\endinput