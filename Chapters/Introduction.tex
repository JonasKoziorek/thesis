\chapter{Introduction}
\label{sec:Introduction}

The field of dynamical systems is a modern branch of mathematics that studies complex systems.
Its main goal is to understand the long term behavior of systems that evolve in time.
Real-world models from physics, biology, engineering as well as purely theoretical models are being studies.
Subfields such as chaos theory and fractals are part of this field that caught a lot of attention from the general public.
The buzzword "butterfly effect" coined by Edward Lorenz, famous for the Lorenz attractor~\cite{Lorenz2004}, has been used in many movies and books.
\par
Dynamical systems are difficult to study analytically but also numerically.

\textcolor{red}{
% The mathematical theory of dynamical systems develops dynamical notions and their relations.
% On the other hand, there is a desire to quantify developed qualifications, which is done numerically by numerous procedures.
The mathematical theory of dynamical systems studies dynamic concepts and their interconnections.
Conversely, there is a desire to quantify these conceptual advancements, achieved through various numerical procedures.
}


Many algorithms are being developed to detect and measure various dynamical properties and to control and predict the behavior of dynamical systems.
The initial aim was to study the phenomena called intermittency.
Intermittency is a characteristics of a system during which system alternates between seemingly predictable and chaotic behavior.
Even though there is many types of intermittency, this text is focused on the type-I intermittency.
It turns out that that type-I intermittency leads to ambiguity of a bifurcation diagram, a tool commonly used for global analysis of a system.
Hence enhancing bifurcation diagram with warning about underlying type-I intermittency could be a useful tool for scientists and engineers.

\bigskip

The thesis is organized as follows:



\endinput