\chapter{Introduction}
\label{sec:Introduction}

Real-world systems such as climate, economics, population dynamics or celestial mechanics are complex.
In fact, they are so complex that, before the advent of computers, some of these systems were immune to thorough scientific exploration.
Fortunately, as technology had advanced, the scientific community was able to tackle increasingly complicated systems and problems.
At the heart of the study of complex systems is the theory of dynamical systems \cite{Devaney20211026, Hirsch2013, Strogatz201854}.
\par
Dynamical systems theory serves as a foundational framework for understanding the behavior of systems that evolve over time, encompassing a diverse array of phenomena across various scientific disciplines.
From celestial mechanics~\cite{Holmes1990} to population dynamics~\cite{Hastings1993,Hadeler2001}, from climate modeling~\cite{Ghil2023} to neural networks~\cite{Cessac2009, Vogt2020, Li2019}, disease spread~\cite{Ritelli20210930} to logistics~\cite{Kumara2003}, dynamical systems theory provides a unified language to describe the evolution of complex systems.
\par
At its core, a dynamical system is characterized by a set of rules that govern how its state changes over time.
These rules can be deterministic or stochastic, linear or nonlinear, discrete or continuous.
The research of dynamical systems involves analyzing the long-term behavior of these systems, seeking to understand their stability, periodicity, chaos, and other emergent properties.
Many algorithms are being developed to detect, control, and predict system's dynamical properties.
\par
\textcolor{blue}{
Arguably the most interesting dynamical property is chaos.
This phenomenon is characterized by a strong sensitivity of a system to initial conditions.
Small perturbations of the initial conditions can lead to drastically different long-term behavior.
Despite the apparent randomness, chaotic systems posses underlying patterns and structures.
Interestingly enough, chaos can emerge in very simple systems~\cite{Lorenz2004,May19760610}.
It is not a surprise that complex such as nature also exhibit chaos~\cite{Toker2020}.
}
\par
Although, chaos has been the focus of much attention, there are many other properties that are worth investigating.
A dynamical property that is central to this thesis is called intermittency.
Intermittency is a fascinating phenomenon of dynamical systems, where seemingly stable behavior can abruptly transition to chaotic behavior, only to return to stability just as unexpectedly.
The mathematical study of intermittency originated in the 1980s through observation of the intermittent transition to turbulence in convective fluids \cite{Pomeau1980}.
Since then, the phenomenon has garnered significant attention across diverse scientific disciplines due to its relevance in understanding phenomena such as gas flow dynamics~\cite{Pizza20110926}, brain dynamics~\cite{Paradisi2013}, and economic fluctuations~\cite{Chian2007}, among others.
\par
Understanding intermittency poses theoretical and practical challenges.
Theoretical investigations delve into the mathematical foundations of intermittency, seeking to unravel the underlying mechanisms that govern its occurrence~\cite{Elaskar2017, Elaskar2023}.
Practical applications, on the other hand, often involve the identification, detection, and characterization of intermittent behavior in real-world systems, paving the way for predictive modeling and control strategies.
\par
Detection of intermittent behavior is the main objective of this thesis, more specifically detection of type-I intermittency~\cite{Pomeau1980,Bussac1982,DelRio2014}.
This type is one of the most well-known and studied types that have been discovered.
% It has been observed in several real-world systems \cite{Zebrowski2004,Zebrowski2005,Parthimos2001,Aikawa1990,Shiau1995,Storchi2010,Dimitriu2008,Chiriac20070701}.
It has been observed in several real-world systems \cite{Zebrowski2004,Parthimos2001,Dimitriu2008,Chiriac20070701}.
Developing better algorithmic tools to deal with this dynamical property is an important step towards better understanding and managing complex systems.
With improved algorithms, researchers and practitioners can more effectively detect, analyze, and predict intermittent behavior, allowing a deeper understanding of the underlying dynamics and facilitating more informed decision-making in various domains.

\bigskip

This thesis aims to construct an algorithmic approach for the detection of type-I intermittency in discrete dynamical systems.
However, before such an approach can be developed, several notions from the standard theory have to be introduced.
\par
The thesis is organized as follows:

\begin{description}
	\item \textbf{Chapter 1} -- provides motivation behind the thesis.
	\item \textbf{Chapter 2} -- introduces the theory of discrete dynamical systems.
	\item \textbf{Chapter 3} -- introduces type-I intermittency.
	\item \textbf{Chapter 4} -- develops a method for the detection of type-I intermittency.
	\item \textbf{Chapter 5} -- discusses the software implementation.
	\item \textbf{Chapter 6} -- concludes the findings of this thesis.
\end{description}

\endinput