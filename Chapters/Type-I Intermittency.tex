\chapter{Type-I Intermittency}
\label{chap:type-I intermittency}

\textcolor{darkred}{
This Chapter is concerned with the phenomena of intermittency, especially with type-I intermittency.
Firstly discovery and history of intermittency is mentioned.
Then intermittency is motivated mathematically.
Furthermore, the idea of laminar length and charasteristic relation is introduced.
Lastly, the ambiguity of bifurcation diagram caused by intermittency is shown.
}

\section{History}
\textcolor{darkred}{
In 1949, the word intermittency was used in context of turbulent flows by Batchelor and Townsend~\cite{Batchelor19491025}.
They used the term to depict signals that altered between approximately flat periods and and burst ones~\cite{Elaskar2017}.
}
\par
\textcolor{darkred}{
About three decades later, Pomeau and Manneville have studied intermittent transition to turbulence in an experiment with convective fluids~\cite{Pomeau1980}.
During this phenomenon, long periodic behavior was interrupted by chaotic bursts as the experiment control parameter changed.
They have shown that intermittent behavior is present in simple models such as the Lorenz model~\cite{Lorenz2004}.
Furthermore, they classified intermittency into three types based on model's Floquet multipliers.
They called these three types as I, II and III.
}
\par
\textcolor{darkred}{
Theoretical analysis of intermittency is centered around certain statistical properties, especially the reinjection probability density function.
While the standard theory considered this function to be constant~\cite{Dubois1983}, new advances in the field provide more complex explanation~\cite{Elaskar2022}.
In addition, to the three original types of intermittency, many new types have been described since then.
Some of them are V, X, on-off, eyelet and ring~\cite{Elaskar2022}.
}

\section{Motivation}
Intermittency is a behavior during which seemingly periodic phases in the trajectory of DDS are abruptly followed by chaotic bursts.
Periodic phases are called laminar.
Phases with chaotic bursts are called turbulent phases.
This naming convention originated in fluid dynamics which motivated study of intermittency.~\cite{Pomeau1980}
\par
\textcolor{darkred}{
To illustrate the concept of intermittency clearly, a new dynamical system is needed.
Let $(\mathbb{R}, \mathcal{P}_{\varepsilon})$ be a DDS.
Map $\mathcal{P}_{\varepsilon}(x) = \left[ (1+\varepsilon)x+(1-\varepsilon)x^2 \right] \mod{1}$ is called the Pomeau-Manneville map~\cite{Manneville1980,Datseris2022}.
With this map, intermittency is clearly visible in its trajectory.
This trajectory is diplayed in Figure~\ref{fig:intermittent_trajectory_example}.
Seemingly 2-periodic behavior which lasts for about 1000 iterations is interrupted by short chaotic bursts.
These chaotic bursts transform into 2-periodic behavior.
This pattern continues.
}

\begin{figure}[!h]
    \centering
    \includegraphics[width=0.95\textwidth]{Figures/type_one_intermittency_example1.png}
    \caption{
        \textcolor{darkred}{
        Trajectory $T^{5500}_{0}(\mathcal{P}_{\varepsilon}, x_0)$ for $x_0 = 0.5$ and $\varepsilon = 4.47458$.
        }
    }
    \label{fig:intermittent_trajectory_example}
\end{figure}

The trajectory in Figure~\ref{fig:intermittent_trajectory_example} exhibits type-I intermittency which is related to \hyperref[def:saddle_node_bif]{saddle-node bifurcation} or its inverse.
Saddle-node bifurcation is a bifurcation in which the tangent of a graph of a map passes through the identity line as parameter changes.
\par
Let parameter $p_{B}$ be a bifurcation point.
At this point, local map of the system touches the identity line in exactly one spot; in other words, the system has fixed point in the area of touching.
For parameters $p < p_{B}$ there are two fixed points, and the curve is intersected in two places.
For parameters $p > p_{B}$ there is no fixed point, since the curve left the identity line.
This relation is illustrated in Figure~\ref{fig:saddle_node_bifurcation}.
The red dot in Figure~\ref{fig:saddle_node_bifurcation} is a fixed point for the curve at the bifurcation point $p_{B}$.
The blue dots in Figure~\ref{fig:saddle_node_bifurcation} are fixed points for the curve for the parameter $p < p_{B}$.
When $p > p_{B}$ is greater but very close to $p_{B}$ a channel forms.
\begin{figure}[!h]
    \centering
    \includegraphics[width=0.7\textwidth]{Figures/intermittency_i_local_map.png}
    \caption{
        \textcolor{darkred}{
        Map $f_{p}(x) = p + x + x^2$ for different parameters $p$. 
        Parameter $p_B = 0$.
        }
    }
    \label{fig:saddle_node_bifurcation}
\end{figure}

\textcolor{darkred}{
The reason why intermittency arises is illustrated in Figure~\ref{fig:intermittent_cobweb_example}.
When the dynamical system $f_p$ is about to undergo saddle-node or inverse saddle-node bifurcation as parameter $p$ changes, the graph of $f_p$ is getting closer to the identity function $g(x)=x$.
Just before they touch, a small passage is formed between the two curves.
If the point $x_n$ gets injected into the narrow passage, it takes many iterations for it to get out of this passage again.
The closer the curve of $f_p$ and the identity function is, the narrower the passage and the longer it gets to iterate through it.
An illustration of the iteration through narrow passage is in Figure~\ref{fig:intermittent_cobweb_example}~(b).
This iteration corresponds to the first laminar phase in Figure~\ref{fig:intermittent_trajectory_example}.
While the point is traveling through a narrow passage (Figure~\ref{fig:intermittent_cobweb_example}~(a),~(b)), the trajectory (Figure~\ref{fig:intermittent_cobweb_example}~(c)) looks stable.
However the point is just trapped in the narrow passage for some finite number of iterations and hence looks periodic.
Nevertheless, this seemingly periodic behavior is not real, since once the point gets out of the narrow passage, it moves chaotically once again.
After some time, it gets reinjected back into the narrow passage and the process repeats.
}

\begin{figure}[!h]
    \centering
    \includegraphics[width=0.95\textwidth]{Figures/type_one_intermittency_example2.png}
    \caption{
        \textcolor{darkred}{
        Subfigures (a) and (b):
        Cobweb diagram of 
        (a) $T^{472}_{38}(\mathcal{P}_{\varepsilon}^{2}, x_0)$ and
        (b) $T^{461}_{40}(\mathcal{P}_{\varepsilon}^{2}, x_0)$. 
        Subfigure (c):
        Trajectory $T^{520}_{0}(\mathcal{P}_{\varepsilon}^{2}, x_0)$.
        Parameters: $x_0 = 0.5$ and $\varepsilon = 4.47458$.
        }
    }
    \label{fig:intermittent_cobweb_example}
\end{figure}

\section{Laminar Phase Length}

For one dimensional maps, type-I intermittency is generated by a local map and a reinjection mechanism.
The local map for type-I intermittency is formulated in a following way:
\begin{equation}
\varepsilon + x + a x^2 \label{eq:int_I_local_map}
\end{equation}
This local map is the same for any DDS exhibiting type-I intermittency.
The reinjection mechanism depends on the specific DDS.
This mechanism dictates how the point is reinjected from the turbulent region to the laminar region.
Laminar region is the narrow passage described earlier, in which the trajectory is laminar.
The turbulent region is the region outside the laminar region.
% The probability that a point gets mapped from the chaotic region back into the laminar region is described by the reinjection probability density function (RPD function).~\cite{Elaskar2022}
\par
In the next chapter, the notion of laminar length and average laminar length will be needed.
The laminar length is the number of iterations a point spends in the laminar region.
For a mathematical description of laminar length, few other notions need to be explained first.
\par
The local map \eqref{eq:int_I_local_map} is related to the saddle-node bifurcation.
During saddle-node bifurcation, two fixed points combine into one fixed point, which shall be denoted $x^{*}$.
Let $(X, f)$ be a DDS. Suppose that $f$ undergoes saddle-node bifurcation and $x^{*}$ is the mentioned fixed point.
The pre-reinjection point $x_p$ is a point such that there exists a point $x_r = f(x_p)$ and $x_r \in I_l$ is in the laminar interval $I_l = [ x^{*}-c, x^{*}+c ]$.
The point $x_r$ shall be called a reinjected point.
The width of the laminar interval $c$ is a relatively small number.
In this case, the laminar region is considered to be an interval around $x^{*}$.
\par
The Figure~\ref{fig:pre_reinjection_example} illustrates the idea of pre-reinjection points and laminar interval $I_l$.
Note that the curve in subfigure (b) locally resembles the local map \eqref{eq:int_I_local_map} around the point $x^{*}$.
\begin{figure}[!h]
    \centering
    \begin{subfigure}{0.49\textwidth}
        \centering
        \includegraphics[width=\textwidth]{Figures/pre_reinjection_points_example{1}.png}
        \caption{}
    \end{subfigure}
    \hfill
    \begin{subfigure}{0.49\textwidth}
        \centering
        \includegraphics[width=\textwidth]{Figures/pre_reinjection_points_example{2}.png}
        \caption{}
    \end{subfigure}

    \caption{
        \textcolor{darkred}{ 
        Pre-reinjection points and laminar interval of $\mathcal{L}_{r}^{3}(x+x^{*})-x^{*}$ for $r = 3.827$ and $x^{*} \approxeq 0.51435$.
        (a) the full graph. 
        (b) a close-up to a region of interest. 
        Red dots: pre-reinjection points.
        Blue dots: reinjected points.
        Laminar interval: $I_l = [ x^{*}-c, x^{*}+c ]$ for $c = 0.03$.
        Orange line: $I_l$.
        Green dot: $x^{*}$.
        }
    }
    \label{fig:pre_reinjection_example}
\end{figure}
\par
Consider that $0 < \varepsilon \ll 1$.
By using the Equation~\eqref{eq:int_I_local_map} the difference between two successive points $x_n - x_{n-1} \approx dx/dl$ can be written as:
\begin{equation}
\frac{dx}{dl} = \varepsilon + a x^2 \label{eq:int_I_diff_eq}
\end{equation}
Thus continuous differential equation is obtained.
Solving from reinjected point $x$ to the end of laminar interval $c$ yields:
\begin{equation}
\int_{x}^{c} \frac{dx}{\varepsilon + a x^2} = \int_{0}^{l} dl \label{eq:int_I_diff_eq_step1}
\end{equation}
Solving for $l$ results in:
\begin{equation}
    l(x, c) = \frac{1}{\sqrt{a \varepsilon}} \left( \tan^{-1} \left( c \sqrt{\frac{a}{\varepsilon}} \right) - \tan^{-1} \left( x \sqrt{\frac{a}{\varepsilon}} \right) \right) \label{eq:laminar_length}
\end{equation}

\par
Hence for a given reinjected point $x$ and laminar interval width $c$ the laminar length $l$ can be calculated.
By calculating laminar length for many reinjected points and some fixed $c$ the average laminar length $l_{avg}$ can be calculated.
However, that is not very practical since it is not always clear how to choose $c$.
Fortunately, this problem was solved by formulation of so-called characteristic relation of type-I intermittency.

\section{Characteristic Relation}
To estimate the average laminar length $l_{avg}$ based on the parameter $\varepsilon$ a characteristic relation was formulated~\cite{Elaskar2017}.
It holds that $l_{avg} \propto 1/\sqrt{\varepsilon}$.
This relation states that the average laminar $l_{avg}$ length is inversely proportional to the square root of the parameter $\varepsilon$.
Thus, by calculating $\varepsilon$ of the local map in the vicinity of a fixed point $x^{*}$ the $l_{avg}$ can be estimated.
\par
To verify that the characteristic relation of type-I intermittency holds, a numerical simulation was conducted.
The relation is verified for the third iterate of the Logistic map $\mathcal{L}_{r}^{3}$.
\par
\textcolor{darkred}{
The third iterate of Logistic map $\mathcal{L}_{r}^{3}$ undergoes saddle-node bifurcation at $r = r^{*} := 1+\sqrt{8}$~\cite{Elaskar2022,Gordon20180411}.
Parameter $r^{*}$ is the start of the $3$-periodic window, the biggest periodic window visible in the bifurcation diagram of $\mathcal{L}_{r}$ (see Figure~\ref{fig:bif_diag_example}).
There are three fixed points of $\mathcal{L}_{r^{*}}$ associated with the bifurcation, $x^{*}_{1} \approx 0.1599$, $x^{*}_{2} \approx 0.5143$ and $x^{*}_{3} \approx 0.9563$.
For parameter $r$ slightly less than $r^{*}$, $\mathcal{L}_{r}^{3}$ in the proximity of each of the points $x^{*}_{1}$, $x^{*}_{2}$ and $x^{*}_{3}$ resembles the local map \eqref{eq:int_I_local_map}.
Figure~\ref{fig:pre_reinjection_example}~(b) shows the local map of $\mathcal{L}_{r}^{3}$ in the proximity of $x^{*}_{2}$.
The following explanation of the simulation will use $x^{*}_{1}$ as an example.
The simulation can be conducted in the same way for $x^{*}_{2}$ and $x^{*}_{3}$.
}
\par
The goal is to compute the average laminar length $l_{avg}$ for a point traveling through the laminar region around $x^{*}_{1}$.
The laminar region exists for the parameters $r$ slightly less than $r^{*}$.
In the experiment $30$ parameters $r$ were randomly chosen from the interval $[ r^{*}-10^{-5}, r^{*}-10^{-13} ]$.
Subsequently, the corresponding $\varepsilon$ parameters were calculated.
\par
To calculate $\varepsilon$ for the local map around $x^{*}_{1}$, the map $\mathcal{L}_{r}^{3}$ was shifted so that $x^{*}_{1}$ is in the origin.
In other words, $\mathcal{L}_{shift}(x) = \mathcal{L}_{r}^{3}(x + x^{*}_{1}) - x^{*}_{1}$.
Parameters $\varepsilon$ and $a$ are calculated using the Taylor polynomial of $\mathcal{L}_{shift}(x)$ at $x = 0$.
Hence, $\varepsilon = | \mathcal{L}_{shift}(0) |$ and $a = \mathcal{L}_{shift}''(0) / 2$.
$\varepsilon$ and $a$ are calculated for each of the $30$ random $r$ parameters.
Next, they are used to calculate the laminar length.
\par
To calculate the laminar length $l$ from \eqref{eq:laminar_length} corresponding to one $r$, reinjection point $x$, laminar interval width $c$ and parameters $\varepsilon$ and $a$ are needed.
The laminar interval width $c$ in the experiment was selected as $0.01, 0.03, 0.05$ and $0.07$.
In the following experiment, $c = 0.01$.
The same procedure can be done for the remaining $c$'s.
$c$, $\varepsilon$ and $a$ are known.
Next reinjection points $x$ are calculated.
\par
To calculate the reinjection points corresponding to one of the $30$ $r$'s, the following procedure is used.
A point $x$ is randomly chosen from $[ 0-x^{*}_{1}, 1-x^{*}_{1} ]$.
Consequently, a check is conducted if the point gets reinjected into laminar interval.
In other words, check if $| \mathcal{L}_{shift}(x) | < c$.
The procedure is repeated until $N = 4000$ reinjected points are obtained.
\par
For each reinjected point $x$ and the laminar length $l$ is calculated using \eqref{eq:laminar_length}.
Then, their average is calculated.
\par
The procedure is repeated for each of the $30$ $r$'s.
The results are plotted in log-log scale.
A linear fit of the results is calculated and the results are shown in Figure~\ref{fig:characteristic_relation_fit}.
Note that the results support that the $l_{avg} \propto 1/\sqrt(\varepsilon) = \varepsilon ^ {-1/2}$.
The linear fit for different $c$'s and $x^{*}$ is shown.
\par
Also note that it holds $l_{avg} \propto \varepsilon ^ {-1/2} \implies log_{10}(l_{avg}) \propto -1/2 \cdot \log_{10}(\varepsilon)$.
The Figure~\ref{fig:characteristic_relation_fit} shows that the slope of the linear fits is approximately $-1/2$.
That implies that the characteristic relation of type-I intermittency holds. 

\begin{figure}[!h]
    \centering
    \begin{subfigure}{0.49\textwidth}
        \centering
        \includegraphics[width=\textwidth]{Figures/characteristic_relation_fit{1}.png}
        \caption{}
    \end{subfigure}
    \hfill
    \begin{subfigure}{0.49\textwidth}
        \centering
        \includegraphics[width=\textwidth]{Figures/characteristic_relation_fit{2}.png}
        \caption{}
    \end{subfigure}
    \hfill
    \begin{subfigure}{0.49\textwidth}
        \centering
        \includegraphics[width=\textwidth]{Figures/characteristic_relation_fit{3}.png}
        \caption{}
    \end{subfigure}

    \caption{
        \textcolor{darkred}{ 
        $x$-axis: $30$ random $r \in [ r^{*}-10^{-5}, r^{*}-10^{-13} ], r^{*} = 1 + \sqrt{8}$ selected, corresponding $\varepsilon$ are calculated.
        $y$-axis: average $l_{avg}$ of laminar lengths (Equation~\eqref{eq:laminar_length}) for $4000$ points reinjected into laminar interval $I_{c} = [x^{*}-c, x^{*}+c], c \in \{ 0.01, 0.03, 0.05, 0.07 \}$ is calculated.
        Colorful lines: linear fit.
        Constant $s$: slope of the linear fit.
        Fixed point $x^{*}$: (a) $x^{*} = 0.1599$ (b) $x^{*} = 0.5143$ and (c) $x^{*} = 0.9563$.
        DDS: $\mathcal{L}_{r}^{3}$.
        }
    }
    \label{fig:characteristic_relation_fit}
\end{figure}

\section{Ambiguous Bifurcation Diagram}
\label{sec:ambiguous_bif_diag}
How does the bifurcation diagram look for an intermittent region?
During the creation of a bifurcation diagram, some fixed number of iterations are plotted.
Based on the number of total iterations and number of last iterations being plotted, the bifurcation diagram may differ.
When plotting a subsection of the trajectory that corresponds only to the laminar phase, the behavior during the turbulent phase is omited.
This causes the bifurcation diagram to look different depending on how many iterations are made and how long the sample period is.
\par
\begin{figure}[!h]
    \centering
    \includegraphics[width=0.95\textwidth]{Figures/bif_diag_ambiguity_creation_sketch.png}
    \caption{
        \textcolor{darkred}{
        Sketch describing how to create an ambiguous bifurcation diagram. 
        Red graph: $T^{17000}_{0}(\mathcal{P}_{4.47458285}, x_0)$. 
        Blue graph: $T^{17000}_{0}(\mathcal{P}_{4.47458285+4 \cdot 10^{-8}}, x_0)$. 
        Initial condition: $x_0 = 0.5$.
        }
    }
    \label{fig:ambiguous_bif_diag}
\end{figure}

To illustrate this ambiguity in bifurcation diagram, one can simply create a generic example.
Figure~\ref{fig:ambiguous_bif_diag} shows two trajectories for two distinct parameters $p_1$ and $p_2$.
Both are close to the breakpoint $p_{b}$ and it holds $p_1 < p_2 < p_{b}$.
The trajectory for parameter $p_1$ is plotted in red in Figure~\ref{fig:ambiguous_bif_diag}.
The trajectory for $p_2$ is blue.
Since $p_2$ is closer to the intermittency threshold $p_{b}$, the laminar phases of its trajectories are longer.
% Parameters $p_1$ and $p_2$ can be selected arbitrarily as long as theso that the laminar phases are long enough and similar situation as in Figure~\ref{fig:ambiguous_bif_diag} occurs.
Chaotic burst of the trajectory with a longer laminar phase occurs in the middle of two laminar phases of the trajectory with shorter laminar phases.
In our case, the chaotic burst $b$ occurs between chaotic bursts $a$ and $c$.
When creating two bifurcation diagrams with total number of iterations $0.95 c$ and plotting the last $c-(b-0.25(b-a))$ or $c-(b-0.75(b-a))$ iterations, two distinct bifurcation diagrams are obtained.

\begin{figure}[!h]
    \centering
    \begin{subfigure}{0.85\textwidth}
        \centering
        \includegraphics[width=\textwidth]{Figures/pomeau_manneville_bif_comparison_big{1}.png}
        \caption{}
    \end{subfigure}
    \hfill
    \begin{subfigure}{0.85\textwidth}
        \centering
        \includegraphics[width=\textwidth]{Figures/pomeau_manneville_bif_comparison_big{2}.png}
        \caption{}
    \end{subfigure}
    \begin{subfigure}{0.85\textwidth}
        \centering
        \includegraphics[width=\textwidth]{Figures/pomeau_manneville_bif_comparison_big{3}.png}
        \caption{}
    \end{subfigure}
    \begin{subfigure}{0.85\textwidth}
        \centering
        \includegraphics[width=\textwidth]{Figures/pomeau_manneville_bif_comparison_big{4}.png}
        \caption{}
    \end{subfigure}

    \caption{
        \textcolor{darkred}{ 
        Bifurcation diagrams of $\mathcal{P}_{\varepsilon}$ for $\varepsilon \in I := [4.4745829085, 4.4745829215]$.
        Interval $I$ is sampled to $400$ uniformly spaced points.
        Trajectories used for diagram construction:
        (a) $\mathcal{T}_{12000}^{25000}(\mathcal{P}_{\varepsilon}, x_0)$
        (b) $\mathcal{T}_{13900}^{25000}(\mathcal{P}_{\varepsilon}, x_0)$ 
        (c) $\mathcal{T}_{20000}^{25000}(\mathcal{P}_{\varepsilon}, x_0)$ 
        (d) $\mathcal{T}_{24000}^{25000}(\mathcal{P}_{\varepsilon}, x_0)$.
        Initial condition: $x_0 = 0.5$.
        }
    }
    \label{fig:ambiguous_bif_diag_example}
\end{figure}

This is illustrated in Figure~\ref{fig:ambiguous_bif_diag_example}, which shows that bifurcation diagrams differ significantly when the number of total iterations is the same, but the number of last iterations being plotted is different.
\par
A question arises what do we expect from a bifurcation diagram when we create it?
 Figure~\ref{fig:ambiguous_bif_diag_example} clearly shows that the same region of the bifurcation diagram can look very different.
An observer not knowing about intermittency could be misled.
It would be useful to have a tool that would warn a researcher looking at the bifurcation diagram about underlying intermittency and ambiguity it is causing.
The next chapter introduces such a tool.

\endinput