\chapter{Type-I Intermittency}
\label{chap:type-I intermittency}

\textcolor{darkred}{
This Chapter is concerned with the phenomena of intermittency, especially with type-I intermittency.
Firstly discovery and history of intermittency is mentioned.
Then the origion of intermittency is motivated.
Furthermore, the idea of laminar length and charasteristic relation is introduced.
Lastly, the ambiguity of bifurcation diagram caused by intermittency is shown.
}

\section{History}
\textcolor{darkred}{
In 1949, the word intermittency was used in context of turbulent flows by Batchelor and Townsend~\cite{Batchelor19491025}.
They used the term to depict signals that altered between approximately flat periods and and burst ones~\cite{Elaskar2017}.
}
\par
\textcolor{darkred}{
About three decades later, Pomeau and Manneville have studied intermittent transition to turbulence in an experiment with convective fluids~\cite{Pomeau1980}.
During this phenomenon, long periodic behavior was interrupted by chaotic bursts as the experiment control parameter changed.
They have shown that intermittent behavior is present in simple models such as the Lorenz model~\cite{Lorenz2004}.
Furthermore, they classified intermittency into three types based on model's Floquet multipliers.
They called these three types as I, II and III.
}
\par
\textcolor{darkred}{
Theoretical analysis of intermittency is centered around certain statistical properties, especially the reinjection probability density function.
While the standard theory considered this function to be constant~\cite{Dubois1983}, new advances in the field provide more complex explanation~\cite{Elaskar2022}.
In addition, to the three original types of intermittency, many new types have been described since then.
Some of them are V, X, on-off, eyelet and ring~\cite{Elaskar2022}.
}

\section{Motivation}
Intermittency is a behavior during which seemingly periodic phases in the trajectory of DDS are abruptly followed by chaotic bursts.
Periodic phases are called laminar.
Phases with chaotic bursts are called turbulent phases.
\par
\textcolor{darkred}{
To illustrate the concept of intermittency, a new dynamical system is needed.
Let $([0, 1], \mathcal{P}_{\varepsilon})$ be a DDS.
Map $\mathcal{P}_{\varepsilon}(x) = \left[ (1+\varepsilon)x+(1-\varepsilon)x^2 \right] \mod{1}$ is called the Pomeau-Manneville map~\cite{Manneville1980,Datseris2022}.
Intermittency is clearly visible in trajectory of $\mathcal{P}_{\varepsilon}$ diplayed in Figure~\ref{fig:intermittent_trajectory_example}.
Seemingly 2-periodic behavior which lasts for about 1000 iterations is interrupted by short chaotic bursts.
These chaotic bursts soon transform into seemingly 2-periodic behavior.
This pattern repeats.
}

\begin{figure}[!h]
    \centering
    \includegraphics[width=0.95\textwidth]{Figures/type_one_intermittency_example1.png}
    \caption{
        \textcolor{darkred}{
        Trajectory $T^{5500}_{0}(\mathcal{P}_{\varepsilon}, x_0)$ for $x_0 = 0.5$ and $\varepsilon = 4.47458$.
        }
    }
    \label{fig:intermittent_trajectory_example}
\end{figure}

\textcolor{darkred}{
Although there are many types of intermittency, the main focus of this thesis is on type-I intermittency~\cite{Pomeau1980,Bussac1982,DelRio2014}.
Such type of intermittent behavior arises when a DDS $(X, f_{p})$ is about to undergo a saddle-node bifurcation.
This type of bifurcation occurs when the tangent of the graph of $f_{p}$ at point $x^{*}$ is equivalent to the identity line.
Thus, point $x^{*}$ is a fixed point and $f_{p}'(x^{*}) = 1$.
As the parameter $p$ is slightly altered, point $x^{*}$ either splits into two fixed points or vanishes.
}
\par
\textcolor{darkred}{
An illustration of saddle-node bifurcation is shown in Figure~\ref{fig:saddle_node_bifurcation}.
A map $f_{p}(x) = p + x + x^2$ is displayed.
For $p = 0$, there is one single fixed point (red point).
Tangent of $f_{p}$ at the red point is $1$.
When $p$ is slightly increased, the fixed point vanishes.
On the other hand, when $p$ is slightly decreased, two fixed point emerge (blue points).
Note that if $p$ is decreased by a very small value, a narrow passage is formed between $f_{p}$ and the identity line.
Imagine that a cobweb diagram (Remark~\ref{def:cobweb}) is initited with initial condition $x_0$ close to the red point.
In that case it would take many iterations for the point to iterate through the narrow passage.
The narrower the passage is, the more iterations it would take.
}

\begin{figure}[!h]
    \centering
    \includegraphics[width=0.7\textwidth]{Figures/intermittency_i_local_map.png}
    \caption{
        \textcolor{darkred}{
        Map $f_{p}(x) = p + x + x^2$ for different parameters $p$. 
        Parameter $p_B = 0$.
        }
    }
    \label{fig:saddle_node_bifurcation}
\end{figure}

\par
\textcolor{darkred}{
Figure~\ref{fig:intermittent_cobweb_example} illustrates the iteration through a narrow passage.
There, the first laminar phase of Figure~\ref{fig:intermittent_trajectory_example} is displayed through the eyes of a cobweb diagram.
The trajectory in Figure~\ref{fig:intermittent_trajectory_example} starts off with a chaotic burst but soon transforms into a stable behavior.
The reason for that is seen in Figure~\ref{fig:intermittent_cobweb_example}(a), the point got trapped in a narrow passage formed between identity line and the graph.
It takes many iterations for the point to iterate through the passage (Figure~\ref{fig:intermittent_cobweb_example}(b)).
Figure~\ref{fig:intermittent_cobweb_example}(c) shows a subset of a trajectory in Figure~\ref{fig:intermittent_trajectory_example}.
The subset corresponds to the iteration through a narrow passage.
Once the point gets out of the narrow passage, it moves chaotically before it gets reinjected back into the narrow passage.
The second entrance into the narrow passage corresponds to the second laminar phase of Figure~\ref{fig:intermittent_trajectory_example}.
}

\begin{figure}[!h]
    \centering
    \includegraphics[width=0.95\textwidth]{Figures/type_one_intermittency_example2.png}
    \caption{
        \textcolor{darkred}{
        Subfigures (a) and (b):
        Cobweb diagram of 
        (a) $T^{472}_{38}(\mathcal{P}_{\varepsilon}^{2}, x_0)$ and
        (b) $T^{461}_{40}(\mathcal{P}_{\varepsilon}^{2}, x_0)$. 
        Subfigure (c):
        Trajectory $T^{520}_{0}(\mathcal{P}_{\varepsilon}^{2}, x_0)$.
        Parameters: $x_0 = 0.5$ and $\varepsilon = 4.47458$.
        }
    }
    \label{fig:intermittent_cobweb_example}
\end{figure}

\par
\textcolor{darkred}{
Intermittency can be though of as a continuous route to chaos (or from chaos).
Intermittency type-I can be spotted in a bifurcation diagram as a breakpoint or a periodic window.
Such periodic window can be seen in Figure~\ref{fig:bif_diag_example} at $r = 1+\sqrt{8} \approx 3.82842$.
At this parameter, the a non-periodic behavior suddently changed to a $3$-periodic behavior.
The process which caused this change is type-I intermittency.
When parameter $r$ gets closer and closer to the breakpoint, laminar phases of the trajectories get longer and longer until the whole trajectory is one infinitely long laminar phase.
}


\section{Laminar Phase Length}

\textcolor{darkred}{
This section describes the notion of laminar phase length, which is the number of iterations a point spends in the narrow passage.
This notion and especially the notion of the average laminar phase length (or average laminar length for short) will be used in the next Chapter.
}
\par
\textcolor{darkred}{
In the following text, the narrow passage from the last section shall be called \emph{laminar region} or \emph{laminar interval}.
Region outside of the laminar region shall be called \emph{turbulent region}.
}
\par
\textcolor{darkred}{
The local shape of the graph above (or below) the laminar region determines the type of intermittency.
The local shape is also called the \emph{local map}.
For maps with type-I intermittency, the local map is the following:
\begin{equation}
\varepsilon + x + a x^2 \label{eq:int_I_local_map}
\end{equation}
However, a correct local map is not enough for intermittency to occur in a DDS.
A reinjection mechanism, which reinjects the point from the turbulent region back into the laminar region, has to be present.
This reinjection mechanism is described by the reinjection probability density function and depends on a specific DDS.
}
\par
\textcolor{darkred}{
Let $(X, f_{p})$ be a DDS. Suppose that $f_{p}$ undergoes saddle-node bifurcation and $x^{*}$ is the corresponding single fixed point.
Suppose that $I_l = [ x^{*}-c, x^{*}+c ]$ is the laminar interval for some small $c$.
A \emph{pre-reinjection point} $x_p$ is a point which gets \emph{reinjected} into $I_l$, which means that there exists a point $x_r = f(x_p)$, $x_r \in I_l$.
The point $x_r$ shall be called a \emph{reinjected point}.
}
\par
The Figure~\ref{fig:pre_reinjection_example} illustrates the idea of pre-reinjection points and laminar interval $I_l$.
Note that the curve in subfigure (b) locally resembles the local map \eqref{eq:int_I_local_map} around the point $x^{*}$.
\begin{figure}[!h]
    \centering
    \begin{subfigure}{0.49\textwidth}
        \centering
        \includegraphics[width=\textwidth]{Figures/pre_reinjection_points_example{1}.png}
        \caption{}
    \end{subfigure}
    \hfill
    \begin{subfigure}{0.49\textwidth}
        \centering
        \includegraphics[width=\textwidth]{Figures/pre_reinjection_points_example{2}.png}
        \caption{}
    \end{subfigure}

    \caption{
        \textcolor{darkred}{ 
        Pre-reinjection points and laminar interval of $\mathcal{L}_{r}^{3}(x+x^{*})-x^{*}$ for $r = 3.827$ and $x^{*} \approxeq 0.51435$.
        (a) the full graph. 
        (b) a close-up to a region of interest. 
        Red dots: pre-reinjection points.
        Blue dots: reinjected points.
        Laminar interval: $I_l = [ x^{*}-c, x^{*}+c ]$ for $c = 0.03$.
        Orange line: $I_l$.
        Green dot: $x^{*}$.
        }
    }
    \label{fig:pre_reinjection_example}
\end{figure}
\par
Consider that $0 < \varepsilon \ll 1$.
By using the Equation~\eqref{eq:int_I_local_map} the difference between two successive points $x_n - x_{n-1} \approx dx/dl$ can be written as:
\begin{equation}
\frac{dx}{dl} = \varepsilon + a x^2 \label{eq:int_I_diff_eq}
\end{equation}
Thus continuous differential equation is obtained.
Solving from reinjected point $x$ to the end of laminar interval $c$ yields:
\begin{equation}
\int_{x}^{c} \frac{dx}{\varepsilon + a x^2} = \int_{0}^{l} dl \label{eq:int_I_diff_eq_step1}
\end{equation}
Solving for $l$ results in:
\begin{equation}
    l(x, c) = \frac{1}{\sqrt{a \varepsilon}} \left( \tan^{-1} \left( c \sqrt{\frac{a}{\varepsilon}} \right) - \tan^{-1} \left( x \sqrt{\frac{a}{\varepsilon}} \right) \right) \label{eq:laminar_length}
\end{equation}

\par
Hence for a given reinjected point $x$ and laminar interval width $c$ the laminar length $l$ can be calculated.
Laminar length means the number of iterations the point has to make to the end of laminar interval.
By calculating laminar length for many reinjected points and some fixed $c$ the average laminar length $l_{avg}$ can be calculated.
However, that is not very practical since it is not always clear how to choose $c$.
Fortunately, this problem was solved by formulation of so-called characteristic relation of type-I intermittency.

\section{Characteristic Relation}
To estimate the average laminar length $l_{avg}$ based on the parameter $\varepsilon$ a characteristic relation was formulated~\cite{Elaskar2017}.
It holds that $l_{avg} \propto 1/\sqrt{\varepsilon}$.
This relation states that the average laminar length $l_{avg}$ is inversely proportional to the square root of the parameter $\varepsilon$.
Thus, by calculating $\varepsilon$ of the local map in the vicinity of a fixed point $x^{*}$ the $l_{avg}$ can be estimated.
\par
To verify that the characteristic relation of type-I intermittency holds, a numerical simulation was conducted.
The relation is verified for the third iterate of the Logistic map $\mathcal{L}_{r}^{3}$.
\par
\textcolor{darkred}{
The third iterate of Logistic map $\mathcal{L}_{r}^{3}$ undergoes saddle-node bifurcation at $r = r^{*} := 1+\sqrt{8}$~\cite{Elaskar2022,Gordon20180411}.
Parameter $r^{*}$ is the start of the $3$-periodic window, the biggest periodic window visible in the bifurcation diagram of $\mathcal{L}_{r}$ (see Figure~\ref{fig:bif_diag_example}).
There are three fixed points of $\mathcal{L}_{r^{*}}$ associated with the bifurcation, $x^{*}_{1} \approx 0.1599$, $x^{*}_{2} \approx 0.5143$ and $x^{*}_{3} \approx 0.9563$.
For parameter $r$ slightly less than $r^{*}$, $\mathcal{L}_{r}^{3}$ in the proximity of each of the points $x^{*}_{1}$, $x^{*}_{2}$ and $x^{*}_{3}$ resembles the local map \eqref{eq:int_I_local_map}.
Figure~\ref{fig:pre_reinjection_example}~(b) shows the local map of $\mathcal{L}_{r}^{3}$ in the proximity of $x^{*}_{2}$.
The following explanation of the simulation will use $x^{*}_{1}$ as an example.
The simulation can be conducted in the same way for $x^{*}_{2}$ and $x^{*}_{3}$.
}
\par
The goal is to compute the average laminar length $l_{avg}$ for a point traveling through the laminar region around $x^{*}_{1}$.
The laminar region exists for the parameters $r$ slightly less than $r^{*}$.
In the experiment $30$ parameters $r$ were randomly chosen from the interval $[ r^{*}-10^{-5}, r^{*}-10^{-13} ]$.
Subsequently, the corresponding $\varepsilon$ parameters were calculated.
\par
To calculate $\varepsilon$ for the local map around $x^{*}_{1}$, the map $\mathcal{L}_{r}^{3}$ was shifted so that $x^{*}_{1}$ is in the origin.
In other words, $\mathcal{L}_{shift}(x) = \mathcal{L}_{r}^{3}(x + x^{*}_{1}) - x^{*}_{1}$.
Parameters $\varepsilon$ and $a$ are calculated using the Taylor polynomial of $\mathcal{L}_{shift}(x)$ at $x = 0$.
Hence, $\varepsilon = | \mathcal{L}_{shift}(0) |$ and $a = \mathcal{L}_{shift}''(0) / 2$.
$\varepsilon$ and $a$ are calculated for each of the $30$ random $r$ parameters.
Next, they are used to calculate the laminar length.
\par
To calculate the laminar length $l$ from \eqref{eq:laminar_length} corresponding to one parameter $r$, reinjection point $x$, parameters $c$, $\varepsilon$ and $a$ are needed.
The laminar interval width $c$ in the experiment was selected as $0.01, 0.03, 0.05$ and $0.07$.
In the following experiment, $c = 0.01$.
The same procedure can be done for the remaining $c$'s.
Parameters $c$, $\varepsilon$ and $a$ are known.
Next, reinjection points $x$ are calculated.
\par
To calculate the reinjection points corresponding to one of the $30$ $r$'s, the following procedure is used.
A point $x$ is randomly chosen from $[ 0-x^{*}_{1}, 1-x^{*}_{1} ]$.
Consequently, a check is conducted if the point gets reinjected into laminar interval.
In other words, check if $| \mathcal{L}_{shift}(x) | < c$.
The procedure is repeated until $N = 4000$ reinjected points are obtained.
\par
For each reinjected point $x$, the laminar length $l$ is calculated using \eqref{eq:laminar_length}.
Then, their average is calculated.
\par
The procedure is repeated for each of the $30$ $r$'s.
The results are plotted in log-log scale.
A linear fit of the results is calculated and the results are shown in Figure~\ref{fig:characteristic_relation_fit}.
Note that the results support that the $l_{avg} \propto 1/\sqrt(\varepsilon) = \varepsilon ^ {-1/2}$.
The linear fit for different $c$'s and $x^{*}$ is shown.
\par
Also note that it holds $l_{avg} \propto \varepsilon ^ {-1/2} \implies log_{10}(l_{avg}) \propto -1/2 \cdot \log_{10}(\varepsilon)$.
The Figure~\ref{fig:characteristic_relation_fit} shows that the slope of the linear fits is approximately $-1/2$.
That implies that the characteristic relation of type-I intermittency holds. 

\begin{figure}[!h]
    \centering
    \begin{subfigure}{0.49\textwidth}
        \centering
        \includegraphics[width=\textwidth]{Figures/characteristic_relation_fit{1}.png}
        \caption{}
    \end{subfigure}
    \hfill
    \begin{subfigure}{0.49\textwidth}
        \centering
        \includegraphics[width=\textwidth]{Figures/characteristic_relation_fit{2}.png}
        \caption{}
    \end{subfigure}
    \hfill
    \begin{subfigure}{0.49\textwidth}
        \centering
        \includegraphics[width=\textwidth]{Figures/characteristic_relation_fit{3}.png}
        \caption{}
    \end{subfigure}

    \caption{
        \textcolor{darkred}{ 
        $x$-axis: $30$ random $r \in [ r^{*}-10^{-5}, r^{*}-10^{-13} ], r^{*} = 1 + \sqrt{8}$, selected, corresponding $\varepsilon$ are calculated.
        $y$-axis: average $l_{avg}$ of laminar lengths (Equation~\eqref{eq:laminar_length}) for $4000$ points reinjected into laminar interval $I_{c} = [x^{*}-c, x^{*}+c], c \in \{ 0.01, 0.03, 0.05, 0.07 \}$ is calculated.
        Colorful lines: linear fit.
        Constant $s$: slope of the linear fit.
        Fixed point $x^{*}$: (a) $x^{*} = 0.1599$ (b) $x^{*} = 0.5143$ and (c) $x^{*} = 0.9563$.
        DDS: $\mathcal{L}_{r}^{3}$.
        }
    }
    \label{fig:characteristic_relation_fit}
\end{figure}

\section{Ambiguous Bifurcation Diagram}
\label{sec:ambiguous_bif_diag}
\textcolor{darkred}{
Bifurcation diagram (Remark~\ref{def: bif_diag}) is a common tool for scientists and researchers to study long term behavior of a DDS for varying parameters.
However, type-I intermittency poses a problem to the use of a bifurcation diagram.
The problem is that type-I intermittency causes the bifurcation diagram to be misleading in the proximity of a breakpoint.
}
\par
\textcolor{darkred}{
How intermittency might cause the diagram to be misleading can be illustrated as follows:
Let $(X, f_{p})$ be a DDS.
Suppose a bifurcation diagram of $f_{p}$ is constructed for some $x_0$, $n_{total}$, $n_{last}$ and $p_{range}$ (see Algorithm~\ref{alg:bif_diag}).
Next, assume that for parameter $p$, $f_{p}$ is intermittent.
The bifurcation diagram is constructed by projecting trajectories $\mathcal{T}_{n_{total}-n_{last}}^{n_{total}}(f_{p}, x_0)$.
If the trajectory constructed for iterations $n \in [n_{total}, n_{total}-n_{last}]$ fully lies in the laminar phase, then only finite number of dots will be displayed on the diagram.
On the other hand, if the parameter $n_{last}$ is increased, then the projected trajectory might reach the turbulent phase and the diagram will show a band of points.
This might not be a big deal for individual parameters $p$, but this ambiguity might affect a whole range of parameters $p$ near the breakpoint.
}
\par
\textcolor{darkred}{
How a whole range of parameters might be affected can be illustrated as follows:
Suppose that $p_{R}, p_{B}$ are some parameters near the breakpoint $p_{bif}$.
Assume that $p_{R} \neq p_{B} \wedge p_{R} < p_{B} < p_{bif}$.
If $p_{R}$ and $p_{B}$ are close enough to $p_{bif}$ they both have intermittent trajectories.
Additionally, it follows that the laminar phases of $f_{p_{R}}$ are shorter than laminar phases of $f_{p_{B}}$.
That is because the closer the parameter is to $p_{bif}$, the longer the laminar phases get.
An example of this can be seen in Figure~\ref{fig:ambiguous_bif_diag}.
The red trajectory (which is partially hidden behind the blue one) has shorter laminar phases than the blue trajectory.
Notice the three letters, a, b and c, in the Figure~\ref{fig:ambiguous_bif_diag}.
These letters correspond to an iteration $n$ at which the a specific chaotic burst occurs.
Now, two bifurcation diagrams are created using these parameters:
\begin{enumerate}
    \item $n_{last} = c-(b-0.25(b-a))$
    \item $n_{last} = c-(b-0.75(b-a))$
\end{enumerate}
Other than that, both 1. and 2. have the same parameters: $n_{total} = 0.95c$ and some $x_0$, $p_{range}$ and $f_{p}$.
If such two bifurcation diagrams are constructed, they will both look very different in the proximity of $p_{bif}$.
}

\begin{figure}[!h]
    \centering
    \includegraphics[width=0.95\textwidth]{Figures/bif_diag_ambiguity_creation_sketch.png}
    \caption{
        \textcolor{darkred}{
        Sketch describing how to create an ambiguous bifurcation diagram. 
        Red graph: $T^{17000}_{0}(\mathcal{P}_{4.47458285}, x_0)$. 
        Blue graph: $T^{17000}_{0}(\mathcal{P}_{4.47458285+4 \cdot 10^{-8}}, x_0)$. 
        Initial condition: $x_0 = 0.5$.
        }
    }
    \label{fig:ambiguous_bif_diag}
\end{figure}

\par
\textcolor{darkred}{
Figure~\ref{fig:ambiguous_bif_diag_example} shows an example of $4$ bifurcation diagrams of $\mathcal{P}_{\varepsilon}$.
They are created for the same parameters except for $n_{last}$, which is different for each one of them.
A phenomenon, which shall be called ambiguity in the bifurcation diagram, can be observed.
Ambiguity in this context means that each bifurcation diagram looks drastically different for varying $n_{last}$.
}

\begin{figure}[!h]
    \centering
    \begin{subfigure}{0.85\textwidth}
        \centering
        \includegraphics[width=\textwidth]{Figures/pomeau_manneville_bif_comparison_big{1}.png}
        \caption{}
    \end{subfigure}
    \hfill
    \begin{subfigure}{0.85\textwidth}
        \centering
        \includegraphics[width=\textwidth]{Figures/pomeau_manneville_bif_comparison_big{2}.png}
        \caption{}
    \end{subfigure}
    \begin{subfigure}{0.85\textwidth}
        \centering
        \includegraphics[width=\textwidth]{Figures/pomeau_manneville_bif_comparison_big{3}.png}
        \caption{}
    \end{subfigure}
    \begin{subfigure}{0.85\textwidth}
        \centering
        \includegraphics[width=\textwidth]{Figures/pomeau_manneville_bif_comparison_big{4}.png}
        \caption{}
    \end{subfigure}

    \caption{
        \textcolor{darkred}{ 
        Bifurcation diagrams of $\mathcal{P}_{\varepsilon}$ for $\varepsilon \in I := [4.4745829085, 4.4745829215]$.
        Interval $I$ is sampled to $400$ uniformly spaced points.
        Trajectories used for diagram construction:
        (a) $\mathcal{T}_{12000}^{25000}(\mathcal{P}_{\varepsilon}, x_0)$
        (b) $\mathcal{T}_{13900}^{25000}(\mathcal{P}_{\varepsilon}, x_0)$ 
        (c) $\mathcal{T}_{20000}^{25000}(\mathcal{P}_{\varepsilon}, x_0)$ 
        (d) $\mathcal{T}_{24000}^{25000}(\mathcal{P}_{\varepsilon}, x_0)$.
        Initial condition: $x_0 = 0.5$.
        }
    }
    \label{fig:ambiguous_bif_diag_example}
\end{figure}

\par
\textcolor{darkred}{
The ambiguity in the bifurcation diagram might be misleading to someone, who is not aware of it.
A tool that would warn researchers about intermittency and the ambiguity it is causing might be useful.
Such a tool will be developed in the Chapter~\ref{chapter:intdetection}.
}

\endinput