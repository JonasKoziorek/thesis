\chapter{Type I Intermittency}
Intermittency is a very interesting phenomenon in the field of dynamical systems.
The investigation of intermittency was initiated in 1980 by Pomeau and Manneville \cite{Pomeau1980}.
They were trying to explain an intermittent phenomenon in the fluid dynamics of convective fluids.
During this phenomenon, long periodic behaviour was abrupted by a chaotic burst as the experiment control parameter changed.
Subsequently, Pomeau and Manneville found analogous behaviour in the Lorenz system \cite{Lorenz2004}.
Through a numerical study of this system, they introduced the concept of intermittency and classified it into three categories. \cite{Pomeau1980}
This chapter deals with type I intermittency.

\section{Mathematical Motivation}

Intermittency is a behavior when seemingly periodic phases in the evotion of a dynamical system are abrupted by chaotic bursts.
Periodic phases are called laminar.
Phases with chaotic bursts are called turbulent phases.
This naming convention originated in fluid dynamics which motivated study of intermittency. \cite{Pomeau1980}
\\
In figure \ref{fig:intermittent_evolution_example} you can see an example of intermittency.
In this case intermittency is clearly visible when looking at the evolution graph.
Seemingly 2-periodic behavior which last for about 1000 iteration is interrupted by short chaotic bursts.
These chaotic burst transform back into 2-periodic behavior.
This pattern goes on.

\begin{figure}[!h]
    \centering
    \includegraphics[width=1.0\textwidth]{Thesis/Figures/type_one_intermittency_example1.png}
    \caption{Intermittency in evolution graph}
    \label{fig:intermittent_evolution_example}
\end{figure}

Evolution graph \ref{fig:intermittent_evolution_example} exhibits type I intermittency which is related to \hyperref[def:saddle_node_bif]{saddle-node bifurcation}.
The reason intermittency is born is illustrated in figure \ref{fig:intermittent_cobweb_example}.
When dynamical system $f_P$ is about to undergo inverse saddle-node bifurcation as parameter $P$ changes, graph of $f_P$ is getting closer to identity function $g(x)=x$.
Just before they touch small passage is formed between these two curves.
If point $x_n$ gets mapped into this narrow passage, it takes a lot of iterations for it to get out of this passage again.
The closer the curve of $f_P$ and identity function are the narrower the passage is and the longer it gets to iterate through it.
An illustration of the iteration through narrow passage is in figure \ref{fig:intermittent_cobweb_example} plot B.
This iteration corresponds to first laminar phase from figure \ref{fig:intermittent_evolution_example}.

\begin{figure}[!h]
    \centering
    \includegraphics[width=1.0\textwidth]{Thesis/Figures/type_one_intermittency_example2.png}
    \caption{Intermittency in cobweb diagram}
    \label{fig:intermittent_cobweb_example}
\end{figure}

\section{Numerical Simulation}
We propose that type I intermittency occurs when there is "breakpoint" in bifurcation diagram between chaotic behavior on the left and periodic behavior on the right.
We haven't found literature to support this claim yet.
However, our numerical simulations support this claim.
Results are presented in figures \ref{fig:complex_pomeau_manneville}, \ref{fig:complex_logistic}, \ref{fig:complex_henon}, \ref{fig:complex_duffing}.
In plots marked as D in each of these figures, the breakpoint in bifurcation diagram is shown.
In plots marked as E in each of these figures, intermittent evolution near the breakpoint is shown.
\par
Since left side of the breakpoint is chaotic and right side is periodic, there must be a transition from chaos to stability.
This transition is related to intermittency and narrow passage mentioned in figure \ref{fig:intermittent_cobweb_example}.
(Claim) The breakpoint corresponds to the value of saddle-node bifurcation.
We can call this breakpoint $\epsilon_{sn}$.
The closer the parameter $\epsilon$ gets to breakpoint $\epsilon_{sn}$ the narrower the passage gets.
Hence it takes more iterations to iterate through the passage.
Hence laminar phases visible in the evolution graph get longer and longer as $\epsilon$ gets closer to $\epsilon_{sn}$.
Hence $\epsilon \rightarrow \epsilon_{sn} \implies l_{avg} \rightarrow \infty$ where $l_{avg}$ is the average length of laminar phase.
When $\epsilon = \epsilon_{sn}$ behavior is periodic.
(End Of Claim)

\section{Ambiguous Bifurcation Diagram}

\endinput