\chapter{Selected Models}
We have chosen representatives of various well known discrete dynamical systems in one and two dimensions. These systems come from diverse backgrounds and should illustrate that DDS models can be used in many scientific domains. In the next sections we will describe and motivate each respective model.

\section{1D Models}

\subsection{Logistic map}
\label{subsec:logistic map}

\subsubsection{Motivation}
Probably the most widely known and most throughout studied discrete dynamical system.
Its bifurcation diagram became the most important image representing field of discrete dynamical systems.
This mapping became popular after it was mentioned by May \cite{May19760610}.
The model is used to represent population dynamics in an environment with limited resources.

\subsubsection{Mathematical Representation}
The logistic map is defined in a following way

\begin{equation}
    x_{n+1} = r x_{n} ( 1 - x_{n} )
\end{equation}

Typically the parameter $r$ is chosen from range $r \in [0, 4]$ so that $x$ stays in the range $[0,1]$.
As mentioned earlier, model was used as a simple model for population dynamics.
Variable $x$ represents number of species normalized down to a percentage.
Parameter $r$ represents the growth rate of the population in each generation.

\subsubsection{Numerical Experimentation}
For varying parameters, the mapping is able to exhibit wide range of behaviours.
Some of them are illustrated in the Figure \ref{fig:complex_logistic}.

\subsection{Pomeau-Manneville map}

\subsubsection{Motivation}
Another chosen model is a map we call Pomeau-Manneville map.
This map was first mentioned in article by Manneville \cite{Manneville1980} and purpose of its creation was to study intermittency.
However intermittency was first introduced by Pomeau and Manneville \cite{Pomeau1980} not long before \cite{Manneville1980} and hence we call the map by name of both authors.

\subsubsection{Mathematical Representation}

\begin{align}
    x_{n+1} = ((1 + \varepsilon) x_{n} + (1 - \varepsilon) x_{n}^2) \pmod{1}
\end{align}

\subsubsection{Numerical Experimentation}
For varying parameters, the mapping is able to exhibit wide range of behaviours.
Some of them are illustrated in the Figure \ref{fig:complex_pomeau_manneville}.

\section{2D Models}

\subsection{Henon map}

\subsubsection{Motivation}

This model was first suggested by Henon \cite{Henon1976} as a poincare map of the lorenz system \cite{Lorenz2004}.
The aim was to come up with a map which would preserve some of the important characteristics of lorenz system but would be easier to study numerically.

\subsubsection{Mathematical Representation}
\begin{equation}
\begin{split}
    x_{n+1} &=  1 - a x_{n}^2 + y_{n} \\
    y_{n+1} &=  b x_{n}
\end{split}
\end{equation}

\subsubsection{Numerical Experimentation}
For varying parameters, the mapping is able to exhibit wide range of behaviours.
Some of them are illustrated in the Figure \ref{fig:complex_henon}.

\subsection{Duffing map}

\subsubsection{Motivation}

This map is a discrete version of duffing equation first introduced by engineer Georg Duffing.
Duffing equation is used to model some damped and driven oscillators \cite{Urrea2022}.


\subsubsection{Mathematical Representation}
\begin{equation}
\begin{split}
    x_{n+1} &=  y_{n} \\
    y_{n+1} &=  -b x_{n} + a y_{n} - y_{n}^3
\end{split}
\end{equation}

\subsubsection{Numerical Experimentation}
For varying parameters, the mapping is able to exhibit wide range of behaviours.
Some of them are illustrated in the Figure \ref{fig:complex_duffing}.

\endinput