\documentclass[lualatex,hyperref={pdfencoding=auto}]{beamer}
\usepackage[czech]{babel}
\usepackage{amssymb}
\usepackage{amsmath}
\usepackage{float}
\usepackage{tikz,etoolbox,environ,ifthen,kvsetkeys,kvoptions}
\usetheme[fei]{vsb}
\usepackage{qcircuit}
\usepackage{subfig}

\title[Obhajoba bakalářské práce]{Algoritmická detekce intermitence typu I}

\author{Jonáš Koziorek}
\institute[KOZ0325]
    {
        \vspace{2mm}Vedoucí: prof. RNDr. Marek Lampart, Ph.D. \\
        VŠB -- Technická univerzita Ostrava
    }
\date[5.~6.~2024]{5.~června 2024}


\begin{document}

\begin{frame}
	\frametitle{Cíl práce}

\begin{itemize}
    \item detekovat intermitenci typu I
    \begin{itemize}
        \item diskrétní dynamické systémy
        \item systémy 1. dimenze
    \end{itemize}
    \item upozornit na nejednoznačnost bifurkačního diagramu
\end{itemize}

\end{frame}




\begin{frame}
	\frametitle{Úvod do dynamických systémů}
 
\begin{itemize}
    \item Diskrétní dynamický systém: $(X, f)$,  $f: X \rightarrow X$
    \item Počáteční podmínka: $x_0$
    \item $n$-tá iterace: $x_{n}=f^{n}(x_0)=\underbrace{f \circ f \circ \cdots \circ f}_{n\text{-krát}}(x_0)$
    \item Pevný bod: $f(x_{0})=x_{0}$
    \item Periodický bod periody $n$: $f^{n}(x_{0})=x_{0}$
    \item Periodická orbita: $\{f^{n+m}(x_0):m \in \mathbb{N}\}$
    \item Logistické zobrazení: $\mathcal{L}_{r}(x)=rx(1-x)$, $r \in [0, 4]$, $x \in [0, 1]$
\end{itemize}

\end{frame}



\begin{frame}
	\frametitle{Trajektorie}

\begin{itemize}
     \item  Postupné iterace: $f^0(x_0), f^1(x_0), f^2(x_0), f^3(x_0), \cdots  = x_0, x_1, x_2, x_3, \cdots$

\end{itemize}

\begin{figure}[!h]
    \centering
    \includegraphics[width=0.6\textwidth]{./trajectory.png}
    \caption{
        Prvních 50 iterací systému $\mathcal{L}_{3.828}$ s $x_0=0.5$.
    }
\end{figure}

\end{frame}

\begin{frame}
	\frametitle{Bifurkační diagram}
 
\begin{figure}[!h]
    \centering
    \includegraphics[width=0.9\textwidth]{bifurcationdiagram.png}
    \caption{
        Bifurkační diagram systému $\mathcal{L}_{r}$ pro $r \in [2.81, 3.87]$, $x_{0}=0.5$.
    }
\end{figure}
 
 
\end{frame}

\begin{frame}
	\frametitle{Intermitence}
 
\begin{itemize}
     \item Alternace (pseudo) periodického a chaotického chování
    \item $\mathcal{P}_{\varepsilon}(x) = \left[ (1+\varepsilon)x+(1-\varepsilon)x^2 \right] \pmod{1}$, $x \in [0, 1]$
 \end{itemize}
 
\begin{figure}[!h]
    \centering
    \includegraphics[width=1.0\textwidth]{intermittent_trajectory.png}
    \caption{
        Intermitentní trajektorie systému $\mathcal{P}_{4.47458}$ pro $x_0=0.5$.
    }
\end{figure}

\end{frame}


\begin{frame}
	\frametitle{Intermitence typu I}
 
\begin{figure}[!h]
    \centering
    \includegraphics[width=0.8\textwidth]{breakpoint.png}
    \caption{Bod zlomu v bifurkačním diagramu.}
\end{figure}

\end{frame}



\begin{frame}
	\frametitle{Nejednoznačný bifurkační diagram}

\begin{figure}
    \centering
    {\includegraphics[width=0.9\textwidth]{pomeau_manneville_bif_comparison_big{3}.png} }
    \qquad
    {\includegraphics[width=0.9\textwidth]{pomeau_manneville_bif_comparison_big{4}.png} }
    \caption{Porovnání bifurkačních diagramů pro jiný počet promítáných iterací.}
    \label{fig:example}
\end{figure}

\end{frame}


\begin{frame}
    \frametitle{Algoritmus detekce}
    
\begin{itemize}
    \item Globální prohledávání
        \begin{itemize}
            \item Globální prohledávání hrubou silou
            \item Detekce periodických orbit
        \end{itemize}
    \item Lokální prohledávání
        \begin{itemize}
            \item Naivní lokální prohledávání
            \item Optimalizace pomocí hejna částic s vnořenými vrstvami
        \end{itemize}
    \item Vybarvení bifurkačního diagramu
\end{itemize}

\end{frame}


\begin{frame}
    \frametitle{Globální prohledávání hrubou silou}
    
    \begin{figure}[!h]
        \centering
        \includegraphics[width=0.9\textwidth]{naive_global_search.png}
        \caption{Stabilní periodické body detekované hrubou silou.}
    \end{figure}
    
\end{frame}


\begin{frame}
    \frametitle{Detekce periodických orbit}
    
    \begin{figure}[!h]
        \centering
        \includegraphics[width=1.0\textwidth]{upos.png}
        \caption{Periodické body $\mathcal{L}_{3.6}(x)$ periody $6$.}
    \end{figure}
\end{frame}


\begin{frame}
    \frametitle{Globální prohledávání}
    
    \begin{figure}[!h]
        \centering
        \includegraphics[width=0.9\textwidth]{bif_search.png}
        \caption{Přibližné lokace bodů zlomu.}
    \end{figure}
    
\end{frame}


\begin{frame}
    \frametitle{Lokální prohledávání}
    
    \begin{figure}[!h]
        \centering
        \includegraphics[width=0.9\textwidth]{break_point_search_example.png}
        \caption{Přesná lokace bodu zlomu.}
    \end{figure}

\end{frame}


\begin{frame}
    \frametitle{Vybarvení bifurkačního diagramu}
    
\begin{figure}[!h]
    \centering
    \includegraphics[width=0.9\textwidth]{logistic_map_coloring_example.png}
    \caption{Vybarvení bifurkačního diagramu.}
\end{figure}

\end{frame}


\begin{frame}
    \frametitle{Závěr}

\begin{itemize}
    \item Dosaženo detekce pro 1D diskrétní dynamické systémy
    \item Upozornění na intermitenci skrze vybarvení bifurkačního diagramu
    \item Možné vylepšení
    \begin{itemize}
        \item vícedimenzionální systémy
        \item jiné typy intermitence
        \item spojité systémy (Poincarého řez)
        \item numerická kontinuace
    \end{itemize}
\end{itemize}

    
\end{frame}


\end{document}
