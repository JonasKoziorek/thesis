% First of all, include a document class and options
\documentclass[english,bachelor]{diploma}
% There are some additional packages
\usepackage[autostyle=true]{csquotes} % enhanced support for quotation marks, support for biblatex package
\usepackage[backend=biber, style=iso-numeric, alldates=iso]{biblatex} % bibliography
\usepackage{dcolumn} % numeric column type
% \usepackage{subfig} % subtables and subfigures
\usepackage[julia]{diplomalst}
\usepackage{amsmath}
\usepackage{amsfonts}
\usepackage{hyperref}

\usepackage{algorithm}
\usepackage{algpseudocode}
\usepackage{subcaption}
\usepackage{amssymb}


%%%%%%%%%%%%%%%%%%%%%%%%%%%%%%%%%%%%%%%%
%%%%%%%%%%%%%%%%%%%%%%%%%%%%%%%%%%%%%%%%

% my own commands
%%%%%%%%%%%%%%%%%%%%%%%%%%%%%%%%%%%%%%%%
\newcommand\norm[1]{\lVert#1\rVert}
\definecolor{darkred}{RGB}{139,0,0}
%%%%%%%%%%%%%%%%%%%%%%%%%%%%%%%%%%%%%%%%


% Next, enter data for leading pages
\ThesisAuthor{Jonáš Koziorek}

\ThesisSupervisor{prof. RNDr. Marek Lampart, Ph.D.}

\CzechThesisTitle{Algoritmická detekce intermitence typu I}

\EnglishThesisTitle{Algorithmic detection of type-I intermittency}

\SubmissionYear{2024}

\ThesisAssignmentFileName{ThesisAssignment.pdf}

\Acknowledgement{
    I would like to thank prof. RNDr. Marek Lampart, Ph.D. who guided me through the process of writing this thesis. 
    I would also like to thank to all my teachers at VŠB-TUO who taught me so much in such a short time.
}

\CzechAbstract{
    Cílem práce je vyvinout algoritmus k detekci intermitence typu I v diskrétních dynamických systémech.
    Intermitence v dynamickém systému je jev, při kterém se stav systému mění mezi zdánlivě stabilním a chaotickým.
    Práce se zaměřuje na intermitenci typu I, která souvisí se sedlovými bifurkacemi.
    Nejprve je představena teorie diskrétních dynamických systémů a intermitence typu I.
    Dále je vysvětleno, proč má detekce intermitence typu I smysl.
    Kombinací několika algoritmů od různých autorů je detekce dosažena pro logistické zobrazení.
    Výsledky detekce jsou prezentovány graficky pomocí vybarvení bifurkačního diagramu.
    Nakonec je zmíněna implementace v programovacím jazyce Julia.
    }

\CzechKeywords{
    intermitence typu I; diskrétní dynamické systémy; logistické zobrazení; pevný bod; sedlová bifurkace; průměrná délka laminární fáze; programovací jazyk Julia; bifurkační diagram
    }

\EnglishAbstract{
    The goal of the thesis is to develop an algorithm for detection of type-I intermittency in discrete dynamical systems.
    Intermittency in a dynamical system is a phenomenon where the state of the system alternates between being seemingly stable and chaotic.
    The focus of the thesis is on the type-I intermittency which is associated with saddle-node bifurcations.
    Firstly, the theory of discrete dynamical systems and of type-I intermittency is introduced.
    Furthermore, the importance of detection of type-I intermittency is explained.
    By combining several algorithms of other authors, the detection is achieved for the Logistic map.
    The results of the detection are presented graphically through colorization of a bifurcation diagram.
    Lastly the implementation in Julia programming language is mentioned.
    }

\EnglishKeywords{
    type-I intermittency; discrete dynamical systems; Logistic map; fixed point; saddle-node bifurcation; average laminar phase length; Julia programming language; bifurcation diagram
    }


\AddAcronym{$\mathbb{R}$}{Real numbers}
\AddAcronym{$\mathbb{N}$}{Natural numbers}
\AddAcronym{$\mathcal{L}_{r}$}{Logistic map}
\AddAcronym{$\mathcal{P}_{\varepsilon}$}{Pomeau-Manneville map}
\AddAcronym{$l_{avg}$}{average laminar phase length}
\AddAcronym{DDS}{discrete dynamical system}
\AddAcronym{SPO}{stable periodic orbit}
\AddAcronym{UPO}{unstable periodic orbit}
\AddAcronym{DL}{Davidchack-Lai algorithm}
\AddAcronym{SD}{Schmelcher-Diakonos algorithm}
\AddAcronym{BWJ}{Bu-Wang-Jiang algorithm}
\AddAcronym{NLS}{Naive Local Search}
\AddAcronym{NGS}{Naive Global Search}
\AddAcronym{PSO}{Particle Swarm Optimization}
\AddAcronym{NLPSO}{Nested-Layer Particle Swarm Optimization}
\AddAcronym{LSGO}{Local Search using Global Optimization}

% Bibliography resources for BibLaTeX
\addbibresource{bibliography.bib}
\addbibresource{bibliography2.bib}

% New table column type for numeric values
\newcolumntype{d}[1]{D{.}{.}{#1}}

% Beginning of the document
\begin{document}

% Leading pages printing
\MakeTitlePages

% Do theses contain any figures? If so, print the list of figures and then move to the next page.
% If not, delete the next two macros.
\listoffigures
\clearpage

% Do theses contain any tables? If so, print the list of table and then move to the next page.
% If not, delete the next two macros.
% \listoftables
% \clearpage

\chapter{Introduction}
\label{sec:Introduction}

Real world systems such as climate, economics, population dynamics or celestial mechanics are complex.
In fact, they are so complex that, before the advent of computers, some of these systems were immune to thorough scientific exploration.
Fortunately, as technology had advanced, the scientific community was able to tackle increasingly complicated systems and problems.
At the heart of the study of complex systems is the theory of dynamical systems \cite{Devaney20211026, Hirsch2013, Strogatz201854}.
\par
Dynamical systems theory serves as a foundational framework for understanding the behavior of systems that evolve over time, encompassing a diverse array of phenomena across various scientific disciplines.
From celestial mechanics~\cite{Holmes1990} to population dynamics~\cite{Hastings1993,Hadeler2001}, from climate modeling~\cite{Ghil2023} to neural networks~\cite{Cessac2009, Vogt2020, Li2019}, disease spread~\cite{Ritelli20210930} to logistics~\cite{Kumara2003}, dynamical systems theory provides a unified language to describe the evolution of complex systems.
\par
At its core, a dynamical system is characterized by a set of rules that govern how its state changes over time.
These rules can be deterministic or stochastic, linear or nonlinear, discrete or continuous.
The research of dynamical systems involves analyzing the long-term behavior of these systems, seeking to understand their stability, periodicity, chaos, and other emergent properties.
Many algorithms are being developed seeking to detect, control, and predict system's dynamical properties.
\par
\textcolor{blue}{
Arguably the most interesting dynamical property is chaos.
This phenomena is characterized by strong sensitivity of a system to initial conditions.
Small perturbations of the initial conditions can lead to drastically different long-term behavior.
Despite the apparent randomness, chaotic systems posses underlying patterns and structures.
Interestingly enough, chaos can emerge in very simple systems~\cite{Lorenz2004,May19760610}.
It is not a surprise that complex such as nature also exhibit chaos~\cite{Toker2020}.
}
\par
Although, chaos has been the focus of much attention, there are many other properties that are worth researching.
A dynamical property that is central to this thesis is called intermittency.
Intermittency is a fascinating phenomenon of dynamical systems, where seemingly stable behavior can abruptly transition to chaotic behavior, only to return to stability just as unexpectedly.
The mathematical study of intermittency originated in the 1980s through observation of the intermittent transition to turbulence in convective fluids \cite{Pomeau1980}.
Since then, the phenomenon has garnered significant attention across diverse scientific disciplines due to its relevance in understanding phenomena such as gas flow dynamics~\cite{Pizza20110926}, brain dynamics~\cite{Paradisi2013}, and economic fluctuations~\cite{Chian2007}, among others.
\par
Understanding intermittency poses theoretical and practical challenges.
Theoretical investigations delve into the mathematical foundations of intermittency, seeking to unravel the underlying mechanisms that govern its occurrence~\cite{Elaskar2017, Elaskar2023}.
Practical applications, on the other hand, often involve the identification, detection, and characterization of intermittent behavior in real-world systems, paving the way for predictive modeling and control strategies.
\par
Detection of intermittent behavior is the main aim of this thesis, more specifically detection of type-I intermittency~\cite{Pomeau1980,Bussac1982,DelRio2014}.
This type is one of the most well-known and studied types that have been discovered.
% It has been observed in several real-world systems \cite{Zebrowski2004,Zebrowski2005,Parthimos2001,Aikawa1990,Shiau1995,Storchi2010,Dimitriu2008,Chiriac20070701}.
It has been observed in several real-world systems \cite{Zebrowski2004,Parthimos2001,Dimitriu2008,Chiriac20070701}.
Developing better algorithmic tools to deal with this dynamical property is an important step towards better understanding and managing complex systems.
With improved algorithms, researchers and practitioners can more effectively detect, analyze, and predict intermittent behavior, allowing a deeper understanding of the underlying dynamics and facilitating more informed decision-making in various domains.

\bigskip

This thesis aims to construct an algorithmic approach for detection of type-I intermittency in discrete dynamical systems.
However, before such approach can be developed, several notions from the standard theory have to be introduced.
\par
The thesis is organized as follows:

\begin{description}
	\item \textbf{Chapter 1} -- provides motivation behind the thesis.
	\item \textbf{Chapter 2} -- introduces the theory of discrete dynamical systems.
	\item \textbf{Chapter 3} -- introduces type-I intermittency.
	\item \textbf{Chapter 4} -- develops a method for detection of type-I intermittency.
	\item \textbf{Chapter 5} -- discusses the software implementation.
	\item \textbf{Chapter 6} -- concludes the findings of this thesis.
\end{description}

\endinput
\section{Discrete Dynamical Systems}

\begin{definition}[Discrete dynamical system]
    A \emph{discrete dynamical system} (abbr. DDS) is a tuple $\left( X, f \right)$ where X is a non-empty set and $f: X \rightarrow X$ is a map~\cite{Brin20100706}.
    Set X is called a \emph{state space}.
    Set X is \emph{invariant} under $f$ if and only if $f(X) \subseteq X$, moreover set $X$ is \emph{strongly invariant} if and only if $f(X) = X$.
    If map $f$ depends on a parameter $p$, we denote it as $f_p$.
\end{definition}

\begin{remark}[Logistic map]
    \textcolor{red}{
    This thesis primarily focuses on the so called Logistic map.
    This map might be considered the most widely known discrete dynamical system [cite here].
    Although it is mathematically very simple, it exhibits a wide range of complex behavior.
    Logistic map was popularized by May~\cite{May19760610}.
    Originally the map was used to represent population dynamics in an environment with limited resources.
    }
    \par
    \textcolor{red}{
    Mathematically, Logistic map can be thought of as a DDS of the form $\left( [0, 1], \mathbb{L}_{r} \right)$ where $\mathbb{L}_{r}: [0,1] \rightarrow [0,1]$, $\mathbb{L}_{r}(x) = rx(1-x)$.
    In order to ensure that $x$ stays in the range $[0,1]$, parameter $r$ is chosen from range $r \in [0, 4]$.
    }
    \par
    \textcolor{red}{
    As mentioned earlier, Logistic map was used as a simple model for population dynamics.
    Variable $x$ represents a population of some species.
    $x$ is understood as a normalized population.
    $1$ means that the population is at its maximum and $0$ means that the population is extinct.
    Parameter $r$ represents the growth rate of the population in each generation. [cite here]
    }
\end{remark}

\begin{definition}[\textit{$\mathbf{n}$}-th iteration]
    Let $\left( X, f \right)$ be a DDS. 
    The first iteration of map $f$ for initial condition $x_0$ is represented by equation
    \begin{eqnarray}
        x_{1}  & = & f(x_{0})
    \end{eqnarray}
    $n$-th iteration of $x_0$ is
    \begin{equation}
    x_{n} = f^{n}(x_0) =
        \begin{cases}
            f^{0}(x_0) = x_0 & \text{if } n = 0. \\
            \underbrace{f \circ f \circ \cdots \circ f}_\text{$n$ times}(x_0) & \text{if } n > 0. 
        \end{cases}
    \end{equation}
    Hence $f^{n}$ is $n$-fold composition of the map f.
\end{definition}

\begin{definition}[Graph]
\textcolor{red}{
    Let $\left( X, f \right)$ be a DDS. A graph of $f$ is graph of $f(x)$ for each $x \in X$.
}
\end{definition}

\begin{remark}
    \textcolor{red}{
    An example of a graph of $\mathbb{L}$ is shown in Figure~\ref{fig:logistic_graph_example}.
    }
    \par
    \textcolor{red}{
    It is sometimes useful to plot the graph of $n$-th iteration of $f$.
    The Figure~\ref{fig:logistic_nthcomp_example} shows an example of the graph of $n$-th iteration of $\mathbb{L}_{3.8}$ for different $n$.
    }
\end{remark}

\begin{figure}[!h]
    \centering
    \includegraphics[width=0.5\textwidth]{DDS/Figures/logistic_graph_example.png}
    \caption{
        \textcolor{red}{
        Graphs of the Logistic map $\mathbb{L}_{r}$ for different parameters $r$.
        }
    }
    \label{fig:logistic_graph_example}
\end{figure}

\begin{figure}[!h]
    \centering
    \includegraphics[width=1.0\textwidth]{DDS/Figures/logistic_nthcomp_example.png}
    \caption{
        \textcolor{red}{
        Graphs of the Logistic map $\mathbb{L}_{3.8}^{n}$ for different $n$.
        }
    }
    \label{fig:logistic_nthcomp_example}
\end{figure}

\begin{definition}[Trajectory]
    Let $\left( \mathbb{R}, f \right)$ be a DDS. Successive iterates of $f$ with initial condition $x_0 \in X$ form a trajectory $T(f, x_0)$ where
    \begin{eqnarray}
        T(f, x_0) = f^0(x_0), f^1(x_0), f^2(x_0), f^3(x_0), \cdots  = x_0, x_1, x_2, x_3, \cdots
    \end{eqnarray}
    More precisely, for $n \leq m$ we define
    \begin{eqnarray}
        T_{n}^{m}(f, x_0) = x_n, \cdots, x_m
    \end{eqnarray}
\end{definition}

\begin{remark}
    \textcolor{red}{
    Generating a trajectory is the most basic tool for understanding a DDS.
    Plotting the trajectory as a graph is a simple way to visualize the short-term behavior of the system.
    The graph of a trajectory shows, for example, whether the system is periodic, chaotic or intermittent. (Intermittency will be explained later.)
    }
    \par
    \textcolor{red}{
    An example of trajectories of the Logistic map is shown in Figure~\ref{fig:trajectory_example}.
    For each trajectory, projection onto a single line is also provided.
    Understanding how projections relate to the original trajectories is important for understanding other tools for analysing a DDS.
    }
\end{remark}

\begin{figure}[!h]
    \centering
    \begin{subfigure}{0.495\textwidth}
        \centering
        \includegraphics[width=\textwidth]{DDS/Figures/logistic_map_trajectory_example1.png}
        \caption{}
    \end{subfigure}
    \hfill
    \begin{subfigure}{0.495\textwidth}
        \centering
        \includegraphics[width=\textwidth]{DDS/Figures/logistic_map_trajectory_example2.png}
        \caption{}
    \end{subfigure}

    \caption{
        \textcolor{red}{ 
        An example of trajectories and their projections onto a line. 
        Both cases (a) and (b) show a trajectory $T_{50}^{100}(\mathbb{L}_{r}, 0.5)$ (left) and a projection of the trajectory onto a single line (right). 
        Case~(a) exhibits chaotic behavior.
        Case~(b) exhibits $3$-periodic behavior.
        Parameters: 
        (a) $r = 3.828$. 
        (b) $r = 3.829$. 
        }
    }
    \label{fig:trajectory_example}
\end{figure}

\label{def:fixed point}
\begin{definition}[Fixed point]
    Let $\left( \mathbb{R}, f \right)$ be a DDS. Point $x \in X$ is called a \emph{fixed point} of $f$ if it holds $f(x) = x$.
    We denote the set of all fixed points of $f$ as $\emph{Fix}(f)$~\cite{Devaney20211026}.
\end{definition}

\label{def:sfp}
\begin{definition}[Stable fixed point]
    Let $\left(\mathbb{R}, f\right)$ be a DDS, $f:\mathbb{R} \rightarrow \mathbb{R}$. Let $f$ be differentiable on $X$. Fixed point $x \in X$ is \emph{stable} if $|f'(x)| < 1$.
    We call this point \emph{attracting}.
\end{definition}

\label{def:ufp}
\begin{definition}[Unstable fixed point]
    Let $\left(\mathbb{R}, f\right)$ be a DDS, $f:\mathbb{R} \rightarrow \mathbb{R}$. Let $f$ be differentiable on $X$. Fixed point $x \in X$ is \emph{unstable} if $|f'(x)| > 1$.
    We call this point \emph{repelling}.
\end{definition}

\begin{remark}
    \textcolor{red}{
        It is worthy to note what the conditions from Definitions~\ref{def:sfp} and \ref{def:ufp} stand for.
        The point of stability detection is linear approximation, that is replacing the original function by its tangent in the proximity of the fixed point.
        This approach can be generalized to $n$-dimensional system, that is functions $f: \mathbb{R}^{n} \rightarrow \mathbb{R}^{n}$.
        It is done by by evaluating absolute values of eigenvalues of the Jacobian matrix of $f$ at the fixed point.
    }
    \par
    \textcolor{red}{
    Direct evaluation of $\mathbb{L}_{r}(x)=x$ yields Fix$(\mathbb{L}_{r}) = \{ 0, (r-1)/r \}$.
    By evaluating the derivative of $\mathbb{L}_{r}$ at these points, we can determine whether the fixed point is attracting or repelling.
    Hence the fixed point $0$ is attracting if $r \in (0, 1)$ and repelling if $r \in (1, 4)$.
    The fixed point $(r-1)/r$ is attracting if $r \in (1, 3)$ and repelling if $r \in (0, 1) \cup (3, 4) $.
    }
    \par
    \textcolor{red}{
    Figure~\ref{fig:fixed_points_stability_example} displays two fixed points of the Logistic map $\mathbb{L}_{r}$.
    One fixed point is attracting and one is repelling. Test for stability is portrayed graphically.
    }
\end{remark}

\begin{figure}[!h]
    \centering
    \begin{subfigure}{0.495\textwidth}
        \centering
        \includegraphics[width=\textwidth]{DDS/Figures/fixed_points_stability_example{1}.png}
        \caption{}
    \end{subfigure}
    \hfill
    \begin{subfigure}{0.495\textwidth}
        \centering
        \includegraphics[width=\textwidth]{DDS/Figures/fixed_points_stability_example{2}.png}
        \caption{}
    \end{subfigure}

    \caption{
        \textcolor{red}{ 
        Two fixed points of the Logistic map $\mathbb{L}_{r}$ and their stability. 
        The red color represents stability. 
        The red fixed point is attracting.
        The blue color represents instability. 
        The blue fixed point is repelling.
        Parameters: 
        (a) $r = 2.3$. 
        (b) $r = 3.3$. 
        }
    }
    \label{fig:fixed_points_stability_example}
\end{figure}

\begin{definition}[Periodic point]
    Let $\left( \mathbb{R}, f\right)$ be a DDS. Point $x \in X$ is called \emph{$n$-periodic} if $f^{n}(x)=x$.
    We denote the set of all $n$-periodic points of $f$ as $\emph{Per}_{n}(f)$.~\cite{Devaney20211026}
\end{definition}

\begin{definition}[Periodic orbit]
    Let $\left(\mathbb{R}, f\right)$ be a DDS. Let $f$ be differentiable on $X$. For periodic point $x_0$ the set $\{f^{n}(x_0):n \in \mathbb{N}\}$ is called periodic orbit.
    If $x_0$ is attracting, we call periodic orbit stable.
    If $x_0$ is repelling, we call periodic orbit unstable (abbr. UPO).
\end{definition}

\begin{theorem}
    Let $\left(\mathbb{R}, \mathbb{L}_r\right)$ be a DDS. Then there exists at most one stable periodic orbit for each $r$.~\cite[p.~74]{Devaney20211026}
\end{theorem}

\begin{proof}
    Proof can be found in~\cite[p.~74]{Devaney20211026}.
\end{proof}

\label{def:cobweb}
\begin{remark}[Cobweb diagram]
    \textcolor{red}{
    Cobweb diagrams are a visual tools for understanding long term behavior of a DDS.
    They are closely related to the trajectory of a DDS and projection of the trajectory onto a line.
    They are useful, for example, for determining whether a fixed point is attracting or repelling. (These terms will be explained later.)
    }
    \par
    \textcolor{red}{
    An example of cobweb diagrams is given in Figure~\ref{fig:cobweb_diag_example}.
    Projections onto a line are also provided.
    Note the similarity between the Figures~\ref{fig:trajectory_example} and~\ref{fig:cobweb_diag_example}.
    }
    \par
    \textcolor{red}{
    Pseudocode~\ref{cobweb_alg} illustrates how to construct a cobweb diagram algorithmically.
    }
\end{remark}

\begin{figure}[!h]
    \centering
    \begin{subfigure}{0.495\textwidth}
        \centering
        \includegraphics[width=\textwidth]{DDS/Figures/logistic_map_cobweb_diag_example1.png}
        \caption{}
    \end{subfigure}
    \hfill
    \begin{subfigure}{0.495\textwidth}
        \centering
        \includegraphics[width=\textwidth]{DDS/Figures/logistic_map_cobweb_diag_example2.png}
        \caption{}
    \end{subfigure}

    \caption{
        \textcolor{red}{
        Cobweb diagrams of a trajectory $T_{50}^{100}(\mathbb{L}_{r}, 0.5)$ and a projection of the trajectory onto a line. 
        Dashed diagonal line represents the identity function. 
        Parameters: 
        (a) $r = 3.828$. 
        (b) $r = 3.829$. 
        }
        }
    \label{fig:cobweb_diag_example}
\end{figure}


\begin{algorithm}
\caption{Cobweb Diagram Construction}\label{cobweb_alg}
\begin{algorithmic}[1]
\Statex $f \gets$ map
\Statex $x_0 \gets$ initial state
\Statex $p \gets$ parameter of $f$
\Statex $x_{range} \gets$ range of $x$ values
\Statex $n \gets$ number of iterations

\State plot graph of $f_p(x)$ for $x \in x_{range}$
\State plot identity $g(x)=x$ for $x \in x_{range}$
\For{$i$ from $0$ to $n$}
\State plot line from $(f_{p}^{i}(x_0), f_{p}^{i}(x_0))$ to $(f_{p}^{i}(x_0), f_{p}^{i+1}(x_0))$
\State plot line from $(f_{p}^{i}(x_0), f_{p}^{i+1}(x_0))$ to $(f_{p}^{i+1}(x_0), f_{p}^{i+1}(x_0))$
\EndFor

\end{algorithmic}
\end{algorithm}

\label{def:bifurcation}
\begin{definition}[Bifurcation]
    Let $\left( X, f_p \right)$ be a DDS.
    \emph{Bifurcation} occurs when qualiative behavior of $f_p$ changes as parameter $p$ slightly changes.
\end{definition}

\label{def:bifurcation_point}
\begin{definition}[Bifurcation point]
    Let $\left( X, f_p \right)$ be a DDS.
    \emph{Bifurcatoin point} is a value of a parameter $p$ at which bifurcation occurs.
    Qualitative behavior of the system changes when parameter is moved slightly around the bifurcation point.
\end{definition}

% \label{def:saddle_node_bif}
% \begin{definition} \textbf{(Saddle-node bifurcation)} \\
%     To be defined.
% \end{definition}

\label{def: bif_diag} 
\begin{remark}[Bifurcation diagram]
    \textcolor{red}{
    Bifurcation diagram is a tool for global, long-term analysis of a DDS.
    It shows how qualitative behavior of an arbitrary system $f_P(x_0)$ changes for varying parameter P.
    It helps to identify stable regions, chaotic regions and intermittent regions. (Intermittency will be explained later.)
    Bifurcation diagram is especially useful for identifying bifurcations.
    }
    \par
    \textcolor{red}{
    An example of a bifurcation diagram is given in Figure~\ref{fig:bif_diag_example}.
    Notice the similarity with Figures~\ref{fig:trajectory_example} and~\ref{fig:cobweb_diag_example}, especially the projections onto a line.
    Bifurcation diagram is generated as an union of these projections for various parameters.
    Projections fall into one of two categories:
    \begin{itemize}
        \item{A set of isolated points. This case corresponds to the periodic trajectory.}
        \item{A band of points. This is related to more complex phenomena such as chaos or quasi-periodicity.}
    \end{itemize}
    }
    \par
    \textcolor{red}{
    Algorithmic construction of a bifurcation diagram is illustrated in Pseudocode~\ref{bif_diag_alg}.
    }
\end{remark}

\begin{figure}[!h]
    \centering
    \includegraphics[width=0.9\textwidth]{DDS/Figures/logistic_map_bif_diag_example.png}
    \caption{
        \textcolor{red}{
        An example of the bifurcation diagram. 
        Diagram is constructed for Logistic map $\mathbb{L}_{r}$ for parameters $r \in \langle 2.81, 3.87 \rangle$. 
        Parameter interval is sampled to $1000$ points. 
        Trajectories are iterated for $5000$ iterations and last $500$ iterations are plotted.
        }
    }
    \label{fig:bif_diag_example}
\end{figure}

\begin{algorithm}
\caption{Bifurcation Diagram Construction}\label{bif_diag_alg}
\begin{algorithmic}[1]
\Statex $f \gets$ map
\Statex $x_0 \gets$ initial condition
\Statex $p_{range} \gets$ range of parameters for which trajectory will be computed
\Statex $n_{total} \gets$ trajectories will be iterated for $n_{total}$ iterations
\Statex $n_{last} \gets$ $n_{last}$ iterations of a trajectory will be plotted

\For{parameter $p$ in $p_{range}$}
    \State $data \leftarrow T_{n_{total}-n_{last}}^{n_{total}}(f_p, x_0)$
    \For{$y$ in $data$}
        \State plot point $(p, y)$ onto canvas
    \EndFor
\EndFor

\end{algorithmic}
\end{algorithm}

\begin{remark}
    \textcolor{red}{
    In Figure~\ref{fig:bif_diag_example} it is visible that small change of a parameter can cause qualitatively different behavior of the DDS.
    When the qualitative change occurs, we say that bifurcation has occurred.
    An example of a bifurcation point is $r = 1+\sqrt{8} \approx 3.82842$. For this $r$ the $\mathbb{L}_{r}$ changes its behavior from seemingly chaotic to $3$-periodic. 
    Another example of a bifurcation point is $r \approx 3.446$. For this $r$ the $\mathbb{L}_{r}$ changes its behavior from $2$-periodic to $4$-periodic.
    }
\end{remark}
\chapter{Type-I Intermittency}
\label{chap:type-I intermittency}

\textcolor{darkred}{
This Chapter is concerned with the phenomena of intermittency, especially with type-I intermittency.
Firstly discovery and history of intermittency is mentioned.
Then the origion of intermittency is motivated.
Furthermore, the idea of laminar length and charasteristic relation is introduced.
Lastly, the ambiguity of bifurcation diagram caused by intermittency is shown.
}

\section{History}
\textcolor{darkred}{
In 1949, the word intermittency was used in context of turbulent flows by Batchelor and Townsend~\cite{Batchelor19491025}.
They used the term to depict signals that altered between approximately flat periods and and burst ones~\cite{Elaskar2017}.
}
\par
\textcolor{darkred}{
About three decades later, Pomeau and Manneville have studied intermittent transition to turbulence in an experiment with convective fluids~\cite{Pomeau1980}.
During this phenomenon, long periodic behavior was interrupted by chaotic bursts as the experiment control parameter changed.
They have shown that intermittent behavior is present in simple models such as the Lorenz model~\cite{Lorenz2004}.
Furthermore, they classified intermittency into three types based on model's Floquet multipliers.
They called these three types as I, II and III.
}
\par
\textcolor{darkred}{
Theoretical analysis of intermittency is centered around certain statistical properties, especially the reinjection probability density function.
While the standard theory considered this function to be constant~\cite{Dubois1983}, new advances in the field provide more complex explanation~\cite{Elaskar2022}.
In addition, to the three original types of intermittency, many new types have been described since then.
Some of them are V, X, on-off, eyelet and ring~\cite{Elaskar2022}.
}

\section{Motivation}
Intermittency is a behavior during which seemingly periodic phases in the trajectory of DDS are abruptly followed by chaotic bursts.
Periodic phases are called laminar.
Phases with chaotic bursts are called turbulent phases.
\par
\textcolor{darkred}{
To illustrate the concept of intermittency, a new dynamical system is needed.
Let $([0, 1], \mathcal{P}_{\varepsilon})$ be a DDS.
Map $\mathcal{P}_{\varepsilon}(x) = \left[ (1+\varepsilon)x+(1-\varepsilon)x^2 \right] \mod{1}$ is called the Pomeau-Manneville map~\cite{Manneville1980,Datseris2022}.
Intermittency is clearly visible in trajectory of $\mathcal{P}_{\varepsilon}$ diplayed in Figure~\ref{fig:intermittent_trajectory_example}.
Seemingly 2-periodic behavior which lasts for about 1000 iterations is interrupted by short chaotic bursts.
These chaotic bursts soon transform into seemingly 2-periodic behavior.
This pattern repeats.
}

\begin{figure}[!h]
    \centering
    \includegraphics[width=0.95\textwidth]{Figures/type_one_intermittency_example1.png}
    \caption{
        \textcolor{darkred}{
        Trajectory $T^{5500}_{0}(\mathcal{P}_{\varepsilon}, x_0)$ for $x_0 = 0.5$ and $\varepsilon = 4.47458$.
        }
    }
    \label{fig:intermittent_trajectory_example}
\end{figure}

\textcolor{darkred}{
Although there are many types of intermittency, the main focus of this thesis is on type-I intermittency~\cite{Pomeau1980,Bussac1982,DelRio2014}.
Such type of intermittent behavior arises when a DDS $(X, f_{p})$ is about to undergo a saddle-node bifurcation.
This type of bifurcation occurs when the tangent of the graph of $f_{p}$ at point $x^{*}$ is equivalent to the identity line.
Thus, point $x^{*}$ is a fixed point and $f_{p}'(x^{*}) = 1$.
As the parameter $p$ is slightly altered, point $x^{*}$ either splits into two fixed points or vanishes.
}
\par
\textcolor{darkred}{
An illustration of saddle-node bifurcation is shown in Figure~\ref{fig:saddle_node_bifurcation}.
A map $f_{p}(x) = p + x + x^2$ is displayed.
For $p = 0$, there is one single fixed point (red point).
Tangent of $f_{p}$ at the red point is $1$.
When $p$ is slightly increased, the fixed point vanishes.
On the other hand, when $p$ is slightly decreased, two fixed point emerge (blue points).
Note that if $p$ is decreased by a very small value, a narrow passage is formed between $f_{p}$ and the identity line.
Imagine that a cobweb diagram (Remark~\ref{def:cobweb}) is initited with initial condition $x_0$ close to the red point.
In that case it would take many iterations for the point to iterate through the narrow passage.
The narrower the passage is, the more iterations it would take.
}

\begin{figure}[!h]
    \centering
    \includegraphics[width=0.7\textwidth]{Figures/intermittency_i_local_map.png}
    \caption{
        \textcolor{darkred}{
        Map $f_{p}(x) = p + x + x^2$ for different parameters $p$. 
        Parameter $p_B = 0$.
        }
    }
    \label{fig:saddle_node_bifurcation}
\end{figure}

\par
\textcolor{darkred}{
Figure~\ref{fig:intermittent_cobweb_example} illustrates the iteration through a narrow passage.
There, the first laminar phase of Figure~\ref{fig:intermittent_trajectory_example} is displayed through the eyes of a cobweb diagram.
The trajectory in Figure~\ref{fig:intermittent_trajectory_example} starts off with a chaotic burst but soon transforms into a stable behavior.
The reason for that is seen in Figure~\ref{fig:intermittent_cobweb_example}(a), the point got trapped in a narrow passage formed between identity line and the graph.
It takes many iterations for the point to iterate through the passage (Figure~\ref{fig:intermittent_cobweb_example}(b)).
Figure~\ref{fig:intermittent_cobweb_example}(c) shows a subset of a trajectory in Figure~\ref{fig:intermittent_trajectory_example}.
The subset corresponds to the iteration through a narrow passage.
Once the point gets out of the narrow passage, it moves chaotically before it gets reinjected back into the narrow passage.
The second entrance into the narrow passage corresponds to the second laminar phase of Figure~\ref{fig:intermittent_trajectory_example}.
}

\begin{figure}[!h]
    \centering
    \includegraphics[width=0.95\textwidth]{Figures/type_one_intermittency_example2.png}
    \caption{
        \textcolor{darkred}{
        Subfigures (a) and (b):
        Cobweb diagram of 
        (a) $T^{472}_{38}(\mathcal{P}_{\varepsilon}^{2}, x_0)$ and
        (b) $T^{461}_{40}(\mathcal{P}_{\varepsilon}^{2}, x_0)$. 
        Subfigure (c):
        Trajectory $T^{520}_{0}(\mathcal{P}_{\varepsilon}^{2}, x_0)$.
        Parameters: $x_0 = 0.5$ and $\varepsilon = 4.47458$.
        }
    }
    \label{fig:intermittent_cobweb_example}
\end{figure}

\par
\textcolor{darkred}{
Intermittency can be though of as a continuous route to chaos (or from chaos).
Intermittency type-I can be spotted in a bifurcation diagram as a breakpoint or a periodic window.
Such periodic window can be seen in Figure~\ref{fig:bif_diag_example} at $r = 1+\sqrt{8} \approx 3.82842$.
At this parameter, the a non-periodic behavior suddently changed to a $3$-periodic behavior.
The process which caused this change is type-I intermittency.
When parameter $r$ gets closer and closer to the breakpoint, laminar phases of the trajectories get longer and longer until the whole trajectory is one infinitely long laminar phase.
}


\section{Laminar Phase Length}

\textcolor{darkred}{
This section describes the notion of laminar phase length, which is the number of iterations a point spends in the narrow passage.
This notion and especially the notion of the average laminar phase length (or average laminar length for short) will be used in the next Chapter.
}
\par
\textcolor{darkred}{
In the following text, the narrow passage from the last section shall be called \emph{laminar region} or \emph{laminar interval}.
Region outside of the laminar region shall be called \emph{turbulent region}.
}
\par
\textcolor{darkred}{
The local shape of the graph above (or below) the laminar region determines the type of intermittency.
The local shape is also called the \emph{local map}.
For maps with type-I intermittency, the local map is the following:
\begin{equation}
\varepsilon + x + a x^2 \label{eq:int_I_local_map}
\end{equation}
However, a correct local map is not enough for intermittency to occur in a DDS.
A reinjection mechanism, which reinjects the point from the turbulent region back into the laminar region, has to be present.
This reinjection mechanism is described by the reinjection probability density function and depends on a specific DDS.
}
\par
\textcolor{darkred}{
Let $(X, f_{p})$ be a DDS. Suppose that $f_{p}$ undergoes saddle-node bifurcation and $x^{*}$ is the corresponding single fixed point.
Suppose that $I_l = [ x^{*}-c, x^{*}+c ]$ is the laminar interval for some small $c$.
A \emph{pre-reinjection point} $x_p$ is a point which gets \emph{reinjected} into $I_l$, which means that there exists a point $x_r = f(x_p)$, $x_r \in I_l$.
The point $x_r$ shall be called a \emph{reinjected point}.
}
\par
The Figure~\ref{fig:pre_reinjection_example} illustrates the idea of pre-reinjection points and laminar interval $I_l$.
Note that the curve in subfigure (b) locally resembles the local map \eqref{eq:int_I_local_map} around the point $x^{*}$.
\begin{figure}[!h]
    \centering
    \begin{subfigure}{0.49\textwidth}
        \centering
        \includegraphics[width=\textwidth]{Figures/pre_reinjection_points_example{1}.png}
        \caption{}
    \end{subfigure}
    \hfill
    \begin{subfigure}{0.49\textwidth}
        \centering
        \includegraphics[width=\textwidth]{Figures/pre_reinjection_points_example{2}.png}
        \caption{}
    \end{subfigure}

    \caption{
        \textcolor{darkred}{ 
        Pre-reinjection points and laminar interval of $\mathcal{L}_{r}^{3}(x+x^{*})-x^{*}$ for $r = 3.827$ and $x^{*} \approxeq 0.51435$.
        (a) the full graph. 
        (b) a close-up to a region of interest. 
        Red dots: pre-reinjection points.
        Blue dots: reinjected points.
        Laminar interval: $I_l = [ x^{*}-c, x^{*}+c ]$ for $c = 0.03$.
        Orange line: $I_l$.
        Green dot: $x^{*}$.
        }
    }
    \label{fig:pre_reinjection_example}
\end{figure}
\par
Consider that $0 < \varepsilon \ll 1$.
By using the Equation~\eqref{eq:int_I_local_map} the difference between two successive points $x_n - x_{n-1} \approx dx/dl$ can be written as:
\begin{equation}
\frac{dx}{dl} = \varepsilon + a x^2 \label{eq:int_I_diff_eq}
\end{equation}
Thus continuous differential equation is obtained.
Solving from reinjected point $x$ to the end of laminar interval $c$ yields:
\begin{equation}
\int_{x}^{c} \frac{dx}{\varepsilon + a x^2} = \int_{0}^{l} dl \label{eq:int_I_diff_eq_step1}
\end{equation}
Solving for $l$ results in:
\begin{equation}
    l(x, c) = \frac{1}{\sqrt{a \varepsilon}} \left( \tan^{-1} \left( c \sqrt{\frac{a}{\varepsilon}} \right) - \tan^{-1} \left( x \sqrt{\frac{a}{\varepsilon}} \right) \right) \label{eq:laminar_length}
\end{equation}

\par
Hence for a given reinjected point $x$ and laminar interval width $c$ the laminar length $l$ can be calculated.
Laminar length means the number of iterations the point has to make to the end of laminar interval.
By calculating laminar length for many reinjected points and some fixed $c$ the average laminar length $l_{avg}$ can be calculated.
However, that is not very practical since it is not always clear how to choose $c$.
Fortunately, this problem was solved by formulation of so-called characteristic relation of type-I intermittency.

\section{Characteristic Relation}
To estimate the average laminar length $l_{avg}$ based on the parameter $\varepsilon$ a characteristic relation was formulated~\cite{Elaskar2017}.
It holds that $l_{avg} \propto 1/\sqrt{\varepsilon}$.
This relation states that the average laminar length $l_{avg}$ is inversely proportional to the square root of the parameter $\varepsilon$.
Thus, by calculating $\varepsilon$ of the local map in the vicinity of a fixed point $x^{*}$ the $l_{avg}$ can be estimated.
\par
To verify that the characteristic relation of type-I intermittency holds, a numerical simulation was conducted.
The relation is verified for the third iterate of the Logistic map $\mathcal{L}_{r}^{3}$.
\par
\textcolor{darkred}{
The third iterate of Logistic map $\mathcal{L}_{r}^{3}$ undergoes saddle-node bifurcation at $r = r^{*} := 1+\sqrt{8}$~\cite{Elaskar2022,Gordon20180411}.
Parameter $r^{*}$ is the start of the $3$-periodic window, the biggest periodic window visible in the bifurcation diagram of $\mathcal{L}_{r}$ (see Figure~\ref{fig:bif_diag_example}).
There are three fixed points of $\mathcal{L}_{r^{*}}$ associated with the bifurcation, $x^{*}_{1} \approx 0.1599$, $x^{*}_{2} \approx 0.5143$ and $x^{*}_{3} \approx 0.9563$.
For parameter $r$ slightly less than $r^{*}$, $\mathcal{L}_{r}^{3}$ in the proximity of each of the points $x^{*}_{1}$, $x^{*}_{2}$ and $x^{*}_{3}$ resembles the local map \eqref{eq:int_I_local_map}.
Figure~\ref{fig:pre_reinjection_example}~(b) shows the local map of $\mathcal{L}_{r}^{3}$ in the proximity of $x^{*}_{2}$.
The following explanation of the simulation will use $x^{*}_{1}$ as an example.
The simulation can be conducted in the same way for $x^{*}_{2}$ and $x^{*}_{3}$.
}
\par
The goal is to compute the average laminar length $l_{avg}$ for a point traveling through the laminar region around $x^{*}_{1}$.
The laminar region exists for the parameters $r$ slightly less than $r^{*}$.
In the experiment $30$ parameters $r$ were randomly chosen from the interval $[ r^{*}-10^{-5}, r^{*}-10^{-13} ]$.
Subsequently, the corresponding $\varepsilon$ parameters were calculated.
\par
To calculate $\varepsilon$ for the local map around $x^{*}_{1}$, the map $\mathcal{L}_{r}^{3}$ was shifted so that $x^{*}_{1}$ is in the origin.
In other words, $\mathcal{L}_{shift}(x) = \mathcal{L}_{r}^{3}(x + x^{*}_{1}) - x^{*}_{1}$.
Parameters $\varepsilon$ and $a$ are calculated using the Taylor polynomial of $\mathcal{L}_{shift}(x)$ at $x = 0$.
Hence, $\varepsilon = | \mathcal{L}_{shift}(0) |$ and $a = \mathcal{L}_{shift}''(0) / 2$.
$\varepsilon$ and $a$ are calculated for each of the $30$ random $r$ parameters.
Next, they are used to calculate the laminar length.
\par
To calculate the laminar length $l$ from \eqref{eq:laminar_length} corresponding to one parameter $r$, reinjection point $x$, parameters $c$, $\varepsilon$ and $a$ are needed.
The laminar interval width $c$ in the experiment was selected as $0.01, 0.03, 0.05$ and $0.07$.
In the following experiment, $c = 0.01$.
The same procedure can be done for the remaining $c$'s.
Parameters $c$, $\varepsilon$ and $a$ are known.
Next, reinjection points $x$ are calculated.
\par
To calculate the reinjection points corresponding to one of the $30$ $r$'s, the following procedure is used.
A point $x$ is randomly chosen from $[ 0-x^{*}_{1}, 1-x^{*}_{1} ]$.
Consequently, a check is conducted if the point gets reinjected into laminar interval.
In other words, check if $| \mathcal{L}_{shift}(x) | < c$.
The procedure is repeated until $N = 4000$ reinjected points are obtained.
\par
For each reinjected point $x$, the laminar length $l$ is calculated using \eqref{eq:laminar_length}.
Then, their average is calculated.
\par
The procedure is repeated for each of the $30$ $r$'s.
The results are plotted in log-log scale.
A linear fit of the results is calculated and the results are shown in Figure~\ref{fig:characteristic_relation_fit}.
Note that the results support that the $l_{avg} \propto 1/\sqrt(\varepsilon) = \varepsilon ^ {-1/2}$.
The linear fit for different $c$'s and $x^{*}$ is shown.
\par
Also note that it holds $l_{avg} \propto \varepsilon ^ {-1/2} \implies log_{10}(l_{avg}) \propto -1/2 \cdot \log_{10}(\varepsilon)$.
The Figure~\ref{fig:characteristic_relation_fit} shows that the slope of the linear fits is approximately $-1/2$.
That implies that the characteristic relation of type-I intermittency holds. 

\begin{figure}[!h]
    \centering
    \begin{subfigure}{0.49\textwidth}
        \centering
        \includegraphics[width=\textwidth]{Figures/characteristic_relation_fit{1}.png}
        \caption{}
    \end{subfigure}
    \hfill
    \begin{subfigure}{0.49\textwidth}
        \centering
        \includegraphics[width=\textwidth]{Figures/characteristic_relation_fit{2}.png}
        \caption{}
    \end{subfigure}
    \hfill
    \begin{subfigure}{0.49\textwidth}
        \centering
        \includegraphics[width=\textwidth]{Figures/characteristic_relation_fit{3}.png}
        \caption{}
    \end{subfigure}

    \caption{
        \textcolor{darkred}{ 
        $x$-axis: $30$ random $r \in [ r^{*}-10^{-5}, r^{*}-10^{-13} ], r^{*} = 1 + \sqrt{8}$, selected, corresponding $\varepsilon$ are calculated.
        $y$-axis: average $l_{avg}$ of laminar lengths (Equation~\eqref{eq:laminar_length}) for $4000$ points reinjected into laminar interval $I_{c} = [x^{*}-c, x^{*}+c], c \in \{ 0.01, 0.03, 0.05, 0.07 \}$ is calculated.
        Colorful lines: linear fit.
        Constant $s$: slope of the linear fit.
        Fixed point $x^{*}$: (a) $x^{*} = 0.1599$ (b) $x^{*} = 0.5143$ and (c) $x^{*} = 0.9563$.
        DDS: $\mathcal{L}_{r}^{3}$.
        }
    }
    \label{fig:characteristic_relation_fit}
\end{figure}

\section{Ambiguous Bifurcation Diagram}
\label{sec:ambiguous_bif_diag}
How does the bifurcation diagram look for an intermittent region?
During the creation of a bifurcation diagram, some fixed number of iterations are plotted.
Based on the number of total iterations and number of last iterations being plotted, the bifurcation diagram may differ.
When plotting a subsection of the trajectory that corresponds only to the laminar phase, the behavior during the turbulent phase is omited.
This causes the bifurcation diagram to look different depending on how many iterations are made and how long the sample period is.
\par
\begin{figure}[!h]
    \centering
    \includegraphics[width=0.95\textwidth]{Figures/bif_diag_ambiguity_creation_sketch.png}
    \caption{
        \textcolor{darkred}{
        Sketch describing how to create an ambiguous bifurcation diagram. 
        Red graph: $T^{17000}_{0}(\mathcal{P}_{4.47458285}, x_0)$. 
        Blue graph: $T^{17000}_{0}(\mathcal{P}_{4.47458285+4 \cdot 10^{-8}}, x_0)$. 
        Initial condition: $x_0 = 0.5$.
        }
    }
    \label{fig:ambiguous_bif_diag}
\end{figure}

To illustrate this ambiguity in bifurcation diagram, one can simply create a generic example.
Figure~\ref{fig:ambiguous_bif_diag} shows two trajectories for two distinct parameters $p_1$ and $p_2$.
Both are close to the breakpoint $p_{b}$ and it holds $p_1 < p_2 < p_{b}$.
The trajectory for parameter $p_1$ is plotted in red in Figure~\ref{fig:ambiguous_bif_diag}.
The trajectory for $p_2$ is blue.
Since $p_2$ is closer to the intermittency threshold $p_{b}$, the laminar phases of its trajectories are longer.
% Parameters $p_1$ and $p_2$ can be selected arbitrarily as long as theso that the laminar phases are long enough and similar situation as in Figure~\ref{fig:ambiguous_bif_diag} occurs.
Chaotic burst of the trajectory with a longer laminar phase occurs in the middle of two laminar phases of the trajectory with shorter laminar phases.
In our case, the chaotic burst $b$ occurs between chaotic bursts $a$ and $c$.
When creating two bifurcation diagrams with total number of iterations $0.95 c$ and plotting the last $c-(b-0.25(b-a))$ or $c-(b-0.75(b-a))$ iterations, two distinct bifurcation diagrams are obtained.

\begin{figure}[!h]
    \centering
    \begin{subfigure}{0.85\textwidth}
        \centering
        \includegraphics[width=\textwidth]{Figures/pomeau_manneville_bif_comparison_big{1}.png}
        \caption{}
    \end{subfigure}
    \hfill
    \begin{subfigure}{0.85\textwidth}
        \centering
        \includegraphics[width=\textwidth]{Figures/pomeau_manneville_bif_comparison_big{2}.png}
        \caption{}
    \end{subfigure}
    \begin{subfigure}{0.85\textwidth}
        \centering
        \includegraphics[width=\textwidth]{Figures/pomeau_manneville_bif_comparison_big{3}.png}
        \caption{}
    \end{subfigure}
    \begin{subfigure}{0.85\textwidth}
        \centering
        \includegraphics[width=\textwidth]{Figures/pomeau_manneville_bif_comparison_big{4}.png}
        \caption{}
    \end{subfigure}

    \caption{
        \textcolor{darkred}{ 
        Bifurcation diagrams of $\mathcal{P}_{\varepsilon}$ for $\varepsilon \in I := [4.4745829085, 4.4745829215]$.
        Interval $I$ is sampled to $400$ uniformly spaced points.
        Trajectories used for diagram construction:
        (a) $\mathcal{T}_{12000}^{25000}(\mathcal{P}_{\varepsilon}, x_0)$
        (b) $\mathcal{T}_{13900}^{25000}(\mathcal{P}_{\varepsilon}, x_0)$ 
        (c) $\mathcal{T}_{20000}^{25000}(\mathcal{P}_{\varepsilon}, x_0)$ 
        (d) $\mathcal{T}_{24000}^{25000}(\mathcal{P}_{\varepsilon}, x_0)$.
        Initial condition: $x_0 = 0.5$.
        }
    }
    \label{fig:ambiguous_bif_diag_example}
\end{figure}

This is illustrated in Figure~\ref{fig:ambiguous_bif_diag_example}, which shows that bifurcation diagrams differ significantly when the number of total iterations is the same, but the number of last iterations being plotted is different.
\par
A question arises what do we expect from a bifurcation diagram when we create it?
 Figure~\ref{fig:ambiguous_bif_diag_example} clearly shows that the same region of the bifurcation diagram can look very different.
An observer not knowing about intermittency could be misled.
It would be useful to have a tool that would warn a researcher looking at the bifurcation diagram about underlying intermittency and ambiguity it is causing.
The next chapter introduces such a tool.

\endinput
\chapter{Intermittency Detection}
\label{chapter:intdetection}
Section \ref{sec:ambiguous_bif_diag} of the previous chapter introduced the problem of ambiguity in the bifurcation diagram.
\textcolor{red}{This chapter develops a novel algorithm that aims to detect and highlight the ambiguous regions.}
The algorithm consists of 3 core parts:

\begin{enumerate}
	\item \textbf{Global Search} -- searching for areas where breakpoints occur.
	\item \textbf{Local Search} -- searching for precise locations of the breakpoints.
	\item \textbf{Coloring} -- coloring the bifurcation diagram in the proximity of the breakpoints.
\end{enumerate}

Each of these parts are described in their own sections.
The last section of this chapter combines all the parts together.

\section{Global Search}
\label{sec:globsearch}
The initial phase of the algorithm, which shall be called the Global Search, aims to find approximate parameter intervals in which type-I intermittency occurs.
The algorithm searches through the whole parameter space globally to find these intervals.
More specifically, the search for type-I intermittency regions is equivalent to a search for the breakpoints in the bifurcation diagram.
Global Search consists of two parts, Naive Global Search and detection of periodic points.
These two parts are described in the following sections and then combined together.
\subsection{Naive Global Search}

A breakpoint is a parameter value at which the bifurcation diagram transitions from nonperiodic behavior to periodic behavior.
Such breakpoint can be seen in Figure~\ref{fig:break_point_search_example}.
The figure shows that there is nonperiodic behavior to the left of the breakpoint and periodic behavior to the right of it.
With this knowledge, each breakpoint can be identified by a unique number $n$ which corresponds to $n$ periodic behavior to the right of it.
If the parameter space could be searched through and the period of the system for each parameter value could be determined, the breakpoints could easily be found.
To do that, an algorithm is needed to determine the period of the system for any given parameter value.
\par
Let $(X, f_{p})$ be a DDS. Suppose a bifurcation diagram of $f_{p}$ is constructed for some $x_0$, $n_{total}$, $n_{last}$, and $p_{range}$ (see Algorithm~\ref{alg:bif_diag}).
The fact that the bifurcation diagram shows $n$-periodic behavior ($n$ dots) for a parameter $p$ implies that map $f_{p}$ is $n$-periodic or eventually $n$-periodic.
Two situations can happen:
\begin{enumerate}
    \item map $f_{p}$ has a stable periodic orbit and the $x_0$ has converged to it in $n_{total}-n_{last}$ iterations,
    \item the initial condition $x_0$ belongs to a stable or unstable periodic orbit of $f_p$.
\end{enumerate}
When talking about periodic behavior in a bifurcation diagram, the focus will be on case $1$.
Case $2$. will be ignored since it is not very probable that an initial condition will belong to a periodic orbit.
\par
Theorem~\ref{theorem:single_orbit} states that the Logistic map $\mathcal{L}_r$ can only have at most one unique stable $n$-periodic orbit for each parameter $r$.
Therefore, if the bifurcation diagram exhibits $n$-periodic behavior at $r$, it can be concluded that $\mathcal{L}_{r}$ has a $n$-periodic SPO.
Consequently, finding the smallest SPO of $\mathcal{L}_{r}$ would indicate how many dots there are in the bifurcation diagram at $r$.
\par
There exist algorithms to find the periodic points of an arbitrary DDS.
They will be discussed in detail in Subsection~\ref{subsec: detection_of_periodic_points}.
Although they are efficient, they are not efficient enough for the current goal.
The aim is to search the entire parameter space $p_{range}$ and try to detect an SPO for each parameter $p \in p_{range}$.
\par
A naive approach is to use \emph{brute-force approach}~\cite{Parker1989} or equivalently find \emph{pseudo periodic orbits}~\cite{Galias2023}.
The initial condition $x_0$ is iterated $m$ times $x_{m} = f^{m}_{p}(x_0)$, where $m = n_{total}-n_{last}$.
Then, it is checked if $|f^{n}_{p}(x_m) - x_m| \leq \delta$, where $\delta$ is some small tolerance.
If $n$ is the smallest number for which the inequality holds, then $x_m$ can be considered a $n$-periodic point.
Consequently, the bifurcation diagram has $n$ dots at $p$.
If the inequality is checked for $n = 1,2,\dots,o$ in ascending order, then it is clear which $n$ is the smallest.
The number $o$ denotes the maximal period which is checked.
\par
The procedure that was just described shall be called a Naive Global Search (NGS).
NGS is a quick way to estimate the periodicity in the bifurcation diagram for any parameter $p$.
The pseudocode for this procedure is presented in Algorithm~\ref{alg:naive_global_search}.
The result of NGS is portrayed graphically in Figure~\ref{fig:naive_global_search}.
There, the points that were identified by NGS as periodic are colored blue.
\par
After applying NGS, the breakpoint can be easily found.
It can be concluded that the breakpoint occurs between neighboring parameters $[p_{A}, p_{B}]$ such that NGS found no periodicity for $p_{A}$ and $n$-periodicity for $p_{B}$.
However, the shortcoming of NGS is that it is misleading in the proximity of a breakpoint.
Imagine that $f^{m}_{p}(x_0)$ for $m = n_{total}-n_{last}$ lies in the laminar phase of an intermittent trajectory of $f_{p}$.
Then $p$ will be determined as periodic even if it is not.
% In addition, the bifurcation diagram at $p$ might or might not show $n$ dots.
Since $\mathcal{T}_{m}^{n_{total}}(f_{p}, x_{0})$ is intermittent, the bifurcation diagram could show a band of points or just $n$ dots.
That depends on whether the trajectory subset is in the laminar or turbulent phase.
\par
To overcome this issue, it is useful to verify that the period of $p_{A}$ and $p_{B}$ was correctly estimated.
The next subsection introduces efficient algorithms for computing the periodic points of an arbitrary DDS.
By checking how many periodic points are stable for $f_{p_{A}}$ and $f_{p_{B}}$, the period of $p_{A}$ and $p_{B}$ can be verified.

\begin{algorithm}[!h]
    \caption{Naive Global Search}
    \label{alg:naive_global_search}
    \begin{algorithmic}[1]
        \Statex $f \gets$ map
        \Statex $p_{A} \gets$ left boundary of the parameter space
        \Statex $p_{B} \gets$ right boundary of the parameter space
        \Statex $n \gets$ number of samples in the parameter space
        \Statex $m \gets$ number of iterations
        \Statex $o \gets$ maximum period to check
        \Statex $\delta \gets$ tolerance
        \Statex $x_0 \gets$ initial condition
        \State $p \gets p_{A}$
        \State $p_{step} \gets \frac{p_{B} - p_{A}}{n}$
        \While{$p \leq p_{B}$}
            \State $x_m \gets f_{p}^{m}(x_0)$
            \For{$i \gets 1$ to $o$}
                \State $x \gets f_{p}^{i}(x_m)$
                \If{$|x_m - x| \leq \delta$}
                    \State $f_p$ is $i$-periodic
                \EndIf
            \EndFor
            \State $p \gets p + p_{step}$
        \EndWhile
    \end{algorithmic}
\end{algorithm}

\begin{figure}[!h]
    \centering
    \includegraphics[width=0.95\textwidth]{Figures/naive_global_search.png}
    \caption{
        NGS of $\mathcal{L}_{r}$.
        Grey points: bifurcation diagram constructed of trajectories $\mathcal{T}_{3500}^{4000}(\mathcal{L}_{r}, 0.5)$ for $r \in [3.55, 3.75]$.
        Blue points: points identified as periodic. 
        Parameters of Algorithm~\ref{alg:naive_global_search}: $f = \mathcal{L}_{r}$, $p_{A} = 3.55$, $p_{B} = 3.75$, $n=5000$, $m=1800$, $o=40$ and $x_0 = 0.5$.
    }
    \label{fig:naive_global_search}
\end{figure}

\subsection{Detection of Periodic Points}
\label{subsec: detection_of_periodic_points}

Detecting periodic points numerically can be approached in a straightforward manner.
An $n$-periodic point is a fixed point of the $n$-th iterate of a map.
Hence, finding roots of $f^{n}_{p}(x)-x = 0$ yields all periodic points $x$ of $n$-th iteration of map $f_{p}$.
Standard root-finding algorithms such as Newton-Raphson~\cite{Haeseler1988} can be applied to the function $g(x) = f^{n}(x)-x$ to find its roots.
Newton-Raphson algorithm converges to a single root. To ensure that all roots were found, the search space can be sampled to $N$ seeds.
Seeds can be created, for example, as a uniform grid of the search space.
Detecting roots when $n$ is relatively low is successful with proper seeding; however, as $n$ increases, the number of roots increases exponentially \cite{Davidchack1999}.
As the number of roots increases, the need for finer and finer seeding arises.
With the increasing number of seeds, the algorithm becomes computationally very expensive.
Another issue with Newton-Raphson is that the basins of convergence to each respective root are not very big \cite{Davidchack1999}.
\par
To overcome the issues with standard root-finding algorithms, researchers came up with new strategies to detect periodic points.
The central idea of the new approach is to use stabilizing transformations.
A set of transformations is proposed so that these transformations stabilize some of the fixed points so that they become attractive.
Afterwards, an iterative scheme is introduced, which converges to these fixed points.
Schmelcher and Diakonos~\cite{Schmelcher1997,Pingel2000, Pingel2001} were the first to come up with the idea of stabilizing transformations.
Their algorithm is globally convergent and detects all periodic points of low periods.
The main problem with their approach is that the number of stabilizing transformations grows rapidly with increasing dimension of the dynamical system.
Another issue is that the algorithm relies on fine seeding in the same way that Newton-Raphson does, although it usually needs much less seeds.
\par
The algorithm of Schmelcher and Diakonos was subsequently improved by Davidchack and Lai~\cite{Davidchack1999, Davidchack2001, Klebanoff2001}.
They introduced two improvements: a smarter seeding procedure and an improved iterative scheme with enhanced convergence speed.
Through their modifications, they were able to detect periodic points of higher periods than was previously possible.
However, their method is still not very applicable to high dimensional dynamical systems.
\par
Davidchack's and Lai's algorithm was futher improved by Davidchack's PhD student Crofts~\cite{Crofts2005,Crofts2007,Crofts2008,Crofts20090901}.
His version proposes a smaller set of stabilizing transformations such that the method can be used for high dimensional dynamical systems.
\par
Another approach was proposed by Bu-Wang-Jiang (BWJ)~\cite{Bu2004}.
This approach is not based on stabilizing transformation, but on an iterative scheme together with fine seeding.
\par
For the purposes of the Global Search, Bu-Wang-Jiang is chosen for the detection of periodic points.
It is easy to implement and works well enough.
The pseudocode for this algorithm is presented in Algorithm~\ref{alg:bwj}.

\begin{figure}[!h]
    \centering
    \includegraphics[width=0.95\textwidth]{Figures/upo_search_example.png}
    \caption{
        Detection of the fixed point of $\mathcal{L}_{r}^{6}$ for $r = 3.7$ using the Davidchack-Lai algorithm.
    }
    \label{fig:upo_search_example}
\end{figure}

\begin{algorithm}[!h]
    \caption{Bu-Wang-Jiang (BWJ)}
    \label{alg:bwj}
    \begin{algorithmic}[1]
        \Statex $f \gets$ map
        \Statex $p \gets$ period
        \Statex $seeds \gets$ seeds
        \Statex $maxiter \gets$ maximum number of iterations
        \Statex $tol \gets$ tolerance for determining a fixed point
        \For{each s in seeds}
            \State $\textbf{x}_{0} \gets s$
            \While{current iteration $< maxiter$}
                \State $J(\textbf{x}_{0}) = \partial f^{p}(\textbf{x}_{0}) / \partial \textbf{x}$ is the Jacobian at $\textbf{x}_{0}$
                \State $Q \gets (cI-J(\textbf{x}_{0}))(J(\textbf{x}_{0})-I)^{-1}$ where $I$ is identity matrix, $c \in (-1, 1)$ is a constant 
                \State $\textbf{x}_1 \gets f^{p}(\textbf{x}_{0}) + Q(f(\textbf{x}_{0})^{p}-\textbf{x}_{0})$
                \If{$\norm{f^{p}(\textbf{x}_1)-\textbf{x}_1} < tol$}
                    \State $\textbf{x}_{1}$ is a fixed point of $f^{p}$
                \EndIf
                \State $\textbf{x}_{0} \gets \textbf{x}_{1}$
            \EndWhile
        \EndFor
    \end{algorithmic}
\end{algorithm}

Note that the $seeds$ parameter in Algorithm~\ref{alg:bwj} is a set of initial conditions from which the algorithm starts.
This parameter can be generated by uniformly sampling the search space.
In case of $\mathcal{L}_{r}$, the interval $[0, 1]$ can be uniformly sampled to $N$ points.

\bigskip
This section has introduced an algorithm to search for breakpoints in the parameter space.
First, a discretization of the parameter space is performed and the NGS is used to approximate the periodicity of each parameter.
Afterwards, neighboring pairs of parameters $p_{A}$ and $p_{B}$ such that $p_{A}$ is nonperiodic and $p_{B}$ is periodic are chosen.
BWJ algorithm is used to verify that the periodicity of $p_{A}$ and $p_{B}$ is correct.
If so, it is concluded that there is a breakpoint in a bifurcation diagram somewhere in the interval $[p_{A}, p_{B}]$.
At the end of the algorithm, approximate intervals $[p_{A}, p_{B}]$ and periods of $p_{B}$ are known for each breakpoint.
This algorithm is called the Global Search.
The approximate locations of the breakpoints evaluated by this algorithm are shown in Figure~\ref{fig:bif_diag_search_example}.

\begin{figure}[!h]
    \centering
    \includegraphics[width=0.95\textwidth]{Figures/bif_diag_search_example.png}
    \caption{
        Global Search of $\mathcal{L}_{r}$.
        The bifurcation diagram consists of projections of $\mathcal{T}_{900}^{1000}(\mathcal{L}_{r}, 0.5)$.
        Red lines: detected breakpoints up to period $16$.
    }
    \label{fig:bif_diag_search_example}
\end{figure}

\section{Local Search}

The previous section introduced a so-called Global Search, an approach to find approximate intervals in the parameter space where a breakpoint occurs.
This section introduces algorithms to find the exact location of each breakpoint.
These locations will later be used to color the bifurcation diagram in the proximity of the breakpoint.
\par
This section discusses two algorithms for the Local Search - Naive Local Search and Nested-Layer Particle Swarm Optimization.
The Global Search has already identified two boundaries $[p_A, p_B]$, the period of the system at parameter $p_B$ and the fact that there is a breakpoint between $[p_A, p_B]$.
Subsequently, a better approximation of the parameter at which the breakpoint occurs is needed.
The two algorithms are able to approximate the parameter using information from the Global Search.
Both algorithms solve the problem, but they use different approaches.

\subsection{Naive Local Search}
\label{subsec:naive_local_search}

Global Search identified intervals that contain a breakpoint.
Naive Local Search (NLS) uses these intervals to find precise locations of the breakpoints.
To achieve that, it uses the Bisection method~\cite{Burden2015-lp} combined with the BWJ algorithm.
\par
Suppose that $[p_{A}, p_{B}]$ is the interval identified during the Global Search.
Furthermore, it is known that $f_{p_{A}}$ is non-periodic and $f_{p_{B}}$ is $n$-periodic.
Next, split the interval $[p_{A}, p_{B}]$ to obtain a midpoint $p_{C} = (p_{A}+p_{B})/2$.
If $p_{C}$ is $n$-periodic then it is known that the breakpoint is in the interval $[p_{A}, p_{C}]$.
If $p_{C}$ is nonperiodic, then it is known that the breakpoint is in the interval $[p_{C}, p_{B}]$.
Whether a parameter is $n$-periodic or not is determined using the BWJ algorithm.
The new interval is halved again.
The halving process is repeated for some desired number of iterations $maxiter$.
The pseudocode for the NLS of is given in Algorithm~\ref{alg:local_search}.

\begin{algorithm}[!h]
    \caption{NLS}
    \label{alg:local_search}
    \begin{algorithmic}[1]
        \Statex $f \gets$ map
        \Statex $p_{A} \gets$ left boundary of the parameter space
        \Statex $p_{B} \gets$ right boundary of the parameter space
        \Statex $n \gets$ period of $f{p_{B}}$
        \Statex $maxiter \gets$ number of iterations
        \For{$i \leq maxiter$}
            \State $p_{C} \gets \frac{p_{A}+p_{B}}{2}$
            \If{$f_{p_{C}}$ is $n$-periodic}
                \State $p_{B} \gets p_{C}$
            \Else
                \State $p_{A} \gets p_{C}$
            \EndIf
        \EndFor
        \State \textbf{return} $[p_{A}, p_{B}]$
    \end{algorithmic}
\end{algorithm}

\par
The Algorithm~\ref{alg:local_search} returns a small interval $[p_{A}, p_{B}]$ which contains the breakpoint.
By calculating the average of the left and right boundary, a good approximation of the parameter at which the breakpoint occurs is obtained.
Precision can be controlled by the number of iterations $maxiter$.
An example of NLS for $\mathcal{L}_{r}$ is shown in Figure~\ref{fig:break_point_search_example}.

\begin{figure}[!h]
    \centering
    \includegraphics[width=0.95\textwidth]{Figures/break_point_search_example.png}
    \caption{
        NLS with parameters $f = \mathcal{L}_{r}$, $p_A = 3.825$, $p_B = 3.83$, $n = 3$ and $maxiter = 20$.
        The bifurcation diagram consists of projections of $\mathcal{T}_{900}^{1000}(\mathcal{L}_{r}, 0.5)$.
        Red line: estimate of the period $3$ breakpoint.
    }
    \label{fig:break_point_search_example}
\end{figure}


\subsection{Nested-Layer Particle Swarm Optimization}
The NLS introduced in the previous subsection approximates the parameter at which the breakpoint occurs.
It is based on the idea of the Bisection method.
However, the search for the breakpoint can be rephrased as a search for a parameter where saddle-node bifurcation occurs.
Matsushita, Kurokawa, and Kousaka~\cite{Matsushita2019} introduced an approach to search for saddle-node bifurcation of a DDS.
Their method uses the Nested-Layer Particle Swarm Optimization (NLPSO) method.
In addition, their approach can be used for the detection of other types of bifurcations~\cite{Matsushita20170721}.
\par
Understanding how NLPSO detection of saddle-node bifurcation works requires understanding of the Particle Swarm Optimization (PSO).
PSO is a popular population-based evolutionary algorithm.
It tracks several particles that represent potential solutions.
Each particle has its own position $pos \in \mathbb{R}^{n}$ and velocity $vel \in \mathbb{R}^{n}$. It also tracks its previous best position $b_{pos} \in \mathbb{R}^{n}$ and score $b_{score} \in \mathbb{R}$, which is evaluated by a cost function.
Each particle moves through the search space based on its velocity and position.
Its movement is also influenced by its previous best position, the best positions of other particles $g_{pos}$ and the global best score $g_{score}$.~\cite{Matsushita2019}
\par
% Each particle has a position $pos \in \mathbb{R}^{n}$, a velocity $vel \in \mathbb{R}^{n}$, a best position $b_{pos} \in \mathbb{R}^{n}$ and a best score $b_{score} \in \mathbb{R}$.
% The algorithm also tracks the best global position $g_{pos}$ and global best score $g_{score}$.
The pseudocode for the PSO algorithm is given in Algorithm~\ref{alg:pso}.
The parameters $w$, $c_{1}$ and $c_{2}$ in Algorithm~\ref{alg:pso} shall be set as $w=0.729$ and $c_{1}=c_{2}=1.494$.~\cite{Matsushita2019}

\begin{algorithm}[!h]
    \caption{Particle Swarm Optimization (PSO)}
    \label{alg:pso}
    \begin{algorithmic}[1]
        \Statex $f \gets$ function to minimize, $f: \mathbb{R}^{m} \rightarrow \mathbb{R}$.
        \Statex $(a, b) \gets$ search-space range
        \Statex $n \gets$ number of particles
        \Statex $maxiter \gets$ maximum number of iterations
        \Statex $tol \gets$ tolerance for determining solution
        \Statex $w, c_{1}, c_{2} \gets$ parameters described in the text
        \State Create $n$ particles.
        \For{each particle}
            \State $pos \gets$ $m$-dimensional vector of uniform random numbers between $a$ and $b$.
            \State $vel \gets$ $m$-dimensional zero vector.
            \State $b_{pos} \gets$ $m$-dimensional vector of uniform random numbers between $a$ and $b$.
            \State $b_{score} \gets \infty$
        \EndFor
        \State $g_{pos} \gets$ $m$-dimensional vector of uniform random numbers between $a$ and $b$.
        \State $g_{score} \gets \infty$

        \For{iteration less than $maxiter$}
            \For{each particle}
                \State $score \gets f(pos)$ 
                \If{$score < b_{score}$}
                    \State $b_{score} \gets score$
                    \State $b_{pos} \gets pos$
                \EndIf
                \If{$score < g_{score}$}
                    \State $g_{score} \gets score$
                    \State $g_{pos} \gets pos$
                \EndIf
            \EndFor
            \If{$g_{score} < tol$}
                \State break the loop
            \EndIf
            \For{each particle}
                \State $r_{1}, r_{2} \gets$ random numbers between $0$ and $1$.
                \State $vel \gets w(vel) + c_{1}r_{1}(b_{pos}-pos) + c_{2}r_{2}(g_{pos}-pos)$
                \State $pos \gets pos + vel$
            \EndFor
        \EndFor
    \end{algorithmic}
\end{algorithm}

\par
The next step is to combine two PSOs.
One of them will be looking for a parameter $p_b$ at which the saddle-node bifurcation occurs.
The other one will be looking for the periodic point $x_b$ corresponding to $p_b$.
Its worth noticing that the algorithm works for dynamical systems of arbitrary dimensions.
For that reason, the minimization functions are presented in a general form.
\par
Let $(\mathbb{R}^{m}, f_{p})$ be a DDS.
The first PSO, denoted as PSO1, is searching for a periodic point $x_0$ given some parameter $p_{1}$.
The PSO1 does this by minimizing the function $F_{loss}(x_0) = \norm{f^{n}_{p_{1}}(x_0)-x_0}$.
Furthermore, another PSO, denoted as PSO2, is searching for a parameter $p_{b}$ that corresponds to a saddle-node bifurcation.
It achieves this by minimizing the function $G_{loss}$~\eqref{eq:minimize_p}.
Note that $G_{loss}$ has a minimum of $0$ in the case where $p = p_{b}$ and $x_{0}$ is a periodic point corresponding to $p_{b}$.
The tolerance $tol$ in~\eqref{eq:minimize_p} specifies whether $F_{loss}$ is close enough to the solution.

\begin{equation}
\label{eq:minimize_p}
    G_{loss}(p) =
    \begin{cases}
        |\text{det}(Df^{n}_{p}(x_0)-I)| & \text{if } F_{loss}(x_0) < tol, \\
        \infty & \rm{otherwise}
    \end{cases}
\end{equation}

PSO2 has to evaluate its loss function $G_{loss}(p)$ during its runtime.
However, in order to evaluate $G_{loss}(p)$, it needs to know a periodic point $x_0$ of $f_{p}$.
To do that, it uses PSO1 and sets its parameter $p_{1} = p$.
Consequently, PSO1 converges to a periodic point of $f_{p}$, if there is any.
This nesting of two PSOs enables the PSO2 to converge to a saddle-node bifurcation parameter $p_{b}$.
\par
To initiate PSO2, information from Global Search can be used.
Suppose that $[p_{A}, p_{B}]$ is the interval identified during the Global Search of a DDS $(X, f_{p})$.
Furthermore, it is known that $f_{p_{A}}$ is non-periodic and $f_{p_{B}}$ is $m$-periodic.
Then, the parameters of PSO2 (Algorithm~\ref{alg:pso}) can be chosen as follows: $f = f_{p}$, $(a, b) = (p_{A}, p_{B})$, $n = 10$, $maxiter = 500$ and $tol = 10^{-6}$.
The parameters of PSO1 can be chosen equivalently, except that $(a, b)$ must be chosen with respect to $f_{p}$.
Since $X$ is invariant under $f_{p}$, $(a, b)$ can be chosen appropriately.
For example, for the Logistic map, $(a, b)$ of PSO1 can be set to $[0, 1]$.

\section{Coloring}
The goal of this phase is to color the neighborhood of the breakpoint to warn about its ambiguity.
The parameters at which the breakpoint occurs were found in the previous steps.
The characteristic relation of type-I intermittency is used to color the neighborhood of the breakpoint.

\subsection{Description of the algorithm}
The Local Search found parameters of the parameter space at which the breakpoint occurs.
Let $p_b$ be one of them.
It is known that the parameters on the left side of $p_b$ in its proximity exhibit intermittency.
Nevertheless, there is a need to quantify how far from $p_b$ is intermittency still occurring.
Additionally, it is needed to measure how prominent the intermittency is for some $p$ near $p_b$.
\par
A useful indicator of intermittent behavior is the average laminar length $l_{avg}$ described in Chapter~\ref{chap:type-I intermittency}.
The characteristic relation $l_{avg} \varpropto 1 / \sqrt{\varepsilon}$ can be used to estimate the average laminar length.
The characteristic relation depends on the local map around fixed points associated with saddle-node bifurcation.
Fortunately, these fixed points have been found in the Local Search.
Hence, by calculating $\varepsilon$ for each stable fixed point of $f_{p_b}$, $l_{avg}$ can be estimated for the laminar region around each fixed point.
Taking their average yields an estimate of the average laminar length for $p_b$.
\par
Depending on the parameter $p$, the laminar phases have various lengths.
Far to the left of $p_b$ the laminar phases are about $50$ iterations long.
Very close to the left of $p_b$ the length of the laminar phase can become arbitrarily long.
The closer the parameter $p$ gets to $p_b$, the longer the laminar phase.
\par
To colorize the diagram it is needed to determine the average laminar length for the parameters $p$ to the left of $p_b$ and compare them.
However, the average laminar lengths can get arbitrarily large.
It is convenient to color only the parameters $p$ that have average laminar lengths between $L$ and $U$. 
$L$ denotes the lower bound and $U$ denotes the upper bound.
For example, $L$ can be set to $100$ and $U$ to $1000$.
The next step is to find parameters $p_L$ and $p_U$ such that the parameters $p$ between $p_L$ and $p_U$ have $l_{avg}$ from the range $[ L, U ]$.
\par
The idea behind an algorithm to find $p_L$ and $p_U$ is to step to the left from $p_b$ with incrementally smaller steps.
The algorithm is illustrated in Algorithm~\ref{alg:optimal_bound}.

\begin{algorithm}[!h]
    \caption{Optimal bound search}
    \label{alg:optimal_bound}
    \begin{algorithmic}[1]
        \Statex $f \gets$ map
        \Statex $p \gets$ initial bound estimate
        \Statex $s \gets$ step
        \Statex $B \gets$ desired average laminar length
        \Statex $t \gets$ $B$ deviation tolerance
        \Statex $i_{m} \gets$ maximum of iterations
        \Statex $i_{c} \gets$ current iteration

        \For{$i$ from $i_{c}$ to $i_{m}$}
            \State $p \gets p - s$
            \State $l_{avg} \gets$ average laminar length for $p$
            \If{$B-t \leq l_{avg} \leq B+t$}
                \State Terminate program with $p$ as the optimal bound.
            \EndIf
            \If{$l_{avg} < B-t$}
                \State Repeat the program for $p = p+s$, $s = s/2$ and $i_{c} = i+1$. The rest of the parameters remain unchanged.
            \EndIf
        \EndFor
    \end{algorithmic}
\end{algorithm}

The Algorithm~\ref{alg:optimal_bound} can be used to find both $p_L$ and $p_U$.
The algorithm can be started with $p = p_b$, $s = 0.1$, $i_{m} = 200$, $i_{c} = 0$, $t = 10$.
The parameter $B$ can be chosen as $100$ when searching for $p_{L}$ and $1000$ when searching for $p_{U}$.
This way both $p_L$ and $p_U$ are obtained.
\par
To finish the coloring procedure, a bifurcation diagram is constructed.
A color range is defined for numbers in the range $[ l_{avg}$ for $p_L, l_{avg}$ for $p_U ]$.
The trajectory projections for the parameters $p \in [ p_L, p_U ]$ are colored using this color range.
A trajectory projection at the parameter $p$ is colored based on $l_{avg}$ corresponding to $p$.
The same coloring procedure is repeated for all breakpoint parameters found in the Local Search.
\par
The result of the coloring algorithm for a single breakpoint of the Logistic map is shown in Figure~\ref{fig:coloring_example}.
Note that in order to see the full range of colors, the bifurcation diagram has to be zoomed in to the breakpoint.

\begin{figure}[!h]
    \centering
    \includegraphics[width=0.95\textwidth]{Figures/logistic_map_coloring_example.png}
    \caption{
        Colorization of a single breakpoint of $\mathcal{L}_{r}$ for $r \in I = [ 3.6263, 3.6267 ]$.
        Algorithm~\ref{alg:optimal_bound} was used to specify the colorization bounds.
        Lower bound $L$ = $200$.
        Upper bound $U$ = $1000$.
        The bifurcation diagram consists of projections $\mathcal{T}_{900}^{1000}(\mathcal{L}_{r}, 0.5)$ for $r \in I$.
    }
    \label{fig:coloring_example}
\end{figure}

\section{Complete Algorithm}
In the previous sections, each part of the algorithm was described.
It was also pointed out how each part of the algorithm relates to the other parts.
For clarity, this section explains how to combine all the parts together.
\par
First, the Global Search is used to identify the intervals where a breakpoint could occur.
Secondly, the Local Search is employed for each of the intervals to find precise parameter, periodicity, and the fixed points of the breakpoint.
Two alternatives to Local Search were introduced.
Each of them can be used.
Lastly, areas next to the found breakpoint are colored using the Coloring algorithm.
The result is a bifurcation diagram with identified breakpoints whose left neighborhoods are colored.
\par
An example result of the complete algorithm is shown in Figure~\ref{fig:complete_colorization}.
Note that periodic windows (or equivalently breakpoints) that weren't visible at first are detected through this algorithm.

\begin{figure}[!h]
    \centering
    \includegraphics[width=0.95\textwidth]{Figures/complete_colorization.png}
    \caption{
        Full Colorization of $\mathcal{L}_{r}$ for $r \in [ 3.62, 3.65 ]$.
        The bifurcation diagram consists of projections of $\mathcal{T}_{900}^{1000}(\mathcal{L}_{r}, 0.5)$.
        Colorful lines: areas near the detected breakpoints (up to period $20$).
    }
    \label{fig:complete_colorization}
\end{figure}

\endinput
\chapter{Software}
\label{sec:software}

This chapter discusses the software used for numerical computations in this thesis.
A mention of the source code used for in this thesis is also included.

\section{Julia Programming Language}
Numerical simulations play a crucial role in the field of dynamical systems.
The analytical approach is often impossible, and numerical simulation is the only option.
In the past, it was common to use Fortran or C for fast numerical simulations.
However, these languages require a lot of effort and skill to write even simple programs.
Later, Matlab and Python became popular in the scientific world.
The ease of sharing code and writing human-readable programs made them a good choice for many scientists.
However, Matlab is proprietary software, and Python is an interpreted language which makes the computations slow.
\par
Julia is a high-level programming language with a syntax similar to Matlab and Python.
It aims to bridge the gap between dynamically typed languages like Python and statically typed languages like C.
The language is designed in a way that it allows you to write high-level code that is as fast as C or Fortran while being as readable as Python.
Code written in Julia resembles mathematical abstraction because of its use of generic types and multiple dispatch. 
Sharing code with other scientists is easy because Julia is open source, free, and has a large community. 
The ability to access low-level details enables the user to achieve maximum performance.
Other features include built-in package manager, meta-programming, optional typing, just-in-time compilation, and more.
It is also suitable for parallel and distributed computing~\cite{Bezanson2017,Bezanson20181024}.
\par
Thanks to its vibrant community of scientists and developers, Julia has a lot of packages for scientific computing.
One of the notable ones is the DynamicalSystems.jl open source library.
It provides a framework and efficient algorithms for computations with dynamical systems.
Its generic algorithms work both for discrete-time and continuous-time dynamical systems.
The library contains algorithms for many subfields of dynamical systems theory~\cite{Datseris2018}.
\par
The list of some of the features is presented in the following list~\cite{DynamicalSystems2024}:
\begin{itemize}
    \item Delay coordinates embedding.
    \item Poincaré surface of sections.
    \item Spectrum of Lyapunov exponents.
    \item Detecting and distinguishing chaos using the GALI method.
    \item Automated production of orbit diagrams for continuous systems.
    \item Generalized entropies and permutation entropy.
    \item Recurrence quantification analysis.
    \item 0-1 test for chaos.
    \item and much more...
\end{itemize}

\par
Julia is well suited for numerical simulations in the field of dynamical systems.
Such simulations often require many mathematical algorithms and methods.
Support for differential equation solvers, linear algebra solvers, automatic evaluation of Jacobians, root-finding algorithms, etc. are all needed.
The need for high-quality plotting libraries that are easy to use is also important.
Interactive development via REPL and Jupyter notebooks is also a big advantage.
One of the main advantages is the great support for profiling, benchmarking, and optimization of algorithms.
Every scientist without the skills of a computer scientist can track down bottlenecks and limit number of heap allocations.
Good support for multi-threading and parallel computing is also a big advantage.
% Many of the algorithms used in this thesis are easily parallelizable.
Julia can even run on a supercomputer~\cite{Regier2016-vq}.

\section{Implementation}
All data, plots, and results included in this thesis were generated using the Julia programming language.
The whole source code is available on Github \url{https://github.com/JonasKoziorek/thesis} and as an attachment to this thesis.
For the purposes of exploration of this thesis, I have created my own library called DDS.jl (Discrete Dynamical Systems).
The main features include the following.
\begin{itemize}
    \item Support of multidimensional discrete dynamical systems.
    \item Trajectory generation and its plots.
    \item Cobweb diagram generation.
    \item Bifurcation diagram generation.
    \item Automatic differentiation and Jacobians of an arbitrary system.
    \item Davidchack-Lai algorithm.
    \item Bu-Wang-Jiang algorithm.
    \item Global search, Local search and Colorization algorithms.
\end{itemize}
Many of the algorithms are parallelizable.
Bifurcation diagrams were generated in parallel by splitting the parameter space.
Local search was run in parallel by computing the left boundary estimation and the right boundary estimation separately.
Local search was also run in parallel for each interval found in the global search.
Global search was run in parallel by splitting up the parameter space.
\par
All plots were created using Julia plotting packages Makie.jl~\cite{Danisch2021} and Plots.jl~\cite{Christ2022}.
Vector graphics was created using the Luxor.jl package~\cite{Luxor2024}.

\endinput
\chapter{Conclusion}

The aim of this thesis was to develop an alorithm for automatic detection of type-I intermittency.
In the first chaper we have introduced the theory of discrete dynamical systems, Logistic map, and type-I intermittency.
Subsequently we have explained why it is important to detect type-I intermittency and what are the issues if we don't.

\endinput

\printbibliography[heading=bibintoc]

\appendix
% \chapter{Complex Figures}
\label{sec:Appendix_complex_figures}
In this appenix we present some general graphical representation of the discrete dynamical models discussed in this text.
In each of the figures we present several subfigures:
\begin{itemize}
  \item Bifurcation diagram for selected parameters and their ranges of interest
  \item Trajectory of the dynamics that exhibits chaotic behavior and its locatization
  \item Trajectory of the dynamics that exhibits periodic (or eventually periodic) behavior and its localization
  \item Trajectory of the dynamics that exhibits intermittent behavior and its localization
\end{itemize}

% \begin{figure}[!ht]
% 	\centering
% 	\includegraphics[width=1.0\textwidth]{DDS/Figures/pomeau_manneville_complex.png}
% 	\caption{Intermittency in Pomeau-Manneville map}
% 	\label{fig:complex_pomeau_manneville}
% \end{figure}

\begin{figure}[!ht]
	\centering
	\includegraphics[width=1.0\textwidth]{DDS/Figures/logistic_complex.png}
	\caption{Intermittency in logistic map}
	\label{fig:complex_logistic}
\end{figure}

% \begin{figure}[!ht]
% 	\centering
% 	\includegraphics[width=1.0\textwidth]{DDS/Figures/henon_complex.png}
% 	\caption{Intermittency in henon map}
% 	\label{fig:complex_henon}
% \end{figure}

% \begin{figure}[!ht]
% 	\centering
% 	\includegraphics[width=1.0\textwidth]{DDS/Figures/duffing_complex.png}
% 	\caption{Intermittency in duffing map}
% 	\label{fig:complex_duffing}
% \end{figure}

\endinput
% \chapter{Source Codes}
\label{sec:Appendix_codes}

In this appendix we present some source codes that were used for generating figures and simulations in this thesis.
The source code listing is not complete.
The whole code used for generating every plot in this DDS can be found at \url{https://github.com/JonasKoziorek/bachelors_thesis}

\lstinputlisting[label=general.jl,caption={Functions for iterating discrete dynamical system}]{DDS/src/general.jl}
\lstinputlisting[label=bifurcation.jl,caption={Functions for generating bifurcation diagrams}]{DDS/src/bifurcation.jl}
\lstinputlisting[label=cobweb.jl,caption={Functions for generating cobweb diagrams}]{DDS/src/cobweb.jl}
\lstinputlisting[label=trajectory.jl,caption={Functions for graphing iterations of dynamical systems}]{DDS/src/trajectory.jl}

\endinput

\end{document}